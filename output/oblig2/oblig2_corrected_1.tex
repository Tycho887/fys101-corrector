\documentclass[answers,a4paper,12pt]{exam}
\input{preamble.tex}

\title{{\bf{FYS101 Mekanikk}} \\ \Large{\answersareprinted Oblig XX}} 
\author{Institutt for Fysikk, REALTEK}
\date{Uke xx}

\begin{document}
\maketitle

\begin{questions}

\question Definer nøkkelbegreper i "Del A":
\begin{parts}
\part Definer referansesystem:
\begin{solution}
Et referansesystem er et system vi beskriver bevegelser i forhold til. Det gir oss en ramme for å måle og analysere bevegelse.
\end{solution}

\part Definer treghetssystem:
\begin{solution}
Et treghetssystem er et referansesystem som ikke er under akselerasjon. Det vil si at det beveger seg med konstant hastighet eller er i ro.
\end{solution}

\part Forklar uavhengighetsprinsippet:
\begin{solution}
Uavhengighetsprinsippet innebærer at man kan dekomponere vektorer og analysere hver retning for seg. Hver retning er uavhengig av hverandre, noe som betyr at bevegelse i én retning ikke påvirker bevegelse i en annen.
\end{solution}

\part Definer friksjonskraft:
\begin{solution}
Friksjonskraft er kraften mellom to objekter som er i kontakt med hverandre, som motvirker glidning mellom objektene. Den virker alltid i motsatt retning av bevegelsen.
\end{solution}
\end{parts}



\question Definer viktige konsepter i "Del B":
\begin{parts}
\part Definer friksjonskrefter:
\begin{solution}
Friksjonskrefter virker parallelt med kontaktoverflatene og har en retning som motarbeider glidning mellom objektene. De oppstår på grunn av uregelmessigheter i overflatene som er i kontakt, og kan deles inn i statisk friksjon (når objektene ikke beveger seg relativt til hverandre) og kinetisk friksjon (når objektene glir over hverandre).
\end{solution}
\end{parts}

\question FILL QUESTION
\begin{parts}
\part FILL QUESTION
\begin{solution}
PLACE GRAPHICS HERE
\end{solution}
\end{parts}



\question Hookes lov
\begin{parts}
\part Forklar Hookes lov:
\begin{solution}
Hookes lov beskriver hvordan kraften som virker på en fjær er proporsjonal med forskyvningen fra fjærens likevektsposisjon. Loven er gitt ved ligningen:
\[
F_{x} = -k x
\]
hvor \( k \) er fjærkonstanten, et mål på fjærens stivhet. En negativ verdi av \( x \) betyr at fjæren har blitt presset sammen en avstand \( |x| \) fra likevektsposisjonen. Det negative fortegnet i ligningen indikerer at når fjæren blir strukket eller presset sammen i en retning, virker kraften i motsatt retning.
\end{solution}

\part Beregn kraften som virker på en fjær med en fjærkonstant \( k = \SI{200}{\newton\per\meter} \) når den er strukket \SI{0.1}{\meter} fra likevektsposisjonen:
\begin{solution}
\centering
\begin{align}
F_{x} &= -k x \\
      &= -\SI{200}{\newton\per\meter} \times \SI{0.1}{\meter} \\
      &= \doubleunderline{-\SI{20}{\newton}}
\end{align}
Kraften som virker på fjæren er \SI{20}{\newton} i motsatt retning av forskyvningen.
\end{solution}

\part Illustrer kraften som virker på fjæren ved hjelp av en vektorgrafikk:
\begin{solution}
PLACE GRAPHICS HERE
\end{solution}
\end{parts}



\question Beregn snordraget $\vec{T}$ og normalkraften $\vec{F}_{n}$ for en kropp med følgende data:
\begin{align*}
m &= \SI{50}{\kilo\gram} \\
\theta &= 60^{\circ} \\
g &= \SI{9.81}{\meter\per\second\squared} \\
v &= 0
\end{align*}

\begin{parts}
\part Finn snordraget $\vec{T}$ og normalkraften $\vec{F}_{n}$:
\begin{solution}
Newton's 2. lov i $x$-retning:
\begin{align}
\sum F_{x} &= m a_{x} \quad \text{der } a_{x} = 0 \quad (v=0) \\
\sum F_{x} &= 0 \\
T - F_{g x} &= 0 \\
T &= F_{g x} \\
T &= F_{g} \cdot \sin \theta \\
T &= m \cdot g \cdot \sin \theta \\
T &= \SI{50}{\kilo\gram} \cdot \SI{9.81}{\meter\per\second\squared} \cdot \sin 60^{\circ} \\
T &= \doubleunderline{\SI{0.42}{\kilo\newton}}
\end{align}

Newton's 2. lov i $y$-retning:
\begin{align}
\sum F_{y} &= m a_{y} \quad \text{der } a_{y} = 0 \quad (v=0) \\
\sum F_{y} &= 0 \\
F_{n} - F_{g y} &= 0 \\
F_{n} &= F_{g y} \\
F_{n} &= F_{g} \cdot \cos \theta \\
F_{n} &= m \cdot g \cdot \cos \theta \\
F_{n} &= \SI{50}{\kilo\gram} \cdot \SI{9.81}{\meter\per\second\squared} \cdot \cos 60^{\circ} \\
F_{n} &= \doubleunderline{\SI{0.25}{\kilo\newton}}
\end{align}

Snordraget er \SI{0.42}{\kilo\newton} og normalkraften er \SI{0.25}{\kilo\newton}.
\end{solution}

\part Finn snordraget som en funksjon av $\theta$ og $m$:
\begin{solution}
Vi bruker resonementet fra deloppgave a):
\begin{align}
T &= m \cdot g \cdot \sin \theta \\
\text{Når } \theta &= 0^{\circ} \Rightarrow \sin 0^{\circ} = 0 \\
T &= 0
\end{align}

Når $\theta = 90^{\circ} \Rightarrow \sin 90^{\circ} = 1$
\begin{align}
T &= m \cdot g = F_{g}
\end{align}

Disse resultatene er begge rimelige.
\end{solution}
\end{parts}



\question Analyser kreftene som virker på en skråplan:
\begin{parts}
\part Newtons 2. lov i $y$-retningen:
\begin{solution}
Vi starter med å anvende Newtons 2. lov i $y$-retningen, hvor akselerasjonen $a_y = 0$ siden det ikke er noen bevegelse i denne retningen. Dette gir oss:
\begin{align}
\sum F_{y} &= m \cdot a_{y} \\
0 &= F_{n} - F_{g_y} \\
F_{n} &= F_{g_y}
\end{align}
Her er $F_{n}$ den normale kraften og $F_{g_y}$ er komponenten av tyngdekraften i $y$-retningen.
\end{solution}

\part Beregn den normale kraften $F_n$:
\begin{solution}
Den normale kraften $F_n$ kan uttrykkes som:
\begin{align}
F_{n} &= F_{g} \cdot \cos \theta \\
F_{n} &= m \cdot g \cdot \cos \theta
\end{align}
Hvor $m = \SI{65}{\kilo\gram}$, $g = \SI{9.81}{\meter\per\second\squared}$, og $\theta = 30^\circ$. Vi setter inn verdiene:
\begin{align}
F_{n} &= \SI{65}{\kilo\gram} \cdot \SI{9.81}{\meter\per\second\squared} \cdot \cos 30^\circ \\
F_{n} &= \SI{65}{\kilo\gram} \cdot \SI{9.81}{\meter\per\second\squared} \cdot \frac{\sqrt{3}}{2} \\
F_{n} &= \SI{552.5}{\newton} \\
F_{n} &= \doubleunderline{\SI{0.55}{\kilo\newton}}
\end{align}
Den beregnede normale kraften er \SI{0.55}{\kilo\newton}.
\end{solution}
\end{parts}



\question FILL QUESTION
\begin{parts}

\part Del A: Finn $a_{x}$ og $T$ uttrykt ved $\theta, m_{1}, g, m_{2}$

\begin{solution}
Newtons 2. lov for $m_{1}$ i $x$-retning:

\begin{align}
\sum F_{1 x} &= m_{1} a_{1 x} \\
T - F_{g x 1} &= m_{1} a_{x} \\
T &= m_{1} a_{x} + F_{g x 1} \\
T &= m_{1} a_{x} + m_{1} \cdot g \cdot \sin \theta
\end{align}

Newtons 2. lov for $m_{2}$ i $x$-retning:

\begin{align}
\sum F_{2 x} &= m_{2} a_{2 x} \\
F_{g 2} - T &= m_{2} a_{x} \\
T &= F_{g 2} - m_{2} a_{x} \\
T &= m_{2} \cdot g - m_{2} a_{x}
\end{align}

Vi har nå to uttrykk for $T$. Vi setter de lik hverandre og løser for $a_{x}$:

\begin{align}
m_{1} a_{x} + m_{1} g \sin \theta &= m_{2} g - m_{2} a_{x} \\
m_{1} a_{x} + m_{2} a_{x} &= m_{2} g - m_{1} g \sin \theta \\
a_{x}(m_{1} + m_{2}) &= g(m_{2} - m_{1} \sin \theta) \\
a_{x} &= \frac{g(m_{2} - m_{1} \sin \theta)}{m_{1} + m_{2}}
\end{align}

Vi kan nå sette $a_{x}$ inn i en av de to uttrykkene for $T$ og løse for $T$. Vi setter inn i den andre:

\begin{align}
T &= m_{2} g - m_{2} a_{x} \\
T &= m_{2} g - m_{2} \cdot \frac{g(m_{2} - m_{1} \sin \theta)}{m_{1} + m_{2}} \\
T &= \frac{m_{2}(m_{1} + m_{2}) - m_{2}(m_{2} - m_{1} \sin \theta)}{m_{1} + m_{2}} g \\
T &= \frac{m_{2}(m_{1} + m_{2} - m_{2} + m_{1} \sin \theta)}{m_{1} + m_{2}} g
\end{align}

\begin{align}
T &= \frac{m_{1} m_{2}(1 + \sin \theta)}{m_{1} + m_{2}} g
\end{align}
\end{solution}

\part Del B: $m_{1} = m_{2} = m = \SI{5.0}{\kilo\gram}, \theta = 30^{\circ}$

Finn $a_{x}$:

\begin{solution}
\begin{align}
a_{x} &= \frac{g(m_{2} - m_{1} \sin \theta)}{m_{1} + m_{2}} \\
a_{x} &= \frac{g(m - m \sin \theta)}{m + m} \\
a_{x} &= \frac{mg(1 - \sin \theta)}{2m} \\
a_{x} &= \frac{1}{2} g(1 - \sin \theta) \\
a_{x} &= \frac{1}{2} \cdot \SI{9.81}{\meter\per\second\squared} \cdot (1 - \sin 30^{\circ}) \\
a_{x} &= \doubleunderline{\SI{2.5}{\meter\per\second\squared}}
\end{align}
\end{solution}

Finn $T$:

\begin{solution}
\begin{align}
T &= \frac{m_{1} m_{2}(1 + \sin \theta)}{m_{1} + m_{2}} g \\
T &= \frac{m \cdot m(1 + \sin \theta)}{m + m} g \\
T &= \frac{1}{2} mg(1 + \sin \theta) \\
T &= \frac{1}{2} \cdot \SI{5.0}{\kilo\gram} \cdot \SI{9.81}{\meter\per\second\squared} \cdot (1 + \sin 30^{\circ}) \\
T &= \doubleunderline{\SI{37}{\newton}}
\end{align}
\end{solution}

Akselerasjonen er \SI{2.5}{\meter\per\second\squared} og snordraget er \SI{37}{\newton}.
\end{parts}

\end{questions}
\end{document}
