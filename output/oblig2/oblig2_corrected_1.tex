\documentclass[answers,a4paper,12pt]{exam}
\input{preamble.tex}

\title{{\bf{FYS101 Mekanikk}} \\ \Large{\answersareprinted Oblig XX}} 
\author{Institutt for Fysikk, REALTEK}
\date{Uke xx}

\begin{document}
\maketitle

\begin{questions}
\question Definer nøkkelbegreper i "Del A":
\begin{parts}
\part Definer referansesystem:
\begin{solution}
Et referansesystem er et system vi beskriver bevegelser i forhold til. Det er en ramme eller et perspektiv som brukes for å observere og måle posisjoner og bevegelser av objekter.
\end{solution}

\part Definer treghetssystem:
\begin{solution}
Et treghetssystem er et referansesystem som ikke er under akselerasjon. Det vil si at det beveger seg med konstant hastighet eller er i ro, og Newtons lover gjelder uendret.
\end{solution}

\part Definer uavhengighetsprinsippet:
\begin{solution}
Uavhengighetsprinsippet innebærer at man kan dekomponere vektorer og analysere hver retning for seg. Hver retning er uavhengig av hverandre, noe som betyr at bevegelse i én retning ikke påvirker bevegelse i en annen retning.
\end{solution}

\part Definer friksjonskraft:
\begin{solution}
Friksjonskraft er kraften mellom to objekter som er i kontakt med hverandre, som motvirker glidning mellom objektene. Den virker i motsatt retning av bevegelsen eller den potensielle bevegelsen.
\end{solution}
\end{parts}

\question FILL QUESTION
\begin{parts}

\part Del B: Forklar friksjonskrefter:
\begin{solution}
Friksjonskrefter virker parallelt med kontaktoverflatene mellom to objekter. Disse kreftene har en retning som motarbeider glidning mellom objektene. Friksjon kan deles inn i statisk friksjon, som hindrer bevegelse, og kinetisk friksjon, som virker når objektene allerede glir over hverandre.
\end{solution}

\part Del C: FILL QUESTION
\begin{solution}
FILL SOLUTION
\end{solution}

\end{parts}

\question Hookes lov

Når en fjær blir strukket eller presset sammen fra sin likevektsposisjon, er kraften den utgjør gitt ved

\begin{align}
F_{x} = -k x
\end{align}

der $k$ er fjærkonstanten, et mål på fjærens stivhet. En negativ verdi av $x$ betyr at fjæren har blitt presset sammen en avstand $|x|$ fra likevektsposisjonen. Det negative fortegnet i likningen betyr at når fjæren blir strukket eller presset sammen i en retning, virker kraften i motsatt retning.

\begin{parts}

\part Forklar hva fjærkonstanten $k$ representerer:
\begin{solution}
Fjærkonstanten $k$ er en konstant som beskriver stivheten til en fjær. Den angir hvor mye kraft som kreves for å strekke eller komprimere fjæren med en enhetslengde. Jo høyere verdi av $k$, desto stivere er fjæren.
\end{solution}

\part Beregn kraften som virker på en fjær med fjærkonstant $k = \SI{150}{\newton\per\meter}$ når den er strukket \SI{0.2}{\meter} fra likevektsposisjonen:
\begin{solution}
\centering
\begin{align}
F_{x} &= -k x \\
      &= -\SI{150}{\newton\per\meter} \times \SI{0.2}{\meter} \\
      &= \doubleunderline{-\SI{30}{\newton}}
\end{align}
Kraften som virker på fjæren er \SI{30}{\newton} i motsatt retning av strekkretningen.
\end{solution}

\part Illustrer kraften som virker på fjæren ved hjelp av en vektorgrafikk:
\begin{solution}
Placeholder for graphics: \includegraphics[width=0.5\linewidth]{path/to/image}
\end{solution}

\end{parts}

\question FILL QUESTION
\begin{parts}

\part Part A: Finn snordraget $\vec{T}$ og normalkraften $\vec{F}_{n}$.
\begin{solution}
Vi bruker Newtons 2. lov i $x$-retning:

\begin{align}
\sum F_{x} &= m \cdot a_{x} \quad \text{hvor} \quad a_{x} = 0 \quad (v = 0) \\
\sum F_{x} &= 0 \\
T - F_{gx} &= 0 \\
T &= F_{gx} \\
T &= F_{g} \cdot \sin \theta \\
T &= m \cdot g \cdot \sin \theta \\
T &= 50 \, \mathrm{kg} \cdot 9,81 \, \frac{\mathrm{m}}{\mathrm{s}^{2}} \cdot \sin 60^{\circ} \\
T &= \doubleunderline{\SI{0.42}{\kilo\newton}}
\end{align}

For normalkraften $\vec{F}_{n}$, bruker vi Newtons 2. lov i $y$-retning:

\begin{align}
\sum F_{y} &= m \cdot a_{y} \quad \text{hvor} \quad a_{y} = 0 \quad (v = 0) \\
\sum F_{y} &= 0 \\
F_{n} - F_{gy} &= 0 \\
F_{n} &= F_{gy} \\
F_{n} &= F_{g} \cdot \cos \theta \\
F_{n} &= m \cdot g \cdot \cos \theta \\
F_{n} &= 50 \, \mathrm{kg} \cdot 9,81 \, \frac{\mathrm{m}}{\mathrm{s}^{2}} \cdot \cos 60^{\circ} \\
F_{n} &= \doubleunderline{\SI{0.25}{\kilo\newton}}
\end{align}

Snordraget er \SI{0.42}{\kilo\newton} og normalkraften er \SI{0.25}{\kilo\newton}.
\end{solution}

\part Part B: Finn snordraget som en funksjon av $\theta$ og $m$.
\begin{solution}
Vi bruker resonneringen fra deloppgave a):

\begin{align}
T &= m \cdot g \cdot \sin \theta
\end{align}

Når $\theta = 0^{\circ}$, har vi:

\begin{align}
\sin 0^{\circ} &= 0 \\
T &= 0
\end{align}

Når $\theta = 90^{\circ}$, har vi:

\begin{align}
\sin 90^{\circ} &= 1 \\
T &= m \cdot g = F_{g}
\end{align}

Disse resultatene er begge rimelige.
\end{solution}

\end{parts}


\question FILL QUESTION
\begin{parts}

\part Newtons 2. Lov i $y$-retningen:
\begin{solution}
Vi starter med Newtons 2. lov for krefter i $y$-retningen, hvor akselerasjonen $a_y = 0$ fordi det ikke er noen bevegelse i vertikal retning. Dette gir oss:

\begin{align}
\sum F_{y} &= m \cdot a_{y} \\
0 &= F_{n} - F_{g_y} \\
F_{n} &= F_{g_y}
\end{align}

Her er $F_{n}$ den normale kraften og $F_{g_y}$ er den vertikale komponenten av tyngdekraften.
\end{solution}

\part Beregning av den normale kraften $F_{n}$:
\begin{solution}
Den normale kraften kan uttrykkes som:

\begin{align}
F_{n} &= F_{g} \cdot \cos \theta \\
F_{n} &= m \cdot g \cdot \cos \theta \\
F_{n} &= \SI{65}{\kilo\gram} \cdot \SI{9.81}{\meter\per\second\squared} \cdot \cos 30^{\circ} \\
F_{n} &= \SI{65}{\kilo\gram} \cdot \SI{9.81}{\meter\per\second\squared} \cdot \frac{\sqrt{3}}{2} \\
F_{n} &= \SI{552.5}{\newton} \\
F_{n} &= \doubleunderline{\SI{0.55}{\kilo\newton}}
\end{align}

Derfor er den normale kraften $F_{n}$ lik \SI{0.55}{\kilo\newton}.
\end{solution}

\end{parts}


\question FILL QUESTION
\begin{parts}

\part Del A: Finn $a_{x}$ og $T$ uttrykt ved $\theta, m_{1}, g, m_{2}$

\begin{solution}
Vi starter med Newtons 2. lov for $m_{1}$ i $x$-retning:

\begin{align}
\sum F_{1x} &= m_{1} a_{1x} \\
T - F_{gx1} &= m_{1} a_{x} \\
T &= m_{1} a_{x} + m_{1} \cdot g \cdot \sin \theta
\end{align}

For $m_{2}$ i $x$-retning har vi:

\begin{align}
\sum F_{2x} &= m_{2} a_{2x} \\
F_{g2} - T &= m_{2} a_{x} \\
T &= F_{g2} - m_{2} a_{x} \\
T &= m_{2} \cdot g - m_{2} a_{x}
\end{align}

Vi har nå to uttrykk for $T$. Vi setter de lik hverandre og løser for $a_{x}$:

\begin{align}
m_{1} a_{x} + m_{1} g \sin \theta &= m_{2} g - m_{2} a_{x} \\
m_{1} a_{x} + m_{2} a_{x} &= m_{2} g - m_{1} g \sin \theta \\
a_{x}(m_{1} + m_{2}) &= g(m_{2} - m_{1} \sin \theta) \\
a_{x} &= \frac{g(m_{2} - m_{1} \sin \theta)}{m_{1} + m_{2}}
\end{align}

Nå kan vi sette $a_{x}$ inn i et av uttrykkene for $T$ og løse for $T$:

\begin{align}
T &= m_{2} g - m_{2} a_{x} \\
T &= m_{2} g - m_{2} \cdot \frac{g(m_{2} - m_{1} \sin \theta)}{m_{1} + m_{2}} \\
T &= \frac{m_{2}(m_{1} + m_{2}) - m_{2}(m_{2} - m_{1} \sin \theta)}{m_{1} + m_{2}} g \\
T &= \frac{m_{2}(m_{1} + m_{2} - m_{2} + m_{1} \sin \theta)}{m_{1} + m_{2}} g \\
T &= \frac{m_{1} m_{2}(1 + \sin \theta)}{m_{1} + m_{2}} g
\end{align}
\end{solution}

\part Del B: Når $m_{1} = m_{2} = m = \SI{5.0}{\kilogram}$ og $\theta = 30^{\circ}$, finn $a_{x}$

\begin{solution}
\begin{align}
a_{x} &= \frac{g(m_{2} - m_{1} \sin \theta)}{m_{1} + m_{2}} \\
a_{x} &= \frac{g(m - m \sin \theta)}{m + m} \\
a_{x} &= \frac{mg(1 - \sin \theta)}{2m} \\
a_{x} &= \frac{1}{2} g(1 - \sin \theta) \\
a_{x} &= \frac{1}{2} \cdot \SI{9.81}{\meter\per\second\squared} \cdot (1 - \sin 30^{\circ}) \\
a_{x} &= \doubleunderline{\SI{2.5}{\meter\per\second\squared}}
\end{align}
\end{solution}

\part Finn $T$ for samme verdier

\begin{solution}
\begin{align}
T &= \frac{m_{1} m_{2}(1 + \sin \theta)}{m_{1} + m_{2}} g \\
T &= \frac{m \cdot m(1 + \sin \theta)}{m + m} g \\
T &= \frac{1}{2} mg(1 + \sin \theta) \\
T &= \frac{1}{2} \cdot \SI{5.0}{\kilogram} \cdot \SI{9.81}{\meter\per\second\squared} \cdot (1 + \sin 30^{\circ}) \\
T &= \doubleunderline{\SI{37}{\newton}}
\end{align}
\end{solution}

\end{parts}

Akselerasjonen er \SI{2.5}{\meter\per\second\squared} og snordraget er \SI{37}{\newton}.
\end{questions}
\end{document}
