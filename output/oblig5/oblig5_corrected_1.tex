\documentclass[answers,a4paper,12pt]{exam}
\input{preamble.tex}

\title{{\bf{FYS101 Mekanikk}} \\ \Large{\answersareprinted Oblig XX}} 
\author{Institutt for Fysikk, REALTEK}
\date{Uke xx}

\begin{document}
\maketitle

\begin{questions}
\question FILL QUESTION
\begin{parts}

\part Del A: Forklar begrepet "kraftstøt" og hvordan det måles i både styrke og retning:
\begin{solution}
Kraftstøt, eller impuls, er en vektor som representerer endringen i bevegelsesmengde forårsaket av en kraft som virker over en tidsperiode. Det måles ved å integrere kraften over tiden den virker. Styrken av kraftstøtet er lik størrelsen på denne integralen, mens retningen er gitt av retningen til kraften.
\end{solution}

\part Beregn kraftstøt gitt en konstant kraft:
\begin{solution}
Anta at en konstant kraft $\vec{F}$ virker over en tidsperiode fra $t_i$ til $t_f$. Kraftstøtet $\vec{I}$ kan beregnes som:
\begin{align}
\vec{I} &= \int_{t_i}^{t_f} \vec{F} \, dt \\
        &= \vec{F} \cdot (t_f - t_i)
\end{align}
Hvis $\vec{F} = \SI{10}{\newton}$ og virker fra $t_i = \SI{0}{\second}$ til $t_f = \SI{5}{\second}$, da er:
\begin{align}
\vec{I} &= \SI{10}{\newton} \cdot (\SI{5}{\second} - \SI{0}{\second}) \\
        &= \doubleunderline{\SI{50}{\newton\second}}
\end{align}
\end{solution}

\part Forklar betydningen av retningen i vektoren $\vec{I}$:
\begin{solution}
Retningen til kraftstøtvektoren $\vec{I}$ er den samme som retningen til kraften $\vec{F}$. Dette betyr at endringen i bevegelsesmengde skjer i samme retning som kraften virker. Hvis kraften er i positiv x-retning, vil også bevegelsesmengdeendringen være i positiv x-retning.
\end{solution}

\end{parts}

\question FILL QUESTION
\begin{parts}
\part Forklar hva som menes med et elastisk støt:
\begin{solution}
Et elastisk støt er en kollisjon der den totale kinetiske energien til systemet er bevart før og etter støtet. Dette betyr at ingen kinetisk energi går tapt til andre former for energi, som varme eller lyd, under kollisjonen. Bevegelsesmengden er også bevart i et elastisk støt.
\end{solution}

\part Forklar hva som menes med et uelastisk støt:
\begin{solution}
Et uelastisk støt er en kollisjon der den totale kinetiske energien ikke er bevart. En del av den kinetiske energien blir omdannet til andre former for energi, som varme eller lyd. Til tross for dette er bevegelsesmengden fortsatt bevart i et uelastisk støt.
\end{solution}
\end{parts}


\question FILL QUESTION
\begin{parts}

\part Part A: Beregn sluttfarten etter kollisjonen
\begin{solution}
Vi har en fullstendig elastisk kollisjon hvor bevegelsesmengden er bevart. Gitt:

\begin{align*}
m_1 &= m_2 = \SI{2000}{\kilo\gram} \\
v_{1i} &= \SI{30}{\meter\per\second} \\
v_{2i} &= \SI{10}{\meter\per\second}
\end{align*}

Bevegelsesmengde før og etter kollisjonen:

\begin{align*}
P_{1i} + P_{2i} &= P_f \\
m v_{1i} + m v_{2i} &= 2m v_f
\end{align*}

Løser for $v_f$:

\begin{align*}
v_f &= \frac{v_{1i} + v_{2i}}{2} \\
    &= \frac{\SI{30}{\meter\per\second} + \SI{10}{\meter\per\second}}{2} \\
v_f &= \doubleunderline{\SI{20}{\meter\per\second}}
\end{align*}

Den felles farten etter kollisjonen er \SI{20}{\meter\per\second}.
\end{solution}

\part Part B: Beregn andelen av kinetisk energi som tapes
\begin{solution}
Vi finner uttrykk for den kinetiske energien til systemet før og etter kollisjonen:

\begin{align*}
K_{\text{sys }i} &= K_{1i} + K_{2i} \\
                 &= \frac{1}{2} m v_{1i}^2 + \frac{1}{2} m v_{2i}^2 \\
                 &= \frac{1}{2} m \left(v_{1i}^2 + v_{2i}^2\right) \\
K_{\text{sys }f} &= \frac{1}{2}(2m) v_f^2 \\
                 &= m v_f^2
\end{align*}

Andelen av kinetisk energi som tapes:

\begin{align*}
\text{Andel tapt} &= \frac{\Delta K}{K_{\text{sys }i}} \\
                  &= \frac{K_{\text{sys }i} - K_{\text{sys }f}}{K_{\text{sys }i}} \\
                  &= \frac{\frac{1}{2} m \left(v_{1i}^2 + v_{2i}^2\right) - m v_f^2}{\frac{1}{2} m \left(v_{1i}^2 + v_{2i}^2\right)} \\
                  &= \frac{\frac{1}{2} \left(v_{1i}^2 + v_{2i}^2\right) - v_f^2}{\frac{1}{2} \left(v_{1i}^2 + v_{2i}^2\right)} \\
                  &= 1 - \frac{2 v_f^2}{v_{1i}^2 + v_{2i}^2} \\
                  &= 1 - \frac{2 \cdot (\SI{20}{\meter\per\second})^2}{(\SI{30}{\meter\per\second})^2 + (\SI{10}{\meter\per\second})^2} \\
                  &= \doubleunderline{0,20}
\end{align*}

\SI{20}{\percent} av den kinetiske energien tapes og omdannes til andre energiformer som varme.
\end{solution}

\end{parts}


\question FILL QUESTION
\begin{parts}

\part Part A: Beregn hastigheten $v_1$ etter kollisjonen:
\begin{solution}
Vi starter med å bruke den gitte ligningen for å finne $v_1$:

\begin{align}
\cos \theta_{1} &= \frac{v_{1}}{v_{i1}} \\
v_{1} &= v_{i1} \cos \theta_{1} \\
&= \SI{2.0}{\meter} \cdot \cos 30^{\circ} \\
&= \SI{2.0}{\meter} \cdot \frac{\sqrt{3}}{2} \\
&= \SI{1.732}{\meter}
\end{align}

Dermed er hastigheten $v_1 = \doubleunderline{\SI{1.732}{\meter\per\second}}$.
\end{solution}

\part Part B: Beregn hastigheten $v_2$ ved hjelp av arbeid-energi-setningen:
\begin{solution}
Vi bruker arbeid-energi-setningen for å finne $v_2$:

\begin{align}
K_{i1} + K_{i2} &= K_{1} + K_{2} \\
\frac{1}{2} m v_{i1}^{2} + \frac{1}{2} m v_{i2}^{2} &= \frac{1}{2} m v_{1}^{2} + \frac{1}{2} m v_{2}^{2} \\
v_{2} &= \sqrt{v_{i1}^{2} - v_{1}^{2}} \\
&= \sqrt{\left(\SI{2.0}{\meter\per\second}\right)^{2} - \left(\SI{1.732}{\meter\per\second}\right)^{2}} \\
&= \sqrt{4 - 3} \\
&= \SI{1.0}{\meter\per\second}
\end{align}

Dermed er hastigheten $v_2 = \doubleunderline{\SI{1.0}{\meter\per\second}}$.
\end{solution}

\part Part C: Kommenter på kollisjonens natur:
\begin{solution}
Støtet er dermed elastisk, siden den kinetiske energien er bevart gjennom kollisjonen. Dette betyr at ingen energi går tapt til deformasjon eller varme, og begge legemene fortsetter å bevege seg med hastigheter $v_1 = \SI{1.732}{\meter\per\second}$ og $v_2 = \SI{1.0}{\meter\per\second}$ i henholdsvis $30^{\circ}$ og $60^{\circ}$ i forhold til horisontalen.
\end{solution}

\end{parts}


\question FILL QUESTION
\begin{parts}

\part Del A: Beregn friksjonsarbeidet når klossene bremser opp.
\begin{solution}
Friksjonsarbeidet, $W_f$, er gitt ved:
\begin{align}
W_f &= -f \cdot x \\
f &= \mu \cdot F_n
\end{align}
Ved å bruke Newtons andre lov i y-retning, finner vi $F_n$:
\begin{align}
\sum F_y &= 0 \\
F_n - F_g &= 0 \\
F_n &= (m + M) \cdot g
\end{align}
Derfor blir friksjonsarbeidet:
\begin{align}
W_f &= -\mu \cdot (m + M) \cdot g \cdot x
\end{align}
\end{solution}

\part Del B: Bruk arbeid-energi-likningen for å finne et uttrykk for $V_0$.
\begin{solution}
Arbeid-energi-likningen gir:
\begin{align}
E_1 &= E_0 + W_f \\
K_1 &= K_0 + W_f \\
\frac{1}{2}(m+M) V_1^2 &= \frac{1}{2}(m+M) V_0^2 - \mu (m+M) g x
\end{align}
Ved å løse for $V_0$, får vi:
\begin{align}
\frac{1}{2} V_0^2 &= \mu g x \\
V_0 &= \sqrt{2 \mu g x}
\end{align}
\end{solution}

\part Del C: Beregn $v_0$ for et elastisk støt.
\begin{solution}
For et elastisk støt gjelder $p_1 = p_0$:
\begin{align}
(m+M) V_0 &= m v_0 + M V_1
\end{align}
Løsning for $v_0$ gir:
\begin{align}
v_0 &= \frac{m+M}{m} V_0 \\
&= \frac{0.400 \, \mathrm{kg} + 13 \, \mathrm{kg}}{0.400 \, \mathrm{kg}} \cdot 1.085 \, \mathrm{m/s} \\
v_0 &= \doubleunderline{\SI{36}{\meter\per\second}}
\end{align}
\end{solution}

\end{parts}

\end{questions}
\end{document}
