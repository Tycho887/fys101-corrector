\documentclass[answers,a4paper,12pt]{exam}
\input{preamble.tex}

\title{{\bf{FYS101 Mekanikk}} \\ \Large{\answersareprinted Oblig XX}} 
\author{Institutt for Fysikk, REALTEK}
\date{Uke xx}

\begin{document}
\maketitle

\begin{questions}

\question FILL QUESTION
\begin{parts}
\part Forklar hva som menes med en kraftstøt:
\begin{solution}
En kraftstøt er en endring i bevegelsesmengde som oppstår når en kraft virker på et objekt over en tidsperiode. Det kan uttrykkes som integralet av kraften over tid:
\begin{align}
\vec{I} &= \int_{t_{i}}^{t_{f}} \vec{F} \, dt
\end{align}
Her representerer \(\vec{I}\) impuls, \(\vec{F}\) er kraften, og \(t_i\) og \(t_f\) er henholdsvis start- og sluttidspunktet for kraftens virkning.
\end{solution}

\part Mål både styrke og retning av en kraftstøt:
\begin{solution}
For å måle både styrken og retningen av en kraftstøt, må vi beregne størrelsen av vektoren \(\vec{I}\) og dens retning. Dette kan gjøres ved å evaluere integralet:
\begin{align}
\vec{I} &= \int_{t_{i}}^{t_{f}} \vec{F} \, dt
\end{align}
Styrken til kraftstøtet er gitt ved størrelsen av \(\vec{I}\), mens retningen er gitt av retningen til vektoren \(\vec{I}\).
\end{solution}
\end{parts}


\question Definer og beregn elastisk og uelastisk støt:
\begin{parts}

\part Definer elastisk støt:
\begin{solution}
Et elastisk støt er en kollisjon der den totale kinetiske energien er bevart. Dette betyr at summen av den kinetiske energien til de involverte objektene før kollisjonen er lik summen av den kinetiske energien etter kollisjonen. Bevegelsesmengden er også bevart i et elastisk støt.
\end{solution}

\part Definer uelastisk støt:
\begin{solution}
Et uelastisk støt er en kollisjon der den totale kinetiske energien ikke er bevart. En del av den kinetiske energien blir omdannet til andre energiformer, som varme eller lyd. Bevegelsesmengden er imidlertid fortsatt bevart i et uelastisk støt.
\end{solution}

\part Beregn bevegelsesmengden for et elastisk støt:
\begin{solution}
Anta at vi har to objekter med masser \( m_1 \) og \( m_2 \), og hastigheter før kollisjonen \( v_{1i} \) og \( v_{2i} \). Etter kollisjonen har de hastigheter \( v_{1f} \) og \( v_{2f} \). Bevegelsesmengden er bevart, så vi har:
\begin{align}
m_1 v_{1i} + m_2 v_{2i} &= m_1 v_{1f} + m_2 v_{2f}
\end{align}
\end{solution}

\part Beregn den kinetiske energien for et elastisk støt:
\begin{solution}
Den totale kinetiske energien før og etter kollisjonen er lik:
\begin{align}
\frac{1}{2} m_1 v_{1i}^2 + \frac{1}{2} m_2 v_{2i}^2 &= \frac{1}{2} m_1 v_{1f}^2 + \frac{1}{2} m_2 v_{2f}^2
\end{align}
\end{solution}

\part Beregn bevegelsesmengden for et uelastisk støt:
\begin{solution}
For et uelastisk støt, bevegelsesmengden er fortsatt bevart:
\begin{align}
m_1 v_{1i} + m_2 v_{2i} &= (m_1 + m_2) v_f
\end{align}
hvor \( v_f \) er den felles hastigheten etter kollisjonen.
\end{solution}

\part Beregn den kinetiske energien for et uelastisk støt:
\begin{solution}
Den totale kinetiske energien er ikke bevart, men vi kan beregne den tapte energien:
\begin{align}
\Delta KE &= \left(\frac{1}{2} m_1 v_{1i}^2 + \frac{1}{2} m_2 v_{2i}^2\right) - \left(\frac{1}{2} (m_1 + m_2) v_f^2\right)
\end{align}
\end{solution}

\end{parts}


\question Beregn hastigheten og kinetisk energi for et system etter en kollisjon:
\begin{parts}

\part Del A: Beregn sluttfarten etter en fullstendig elastisk kollisjon
\begin{solution}
Gitt data:
\begin{align}
m_1 &= m_2 = \SI{2000}{\kilo\gram} \\
v_{1i} &= \SI{30}{\meter\per\second} \\
v_{2i} &= \SI{10}{\meter\per\second}
\end{align}

For en fullstendig elastisk kollisjon, er bevegelsesmengden bevart:
\begin{align}
P_{1i} + P_{2i} &= P_f \\
m v_{1i} + m v_{2i} &= 2m v_f
\end{align}

Løsning for $v_f$:
\begin{align}
v_f &= \frac{v_{1i} + v_{2i}}{2} \\
    &= \frac{\SI{30}{\meter\per\second} + \SI{10}{\meter\per\second}}{2} \\
v_f &= \doubleunderline{\SI{20}{\meter\per\second}}
\end{align}
Den felles hastigheten etter kollisjonen er \SI{20}{\meter\per\second}.
\end{solution}

\part Del B: Beregn andelen av kinetisk energi som tapes i kollisjonen
\begin{solution}
Kinetisk energi før kollisjonen:
\begin{align}
K_{\text{sys } i} &= K_{1i} + K_{2i} \\
                  &= \frac{1}{2} m v_{1i}^2 + \frac{1}{2} m v_{2i}^2 \\
                  &= \frac{1}{2} m \left(v_{1i}^2 + v_{2i}^2\right)
\end{align}

Kinetisk energi etter kollisjonen:
\begin{align}
K_{\text{sys } f} &= \frac{1}{2} (2m) v_f^2 \\
                  &= m v_f^2
\end{align}

Andelen av kinetisk energi som tapes:
\begin{align}
\text{Andel tap} &= \frac{\Delta K}{K_{\text{sys } i}} \\
                 &= \frac{K_{\text{sys } i} - K_{\text{sys } f}}{K_{\text{sys } i}} \\
                 &= \frac{\frac{1}{2} m \left(v_{1i}^2 + v_{2i}^2\right) - m v_f^2}{\frac{1}{2} m \left(v_{1i}^2 + v_{2i}^2\right)}
\end{align}

Forenkling:
\begin{align}
&= \frac{\frac{1}{2} \left(v_{1i}^2 + v_{2i}^2\right) - v_f^2}{\frac{1}{2} \left(v_{1i}^2 + v_{2i}^2\right)} \\
&= 1 - \frac{2 v_f^2}{v_{1i}^2 + v_{2i}^2} \\
&= 1 - \frac{2 \cdot (\SI{20}{\meter\per\second})^2}{(\SI{30}{\meter\per\second})^2 + (\SI{10}{\meter\per\second})^2} \\
&= \doubleunderline{0,20}
\end{align}
20\% av den kinetiske energien tapes. Energien omdannes til andre energiformer som varme.
\end{solution}

\end{parts}



\question FILL QUESTION
\begin{parts}

\part Part A: Beregn hastigheten $v_1$ etter kollisjonen:
\begin{solution}
Vi starter med å bruke cosinus for å finne komponenten av hastigheten i retning $\theta_1$:
\begin{align}
u_1 &= v_{i1} \cos \theta_1 \\
    &= \SI{2.0}{\meter} \cdot \cos 30^{\circ} \\
    &= \SI{1.732}{\meter}
\end{align}
Derfor er hastigheten $v_1$ etter kollisjonen \doubleunderline{\SI{1.732}{\meter\per\second}}.
\end{solution}

\part Part B: Beregn hastigheten $v_2$ ved hjelp av arbeid-energi-setningen:
\begin{solution}
Vi bruker bevaring av kinetisk energi:
\begin{align}
K_{i1} + K_{i2} &= K_1 + K_2 \\
\frac{1}{2} m v_{i1}^2 + \frac{1}{2} m v_{i2}^2 &= \frac{1}{2} m v_1^2 + \frac{1}{2} m v_2^2
\end{align}
Siden $v_{i2} = 0$, forenkler vi til:
\begin{align}
v_2 &= \sqrt{v_{i1}^2 - v_1^2} \\
    &= \sqrt{\left(\SI{2.0}{\meter\per\second}\right)^2 - \left(\SI{1.732}{\meter\per\second}\right)^2} \\
    &= \doubleunderline{\SI{1.0}{\meter\per\second}}
\end{align}
\end{solution}

\part Part C: Vurder om støtet er elastisk:
\begin{solution}
Et støt er elastisk hvis den totale kinetiske energien er bevart. Siden vi har brukt bevaring av kinetisk energi i beregningene våre, kan vi konkludere med at støtet er elastisk.
\end{solution}

\end{parts}



\question FILL QUESTION
\begin{parts}

\part Del A: Beregn friksjonsarbeidet når klossene bremser opp.
\begin{solution}
Friksjonsarbeidet er gitt ved:
\begin{align}
W_{f} &= -f \cdot x \\
f &= \mu \cdot F_{n}
\end{align}
Bruker Newtons andre lov i y-retning for å finne et uttrykk for $F_{n}$:
\begin{align}
\sum F_{y} &= 0 \\
F_{n} - F_{g} &= 0 \\
F_{n} &= (m + M) \cdot g
\end{align}
\end{solution}

\part Del B: Bruk arbeid-energi likningen for å finne et uttrykk for $V_{0}$.
\begin{solution}
Arbeid-energi likningen gir:
\begin{align}
E_{1} &= E_{0} + W_{f} \\
K_{1} &= K_{0} + W_{f} \\
\frac{1}{2}(m+M) V_{1}^{2} &= \frac{1}{2}(m+M) V_{0}^{2} - \mu (m+M) g x \\
\frac{1}{2} V_{0}^{2} &= \mu g x \\
V_{0} &= \sqrt{2 \mu g x}
\end{align}
\end{solution}

\part Del C: Beregn $v_{0}$ for et elastisk støt.
\begin{solution}
For et elastisk støt gjelder bevaring av bevegelsesmengde:
\begin{align}
(m+M) V_{0} &= m v_{0} + M V_{1}^{0}
\end{align}
Løser for $v_{0}$:
\begin{align}
v_{0} &= \frac{m+M}{m} V_{0} \\
&= \frac{0.400 \, \mathrm{kg} + 13 \, \mathrm{kg}}{0.400 \, \mathrm{kg}} \cdot 1.085 \, \mathrm{m/s} \\
v_{0} &= \doubleunderline{\SI{36}{\meter\per\second}}
\end{align}
\end{solution}

\end{parts}

\end{questions}
\end{document}
