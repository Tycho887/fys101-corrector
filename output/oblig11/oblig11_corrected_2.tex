\documentclass[answers,a4paper,12pt]{exam}
\input{preamble.tex}

\title{{\bf{FYS101 Mekanikk}} \\ \Large{\answersareprinted Oblig XX}} 
\author{Institutt for Fysikk, REALTEK}
\date{Uke xx}

\begin{document}
\maketitle

\begin{questions}

\question Beskriv og beregn viktige konsepter relatert til svingninger:
\begin{parts}

\part Definer fasekonstanten i en svingning:
\begin{solution}
Fasekonstanten $\delta$ i en svingning, som beskrevet av ligningen $x = A \cos (\omega t + \delta)$, er en konstant som bestemmer fasen til svingningen ved tiden $t = 0$. Den påvirker hvor svingningen starter i sin syklus.
\end{solution}

\part Forklar resonans:
\begin{solution}
Resonans er et fenomen som oppstår når frekvensen til en ytre, periodisk kraft er lik egenfrekvensen til systemet. Dette fører til at systemet oppnår en større amplitude i sine svingninger.
\end{solution}

\part Definer Q-verdi i dempede svingninger:
\begin{solution}
Q-verdi beskriver kvaliteten på dempede svingninger og er relatert til hvor mye energi som tapes i en svingesyklus. Den er gitt ved:
\[
Q = \frac{2 \pi}{\Delta E / E}
\]
hvor $\Delta E$ er energitapet per syklus og $E$ er den totale energien.
\end{solution}

\part Forklar tidskonstanten $\tau$:
\begin{solution}
Tidskonstanten $\tau$ er tiden det tar for energien i svingningen å reduseres med en faktor $e^{-1}$, som tilsvarer omtrent 37\% av den opprinnelige energien.
\end{solution}

\part Beregn Q-verdi gitt energitap:
\begin{solution}
Anta at energitapet per syklus $\Delta E$ er \SI{10}{\joule} og den totale energien $E$ er \SI{100}{\joule}. Da er Q-verdien:
\begin{align}
Q &= \frac{2 \pi}{\Delta E / E} \\
  &= \frac{2 \pi}{\SI{10}{\joule} / \SI{100}{\joule}} \\
  &= \frac{2 \pi}{0.1} \\
  &= \doubleunderline{62.83}
\end{align}
\end{solution}

\end{parts}


\question FILL QUESTION
\begin{parts}

\part Finn frekvensen til partikkelen:
\begin{solution}
Frekvensen \( f \) kan finnes ved å bruke forholdet mellom vinkelfrekvens \( \omega \) og frekvens:
\begin{align}
\omega &= 2 \pi f \\
f &= \frac{\omega}{2 \pi}
\end{align}
Gitt at \( \omega = 6 \pi \), kan vi beregne:
\begin{align}
f &= \frac{6 \pi}{2 \pi} \\
f &= 3.0 \, \text{Hz}
\end{align}
Frekvensen til partikkelen er \doubleunderline{\SI{3.0}{\hertz}}.
\end{solution}

\part Finn perioden til partikkelen:
\begin{solution}
Perioden \( T \) er den inverse av frekvensen:
\begin{align}
T &= \frac{1}{f} \\
T &= \frac{1}{3.0 \, \text{Hz}} \\
T &= \SI{0.33}{\second}
\end{align}
Perioden til partikkelen er \doubleunderline{\SI{0.33}{\second}}.
\end{solution}

\part Finn amplituden fra funksjonen som beskriver posisjonen til en partikkel:
\begin{solution}
Funksjonen for posisjonen er gitt ved:
\begin{align}
x &= A \cos (\omega t + \sigma)
\end{align}
Her ser vi at amplituden \( A \) er koeffisienten foran cosinus-funksjonen:
\begin{align}
A &= \SI{7.0}{\centi\meter}
\end{align}
Amplituden for svingningen er \doubleunderline{\SI{7.0}{\centi\meter}} fra likevektsposisjonen.
\end{solution}

\part Bestem når partikkelen er ved likevekt for første gang etter \( t = 0 \):
\begin{solution}
Partikkelen er ved likevekt når \(\cos(6 \pi t) = 0\). Dette skjer når:
\begin{align}
\cos\left(n \pi + \frac{\pi}{2}\right) &= 0, \quad \text{der } n = 0, \pm 1, \pm 2, \ldots \\
6 \pi t &= \frac{\pi}{2} \\
t &= \frac{1}{12} \\
t &= \SI{0.083}{\second}
\end{align}
Partikkelen er i likevekt etter \doubleunderline{\SI{0.083}{\second}} etter \( t = 0 \), og beveger seg i negativ \( x \)-retning.
\end{solution}

\end{parts}


\question FILL QUESTION
\begin{parts}

\part Del A: Finn vinkelfrekvensen til bevegelsen når klossen slippes.
\begin{solution}
Vinkelfrekvensen $\omega$ for en harmonisk oscillator er gitt ved:
\begin{align}
\omega &= \sqrt{\frac{k}{m}} \\
       &= \sqrt{\frac{\SI{700}{\newton\per\meter}}{\SI{5.00}{\kilogram}}} \\
       &= \doubleunderline{\SI{11.832}{\radian\per\second}}
\end{align}
Frekvensen $f$ er relatert til vinkelfrekvensen ved:
\begin{align}
f &= \frac{\omega}{2\pi} \\
  &= \frac{\SI{11.832}{\radian\per\second}}{2\pi} \\
  &= \doubleunderline{\SI{1.8831}{\hertz}}
\end{align}
Frekvensen når klossen slippes er \SI{1.88}{\hertz}, og siden underlaget er friksjonsløst, vil frekvensen forbli konstant.
\end{solution}

\part Del B: Finn perioden.
\begin{solution}
Perioden $T$ er gitt ved:
\begin{align}
T &= \frac{1}{f} \\
  &= \frac{1}{\SI{1.8831}{\hertz}} \\
  &= \doubleunderline{\SI{0.531}{\second}}
\end{align}
Klossen bruker \SI{0.531}{\second} på én full svingning.
\end{solution}

\part Del C: Finn amplituden ved å bruke likningen for posisjonen.
\begin{solution}
For å finne amplituden $A$, bruker vi posisjonslikningen:
\begin{align}
x(t) &= A \cos(\omega t + \delta)
\end{align}
Siden $v(0) = 0$, har vi:
\begin{align}
v(0) &= -\omega A \sin(\delta) = 0 \\
\sin(\delta) &= 0 \quad \Rightarrow \quad \delta = 0
\end{align}
Dermed er:
\begin{align}
x_0 &= A \cos(0) \\
A &= x_0 = \SI{8.00}{\centi\meter}
\end{align}
Amplituden er \SI{8.00}{\centi\meter}.
\end{solution}

\part Del D: Finn maksimalhastigheten.
\begin{solution}
Maksimalhastigheten $v_{\text{max}}$ er gitt ved:
\begin{align}
v_{\text{max}} &= \omega A \\
               &= \SI{11.832}{\radian\per\second} \cdot \SI{8.00e-2}{\meter} \\
               &= \doubleunderline{\SI{0.947}{\meter\per\second}}
\end{align}
Maksimalhastigheten til klossen er \SI{0.947}{\meter\per\second}.
\end{solution}

\part Del E: Finn maksimalakselerasjonen.
\begin{solution}
Maksimalakselerasjonen $a_{\text{max}}$ er gitt ved:
\begin{align}
a_{\text{max}} &= \omega^2 A \\
               &= \left(\SI{11.832}{\radian\per\second}\right)^2 \cdot \SI{8.00e-2}{\meter} \\
               &= \doubleunderline{\SI{11.2}{\meter\per\second\squared}}
\end{align}
\end{solution}

\part Del F: Finn et uttrykk for tid ved $x=0$.
\begin{solution}
Når $x=0$, har vi:
\begin{align}
x &= A \cos(\omega t) = 0 \\
\cos(\omega t) &= 0 \\
\omega t &= \frac{\pi}{2} \\
t &= \frac{\pi}{2\omega} \\
  &= \frac{\pi}{2 \cdot \SI{11.832}{\radian\per\second}} \\
  &= \doubleunderline{\SI{0.133}{\second}}
\end{align}
\end{solution}

\end{parts}



\question Finner resonansfrekvens for elastisk pendel:
\begin{parts}

\part Definer resonansfrekvens for en elastisk pendel:
\begin{solution}
Resonansfrekvensen for en elastisk pendel er frekvensen ved hvilken pendelen naturlig svinger når den blir forstyrret fra sin likevektsposisjon. Den avhenger av fjærkonstanten \( k \) og massen \( m \) til pendelen.
\end{solution}

\part Beregn resonansfrekvensen når \( k = \SI{400}{\newton\per\meter} \) og \( m = \SI{10}{\kilo\gram} \):
\begin{solution}
\centering
\begin{align}
f &= \frac{1}{2 \pi} \sqrt{\frac{k}{m}} \\
  &= \frac{1}{2 \pi} \cdot \sqrt{\frac{\SI{400}{\newton\per\meter}}{\SI{10}{\kilo\gram}}} \\
  &= \frac{1}{2 \pi} \cdot \sqrt{\SI{40}{\per\second\squared}} \\
  &= \frac{1}{2 \pi} \cdot \SI{6.32}{\per\second} \\
  &= \doubleunderline{\SI{1.0}{\hertz}}
\end{align}
\end{solution}

\part Beregn resonansfrekvensen for en elastisk pendel der \( k = \SI{800}{\newton\per\meter} \) og \( m = \SI{5.0}{\kilo\gram} \):
\begin{solution}
\centering
\begin{align}
f &= \frac{1}{2 \pi} \sqrt{\frac{k}{m}} \\
  &= \frac{1}{2 \pi} \cdot \sqrt{\frac{\SI{800}{\newton\per\meter}}{\SI{5.0}{\kilo\gram}}} \\
  &= \frac{1}{2 \pi} \cdot \sqrt{\SI{160}{\per\second\squared}} \\
  &= \frac{1}{2 \pi} \cdot \SI{12.65}{\per\second} \\
  &= \doubleunderline{\SI{2.0}{\hertz}}
\end{align}
\end{solution}

\part Matematisk pendel med \( L = \SI{2.0}{\meter} \), \( m = \SI{40}{\kilo\gram} \):
\begin{solution}
For en matematisk pendel, resonansfrekvensen \( f \) er gitt ved:
\begin{align}
f &= \frac{1}{2 \pi} \sqrt{\frac{g}{L}} \\
  &= \frac{1}{2 \pi} \cdot \sqrt{\frac{\SI{9.81}{\meter\per\second\squared}}{\SI{2.0}{\meter}}} \\
  &= \frac{1}{2 \pi} \cdot \sqrt{\SI{4.905}{\per\second\squared}} \\
  &= \frac{1}{2 \pi} \cdot \SI{2.214}{\per\second} \\
  &= \doubleunderline{\SI{0.352}{\hertz}}
\end{align}
\end{solution}

\end{parts}


\question FILL QUESTION
\begin{parts}

\part Del A: Finn egenfrekvensen til blokkens svingning
\begin{solution}
Egenfrekvensen, også kjent som naturlig frekvens, for en harmonisk oscillator kan uttrykkes som:
\begin{align}
\omega_{0} &= \sqrt{\frac{k}{m}} \\
&= \sqrt{\frac{\SI{400}{\newton\per\meter}}{\SI{200}{\kilo\gram}}} \\
&= \doubleunderline{\SI{1.414}{\radian\per\second}}
\end{align}
\end{solution}

\part Del B: Bruk denne til å finne amplituden til svingningene
\begin{solution}
Amplituden til svingningene kan beregnes ved å bruke formelen for maksimal utslag:
\begin{align}
A &= \SI{4.98}{\meter}
\end{align}
Dette er den maksimale avstanden fra likevektsposisjonen.
\end{solution}

\part Del C: Resonans oppstår ved den naturlige frekvensen
\begin{solution}
Resonans oppstår når den påførte frekvensen samsvarer med systemets naturlige frekvens, $\omega_{0} = \SI{1.414}{\radian\per\second}$. Ved en vinkel på \SI{14}{\radian} vil svingningene nå en resonansfrekvens.
\begin{align}
A &= \frac{F_{a}}{\sqrt{m^{2}\left(\omega_{0}^{2}-\omega_{d}^{2}\right)^{2}+(b \omega \sqrt{a})^{2}}}
\end{align}
PLACE GRAPHICS HERE
\end{solution}

\end{parts}
\end{questions}
\end{document}
