\documentclass[answers,a4paper,12pt]{exam}
\input{preamble.tex}

\title{{\bf{FYS101 Mekanikk}} \\ \Large{\answersareprinted Oblig XX}} 
\author{Institutt for Fysikk, REALTEK}
\date{Uke xx}

\begin{document}
\maketitle

\begin{questions}

\question Definer nøkkelbegreper i "Del A":
\begin{parts}
\part Definer fasekonstanten i en svingning:
\begin{solution}
Fasekonstanten, ofte betegnet som $\delta$, er en konstant som beskriver fasen til en svingning ved tiden $t=0$. I ligningen $x = A \cos(\omega t + \delta)$, representerer $(\omega t + \delta)$ bevegelsens fase, som angir hvor i svingningssyklusen systemet befinner seg.
\end{solution}

\part Forklar resonans:
\begin{solution}
Resonans er et fenomen som oppstår når frekvensen til en ytre, periodisk kraft er lik egenfrekvensen til systemet. Dette fører til at systemet oppnår en større amplitude i svingningene, noe som kan føre til betydelige vibrasjoner.
\end{solution}

\part Definer Q-verdi:
\begin{solution}
Q-verdi beskriver kvaliteten av dempede svingninger og indikerer hvor mye energi som tapes i en svingesyklus. Den er relatert til energitapet i systemet og kan uttrykkes som:
\begin{align}
Q = \frac{2 \pi}{\left(\frac{\Delta E}{E}\right) \text{ per svingesyklus}}
\end{align}
\end{solution}

\part Forklar tidskonstanten $\tau$:
\begin{solution}
Tidskonstanten $\tau$ er tiden det tar for energien i en svingning å reduseres til $e^{-1}$ (ca. 37\%) av sin opprinnelige verdi. Dette er en viktig parameter i beskrivelsen av dempede svingninger.
\end{solution}
\end{parts}


\question Posisjonen til en partikkel er gitt ved funksjonen:

\[
x = (7.0 \, \mathrm{cm}) \cos (6 \pi t)
\]

\begin{parts}

\part Finner frekvensen til partikkelen:
\begin{solution}
Frekvensen \( f \) kan finnes ved å bruke forholdet mellom vinkelfrekvens \( \omega \) og frekvens:
\begin{align}
\omega &= 2 \pi f \\
f &= \frac{\omega}{2 \pi} \\
  &= \frac{6 \pi}{2 \pi} \\
  &= \doubleunderline{3.0 \, \mathrm{Hz}}
\end{align}
Frekvensen til partikkelen er altså \SI{3.0}{\hertz}.
\end{solution}

\part Beregn perioden til partikkelen:
\begin{solution}
Perioden \( T \) er den inverse av frekvensen:
\begin{align}
T &= \frac{1}{f} \\
  &= \frac{1}{3.0 \, \mathrm{Hz}} \\
  &= \doubleunderline{0.33 \, \mathrm{s}}
\end{align}
Partikkelen bruker \SI{0.33}{\second} på én full svingning.
\end{solution}

\part Finner amplituden fra funksjonen som beskriver posisjonen til en partikkel:
\begin{solution}
Funksjonen for posisjonen er gitt ved:
\[
x = A \cos (\omega t + \phi)
\]
Her ser vi at amplituden \( A \) er:
\[
A = \doubleunderline{7.0 \, \mathrm{cm}}
\]
Amplituden for svingningen er \SI{7.0}{\centi\meter} fra likevektsposisjonen.
\end{solution}

\part Bestem når partikkelen er ved likevekt for første gang etter \( t = 0 \):
\begin{solution}
Partikkelen er ved likevekt når \(\cos(6 \pi t) = 0\). Dette skjer når:
\[
\cos\left(n \pi + \frac{\pi}{2}\right) = 0, \quad \text{der } n = 0, \pm 1, \pm 2, \ldots
\]
For \( n = 0 \), har vi:
\begin{align}
6 \pi t &= \frac{\pi}{2} \\
t &= \frac{1}{12} \\
t &= \doubleunderline{0.083 \, \mathrm{s}}
\end{align}
Partikkelen er i likevekt etter \SI{0.083}{\second} etter \( t = 0 \), og beveger seg i negativ \( x \)-retning.
\end{solution}

\end{parts}


\question FILL QUESTION
\begin{parts}

\part Del A: Finn vinkelfrekvensen til bevegelsen når klossen slippes.
\begin{solution}
Vinkelfrekvensen, \(\omega\), er gitt ved formelen:
\[
\omega = \sqrt{\frac{k}{m}}
\]
hvor \(k\) er fjærkonstanten og \(m\) er massen. For de gitte verdiene:
\[
\omega = \sqrt{\frac{700 \, \mathrm{N/m}}{5.00 \, \mathrm{kg}}} = \doubleunderline{11.832 \, \mathrm{rad/s}}
\]
Frekvensen, \(f\), kan deretter beregnes ved:
\[
f = \frac{\omega}{2\pi} = \frac{11.832 \, \mathrm{rad/s}}{2\pi} = \doubleunderline{1.8831 \, \mathrm{Hz}}
\]
Frekvensen når klossen slippes er \(\doubleunderline{1.88 \, \mathrm{Hz}}\), og siden underlaget er friksjonsløst, vil frekvensen forbli konstant.
\end{solution}

\part Del B: Finn perioden.
\begin{solution}
Perioden, \(T\), er gitt ved:
\[
T = \frac{1}{f} = \frac{1}{1.8831 \, \mathrm{Hz}} = \doubleunderline{0.531 \, \mathrm{s}}
\]
Klossen bruker \(\doubleunderline{0.531 \, \mathrm{s}}\) på én full svingning.
\end{solution}

\part Del C: Finn amplituden ved å bruke likningen for posisjonen.
\begin{solution}
Posisjonen \(x\) som funksjon av tid \(t\) er gitt ved:
\[
x(t) = A \cos(\omega t + \delta)
\]
Vi vet at \(v(0) = v_0 = 0\), hvilket gir \(\sin(\delta) = 0\), så \(\delta = 0\). Dermed er:
\[
x_0 = A \cos(0) = A
\]
Amplituden \(A\) er derfor \(\doubleunderline{8.00 \, \mathrm{cm}}\).
\end{solution}

\part Del D: Finn maksimalhastigheten.
\begin{solution}
Maksimalhastigheten \(v\) er gitt ved:
\[
v = \omega A
\]
For de gitte verdiene:
\[
v = 11.832 \, \mathrm{rad/s} \cdot 8.00 \times 10^{-2} \, \mathrm{m} = \doubleunderline{0.947 \, \mathrm{m/s}}
\]
Maksimalhastigheten til klossen er \(\doubleunderline{0.947 \, \mathrm{m/s}}\).
\end{solution}

\part Del E: Finn maksimalakselerasjonen.
\begin{solution}
Maksimalakselerasjonen \(a\) er gitt ved:
\[
a = \omega^2 A
\]
For de gitte verdiene:
\[
a = (11.832 \, \mathrm{rad/s})^2 \cdot 8.00 \times 10^{-2} \, \mathrm{m} = \doubleunderline{11.2 \, \mathrm{m/s^2}}
\]
\end{solution}

\part Del F: Finn et uttrykk for tiden ved \(x = 0\).
\begin{solution}
Posisjonen \(x\) er gitt ved:
\[
x = 8.00 \, \mathrm{cm} \cdot \cos(11.832 \, \mathrm{rad/s} \cdot t)
\]
Når \(x = 0\), har vi:
\[
\cos(11.832 \, \mathrm{rad/s} \cdot t) = 0
\]
Dette gir:
\[
11.832 \, \mathrm{rad/s} \cdot t = \frac{\pi}{2}
\]
Løsning for \(t\) gir:
\[
t = \frac{\pi}{2 \cdot 11.832} = \doubleunderline{0.133 \, \mathrm{s}}
\]
\end{solution}

\end{parts}


\question Finner resonansfrekvens for elastisk pendel:
\begin{parts}

\part Definer resonansfrekvens for en elastisk pendel:
\begin{solution}
Resonansfrekvensen for en elastisk pendel er frekvensen ved hvilken systemet naturlig oscillerer når det ikke er påvirket av noen dempende krefter. Den kan beregnes ved hjelp av fjærkonstanten \(k\) og massen \(m\) som er festet til fjæren.
\end{solution}

\part Beregn resonansfrekvensen når \(k = \SI{400}{\newton\per\meter}\) og \(m = \SI{10}{\kilogram}\):
\begin{solution}
\centering
\begin{align}
f &= \frac{1}{2\pi} \sqrt{\frac{k}{m}} \\
  &= \frac{1}{2\pi} \sqrt{\frac{\SI{400}{\newton\per\meter}}{\SI{10}{\kilogram}}} \\
  &= \frac{1}{2\pi} \sqrt{\SI{40}{\per\second\squared}} \\
  &= \frac{1}{2\pi} \cdot \SI{6.32}{\per\second} \\
  &= \doubleunderline{\SI{1.0}{\hertz}}
\end{align}
\end{solution}

\part Beregn resonansfrekvensen for en elastisk pendel der \(k = \SI{800}{\newton\per\meter}\) og \(m = \SI{5.0}{\kilogram}\):
\begin{solution}
\centering
\begin{align}
f &= \frac{1}{2\pi} \sqrt{\frac{k}{m}} \\
  &= \frac{1}{2\pi} \sqrt{\frac{\SI{800}{\newton\per\meter}}{\SI{5.0}{\kilogram}}} \\
  &= \frac{1}{2\pi} \sqrt{\SI{160}{\per\second\squared}} \\
  &= \frac{1}{2\pi} \cdot \SI{12.65}{\per\second} \\
  &= \doubleunderline{\SI{2.0}{\hertz}}
\end{align}
\end{solution}

\part Beregn resonansfrekvensen for en matematisk pendel med \(L = \SI{2.0}{\meter}\) og \(m = \SI{40}{\kilogram}\):
\begin{solution}
For en matematisk pendel er resonansfrekvensen avhengig av lengden \(L\) og tyngdeakselerasjonen \(g\). Formelen er:
\begin{align}
f &= \frac{1}{2\pi} \sqrt{\frac{g}{L}} \\
  &= \frac{1}{2\pi} \sqrt{\frac{\SI{9.81}{\meter\per\second\squared}}{\SI{2.0}{\meter}}} \\
  &= \frac{1}{2\pi} \sqrt{\SI{4.905}{\per\second\squared}} \\
  &= \frac{1}{2\pi} \cdot \SI{2.21}{\per\second} \\
  &= \doubleunderline{\SI{0.35}{\hertz}}
\end{align}
\end{solution}

\end{parts}


\question FILL QUESTION
\begin{parts}

\part Del A: Finn ukjent frekvens for blokkens svingning
\begin{solution}
Den ukjente frekvensen for blokkens svingning kan beregnes ved hjelp av formelen for naturlig frekvens:

\begin{align}
\omega_{0} &= \sqrt{\frac{k}{m}} \\
           &= \sqrt{\frac{\SI{400}{\newton\per\meter}}{\SI{200}{\kilo\gram}}} \\
           &= \doubleunderline{\SI{1.414}{\radian\per\second}}
\end{align}

Denne frekvensen brukes til å finne amplituden av svingningene:

\begin{align}
A &= \SI{4.98}{\meter}
\end{align}
\end{solution}

\part Del B: Resonans oppstår ved den naturlige frekvensen
\begin{solution}
Resonans oppstår når den påførte frekvensen samsvarer med den naturlige frekvensen til systemet. Ved en vinkel frekvens på \(\omega_{0} = \SI{14.1}{\radian\per\second}\), vil svingningene nå en resonansfrekvens.

Amplituden ved resonans kan beregnes ved:

\begin{align}
A &= \frac{F_{a}}{\sqrt{m^{2}\left(\omega_{0}^{2}-\omega_{d}^{2}\right)^{2}+(b \omega \sqrt{a})^{2}}}
\end{align}

PLACE GRAPHICS HERE
\end{solution}

\end{parts}

\end{questions}
\end{document}
