\documentclass[answers,a4paper,12pt]{exam}
\input{preamble.tex}

\title{{\bf{FYS101 Mekanikk}} \\ \Large{\answersareprinted Oblig XX}} 
\author{Institutt for Fysikk, REALTEK}
\date{Uke xx}

\begin{document}
\maketitle

\begin{questions}

\question Lesnings forslag oblig 9

\begin{parts}

\part Del A: Beregn tyngdefeltet ved romstasjonen
\begin{solution}
Vi har følgende data:
\begin{align*}
R_{E} &= \SI{6370}{\kilo\meter} \\
h &= \SI{400}{\kilo\meter} \\
M_{E} &= \SI{5.97e24}{\kilo\gram}
\end{align*}

Tyngdefeltet $g$ ved romstasjonen kan beregnes ved å bruke Newtons gravitasjonslov:
\begin{align}
g &= \frac{G M_{E}}{\left(R_{E}+h\right)^{2}} \\
  &= \frac{\SI{6.673e-11}{\newton\meter\squared\per\kilo\gram\squared} \cdot \SI{5.97e24}{\kilo\gram}}{\left(\SI{6370e3}{\meter} + \SI{400e3}{\meter}\right)^{2}} \\
  &= \doubleunderline{\SI{8.69}{\newton\per\kilo\gram}}
\end{align}

Tyngdefeltet har omtrent samme verdi på romstasjonen som ved jordoverflaten.
\end{solution}

\part Del B: Forklar hvorfor astronauter opplever muskelsvakhet
\begin{solution}
Det som fører til muskelsvakhet er ikke mangel på tyngdekraften, men at astronautene ikke opplever noen normal kraft. Mangelen på sammentrykkende krefter fører til at musklene og beina blir svakere.
\end{solution}

\end{parts}




\question Beregn jordas masse ved hjelp av Newtons andre lov og gravitasjonsloven:
\begin{parts}

\part Definer de nødvendige formlene:
\begin{solution}
For å beregne jordas masse, $M_E$, bruker vi gravitasjonskraften og sentripetalakselerasjonen. Vi starter med Newtons andre lov for månen:
\begin{align}
\sum F &= M_M a \quad \left(a = \frac{v^2}{r}\right)
\end{align}
Gravitasjonskraften mellom jorda og månen er gitt ved:
\begin{align}
F_g &= \frac{G M_M M_E}{r_m^2}
\end{align}
Sentripetalakselerasjonen for månen er:
\begin{align}
a &= \frac{v^2}{r_m}
\end{align}
\end{solution}

\part Beregn jordas masse:
\begin{solution}
Vi setter likningene for gravitasjonskraft og sentripetalakselerasjon lik hverandre:
\begin{align}
\frac{G M_M M_E}{r_m^2} &= M_M \frac{v^2}{r_m}
\end{align}
Forenkler ved å kansellere $M_M$:
\begin{align}
G M_E &= v^2 r_m
\end{align}
Hastigheten $v$ er gitt ved:
\begin{align}
v &= \frac{2 \pi r_m}{T}
\end{align}
Setter inn for $v$:
\begin{align}
G M_E &= \left(\frac{2 \pi r_m}{T}\right)^2 r_m
\end{align}
Løser for $M_E$:
\begin{align}
M_E &= \frac{4 \pi^2 r_m^3}{G T^2}
\end{align}
Setter inn verdiene:
\begin{align}
M_E &= \frac{4 \pi^2 \left(3.841 \times 10^8 \, \mathrm{m}\right)^3}{6.673 \times 10^{-11} \, \mathrm{N \cdot m^2/kg^2} \cdot (2358720 \, \mathrm{s})^2} \\
M_E &= \doubleunderline{6.02 \times 10^{24} \, \mathrm{kg}}
\end{align}
\end{solution}

\part Sammenlign med tabellverdi:
\begin{solution}
Den beregnede massen for jorda er $6.02 \times 10^{24} \, \mathrm{kg}$, som er i god overensstemmelse med den tabulerte verdien $5.97 \times 10^{24} \, \mathrm{kg}$.
\end{solution}

\end{parts}



\question Energien til en satellitt med masse \( m \) som går i bane rundt jorda er gitt ved:
\begin{parts}
\part Forklar energien til en satellitt:
\begin{solution}
Energien \( E \) til en satellitt består av kinetisk energi \( K \) og potensiell energi \( U \). Formelen for total energi er:
\begin{align}
E &= K + U \\
E &= \frac{1}{2} m v^2 - \frac{G M_E m}{R}
\end{align}
der \( R \) er avstanden mellom satellitten og jordas sentrum, \( G \) er gravitasjonskonstanten, \( M_E \) er jordas masse, og \( v \) er satellittens hastighet.
\end{solution}

\part Bruker Newtons andre lov for å finne farten \( v \) satellitten beveger seg med i bane:
\begin{solution}
Ifølge Newtons andre lov, summen av kreftene er lik massen multiplisert med akselerasjonen:
\begin{align}
\sum F &= m a \quad \left(a = \frac{v^2}{R}\right) \\
F_g &= \frac{m v^2}{R} \\
\frac{G M_E m}{R^2} &= \frac{m v^2}{R} \\
v^2 &= \frac{G M_E}{R}
\end{align}
\end{solution}

\part Sett likning (II) inn i likning (I) for å finne et uttrykk for totalenergien til satellitten:
\begin{solution}
Ved å sette inn uttrykket for \( v^2 \) fra forrige del:
\begin{align}
E &= \frac{1}{2} \frac{G M_E m}{R} - \frac{G M_E m}{R} \\
E &= -\frac{G M_E m}{2 R}
\end{align}
Dette viser at totalenergien er negativ, noe som indikerer en bundet bane.
\end{solution}

\part Finner energiforskjellen til satellitten for de to ulike banene:
\begin{solution}
Energiforskjellen mellom to baner er gitt ved:
\begin{align}
\Delta E &= E_2 - E_1 \\
&= -\frac{G M_E m}{2(R_E + h_2)} + \frac{G M_E m}{2(R_E + h_1)} \\
&= \frac{1}{2} G M_E m \left(\frac{1}{R_E + h_1} - \frac{1}{R_E + h_2}\right)
\end{align}
Ved å sette inn verdiene:
\begin{align}
\Delta E &= 6.67 \times 10^{-11} \, \mathrm{Nm^2/kg^2} \times 5.97 \times 10^{24} \, \mathrm{kg} \times 500 \, \mathrm{kg} \\
&\quad \times \left(\frac{1}{6370 \times 10^3 \, \mathrm{m} + 1000 \times 10^2 \, \mathrm{m}} - \frac{1}{6370 \times 10^3 + 35790 \times 10^3 \, \mathrm{m}}\right) \\
\Delta E &= \doubleunderline{1.11 \times 10^{10} \, \mathrm{J}}
\end{align}
Det krever \SI{11.1}{\giga\joule} mer energi for å få satellitten i den geosynkrone satellittbanen enn i den "vanlige" satellittbanen.
\end{solution}
\end{parts}

\end{questions}
\end{document}
