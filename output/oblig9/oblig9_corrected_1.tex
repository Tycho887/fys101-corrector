\documentclass[answers,a4paper,12pt]{exam}
\input{preamble.tex}

\title{{\bf{FYS101 Mekanikk}} \\ \Large{\answersareprinted Oblig XX}} 
\author{Institutt for Fysikk, REALTEK}
\date{Uke xx}

\begin{document}
\maketitle

\begin{questions}
\question Lesningsforslag oblig 9

\begin{parts}

\part Part A: Beregn tyngdefeltet ved romstasjonen

\begin{solution}
Vi starter med å bruke Newtons gravitasjonslov for å finne tyngdefeltet $g$ ved romstasjonen. Tyngdefeltet $g$ er gitt ved:

\begin{align}
g &= \frac{G M_{E}}{\left(R_{E}+h\right)^{2}}
\end{align}

Her er $G$ gravitasjonskonstanten, $M_{E}$ massen til jorden, $R_{E}$ jordens radius, og $h$ høyden over jordens overflate. Vi setter inn verdiene:

\begin{align}
g &= \frac{6.673 \times 10^{-11} \, \mathrm{N} \cdot \mathrm{m}^2/\mathrm{kg}^2 \cdot 5.97 \times 10^{24} \, \mathrm{kg}}{\left(6370 \times 10^{3} \, \mathrm{m} + 400 \times 10^{3} \, \mathrm{m}\right)^{2}} \\
g &= \doubleunderline{8.69 \, \mathrm{m/s^2}}
\end{align}

Dette viser at tyngdefeltet ved romstasjonen er omtrent det samme som ved jordoverflaten.
\end{solution}

\part Part B: Diskuter årsaken til muskelsvakhet hos astronauter

\begin{solution}
Muskelsvakhet hos astronauter skyldes ikke mangel på tyngdekraft, men heller fraværet av en normal kraft. På jorden opplever vi en konstant normal kraft som virker mot tyngdekraften, noe som gir musklene våre en konstant belastning. I rommet, derimot, er denne normale kraften fraværende, noe som fører til at musklene og beina ikke blir utsatt for den samme belastningen. Dette resulterer i at musklene og beina blir svakere over tid.
\end{solution}

\end{parts}

\question Beregn jordas masse ved å bruke Newtons andre lov på månen:
\begin{parts}

\part Definer de nødvendige formlene:
\begin{solution}
For å finne jordas masse, bruker vi gravitasjonskraften og Newtons andre lov. Vi starter med å uttrykke gravitasjonskraften mellom jorda og månen:
\[
F_g = \frac{G M_M M_E}{r_m^2}
\]
hvor \( G \) er gravitasjonskonstanten, \( M_M \) er månens masse, \( M_E \) er jordas masse, og \( r_m \) er avstanden mellom jorda og månen.

Ifølge Newtons andre lov er akselerasjonen \( a \) gitt ved:
\[
a = \frac{v^2}{r_m}
\]
hvor \( v \) er månens omløpshastighet.

Ved å sette disse sammen får vi:
\[
M_M \frac{v^2}{r_m} = \frac{G M_M M_E}{r_m^2}
\]
\end{solution}

\part Utled uttrykket for jordas masse:
\begin{solution}
Vi starter med å forenkle uttrykket:
\[
\frac{G M_E}{r_m} = v^2
\]
hvor \( v = \frac{2 \pi r_m}{T} \). Sett inn for \( v \):
\[
\frac{G M_E}{r_m} = \left(\frac{2 \pi r_m}{T}\right)^2
\]

Løs for \( M_E \):
\[
M_E = \frac{4 \pi^2 r_m^3}{G T^2}
\]

Sett inn de gitte verdiene:
\begin{align}
M_E &= \frac{4 \pi^2 (3.841 \times 10^8 \, \mathrm{m})^3}{6.673 \times 10^{-11} \, \mathrm{N \cdot m^2/kg^2} \cdot (2358720 \, \mathrm{s})^2} \\
    &= \doubleunderline{6.02 \times 10^{24} \, \mathrm{kg}}
\end{align}
\end{solution}

\part Sammenlign med tabellverdi:
\begin{solution}
Den beregnede massen av jorda er \SI{6.02e24}{\kilogram}, som stemmer godt overens med den tabulerte verdien \SI{5.97e24}{\kilogram}.
\end{solution}

\end{parts}


\question Beregn energien til en satellitt i bane rundt jorden:
\begin{parts}

\part Del A: Definer energien til en satellitt
\begin{solution}
Energien til en satellitt med masse \( m \) som går i bane rundt jorden er gitt ved:
\begin{align}
E &= K + U \\
E &= \frac{1}{2} m v^{2} - \frac{G M_E m}{R}
\end{align}
der \( R \) er avstanden mellom satellitten og jordens sentrum, \( K \) er den kinetiske energien, og \( U \) er den potensielle energien.
\end{solution}

\part Del B: Bruk Newtons andre lov for å finne farten \( v \) til satellitten
\begin{solution}
Ved å bruke Newtons andre lov, finner vi farten \( v \) som satellitten beveger seg med i banen:
\begin{align}
\sum F &= m a \quad \left(a = \frac{v^{2}}{R}\right) \\
F_g &= \frac{m v^{2}}{R} \\
\frac{G M_E m}{R^{2}} &= \frac{m v^{2}}{R} \\
v^{2} &= \frac{G M_E}{R}
\end{align}
\end{solution}

\part Del C: Sett uttrykket for \( v^2 \) inn i uttrykket for totalenergien
\begin{solution}
Ved å sette uttrykket for \( v^2 \) inn i uttrykket for totalenergien, får vi:
\begin{align}
E &= \frac{1}{2} \frac{G M_E m}{R} - \frac{G M_E m}{R} \\
E &= -\frac{G M_E m}{2 R}
\end{align}
Dette viser at totalenergien til satellitten er negativ, noe som indikerer en bundet bane.
\end{solution}

\part Del D: Beregn energiforskjellen for to ulike baner
\begin{solution}
Energiforskjellen mellom to baner er gitt ved:
\begin{align}
\Delta E &= E_2 - E_1 \\
&= -\frac{G M_E m}{2(R_E + h_2)} + \frac{G M_E m}{2(R_E + h_1)} \\
&= \frac{1}{2} G M_E m \left(\frac{1}{R_E + h_1} - \frac{1}{R_E + h_2}\right)
\end{align}
Med de gitte verdiene:
\begin{align}
\Delta E &= 6.67 \times 10^{-11} \, \mathrm{Nm^2/kg^2} \times 5.97 \times 10^{24} \, \mathrm{kg} \times 500 \, \mathrm{kg} \\
&\quad \times \left(\frac{1}{6370 \times 10^{3} \, \mathrm{m} + 1000 \times 10^{2} \, \mathrm{m}} - \frac{1}{6370 \times 10^{3} + 35790 \times 10^{3} \, \mathrm{m}}\right) \\
\Delta E &= \doubleunderline{1.11 \times 10^{10} \, \mathrm{J}}
\end{align}
Det krever \SI{11.1}{\giga\joule} mer energi for å få satellitten i den geosynkrone satellittbanen enn i den "vanlige" satellittbanen.
\end{solution}

\end{parts}

\end{questions}
\end{document}
