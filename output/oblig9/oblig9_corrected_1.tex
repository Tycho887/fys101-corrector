\documentclass[answers,a4paper,12pt]{exam}
\input{preamble.tex}

\title{{\bf{FYS101 Mekanikk}} \\ \Large{\answersareprinted Oblig XX}} 
\author{Institutt for Fysikk, REALTEK}
\date{Uke xx}

\begin{document}
\maketitle

\begin{questions}
\question FILL QUESTION
\begin{parts}

\part Del A: Beregn tyngdefeltet ved romstasjonen
\begin{solution}
Vi starter med å bruke Newtons gravitasjonslov for å finne tyngdefeltet $g$ ved romstasjonen. Tyngdefeltet $g$ er gitt ved:

\begin{align}
g & = \frac{G M_{E}}{\left(R_{E}+h\right)^{2}}
\end{align}

hvor:
\begin{align*}
G & = \SI{6.673e-11}{\newton\meter\squared\per\kilogram\squared} \\
M_{E} & = \SI{5.97e24}{\kilogram} \\
R_{E} & = \SI{6370e3}{\meter} \\
h & = \SI{400e3}{\meter}
\end{align*}

Setter vi inn verdiene, får vi:

\begin{align}
g & = \frac{6.673 \times 10^{-11} \, \mathrm{N \cdot m^2/kg^2} \cdot 5.97 \times 10^{24} \, \mathrm{kg}}{\left(6370 \times 10^{3} \, \mathrm{m} + 400 \times 10^{3} \, \mathrm{m}\right)^{2}} \\
  & = \doubleunderline{\SI{8.69}{\newton\per\kilogram}}
\end{align}

Dette viser at tyngdefeltet ved romstasjonen er omtrent det samme som ved jordoverflaten.
\end{solution}

\part Del B: Forklar hvorfor astronauter opplever muskelsvakhet
\begin{solution}
Det som fører til muskelsvakhet hos astronauter er ikke mangel på tyngdekraft, men fraværet av en normal kraft. I vektløs tilstand opplever ikke astronautene den vanlige kompresjonen på musklene og beina som de gjør på jorden. Dette fører til at musklene og beina blir svakere over tid på grunn av mangel på belastning og stimulering.
\end{solution}

\end{parts}

\question FILL QUESTION
\begin{parts}

\part Beregn jordas masse ved å bruke Newtons andre lov på månen:
\begin{solution}
Vi starter med å bruke Newtons andre lov (N2L) på månen. For å finne jordas masse, $M_E$, bruker vi gravitasjonskraften og sentripetalakselerasjonen:

\begin{align}
\sum F &= M_M a \quad \left(a = \frac{v^2}{r}\right) \\
F_g &= M_M \frac{v^2}{r_m} \\
\frac{G M_M M_E}{r_m^2} &= M_M \frac{v^2}{r_m} \\
G M_E &= v^2 \quad \left(v = \frac{2 \pi r_m}{T}\right)
\end{align}

Ved å sette inn uttrykket for hastigheten $v$, får vi:

\begin{align}
\frac{G M_E}{r_m} &= \left(\frac{2 \pi r_m}{T}\right)^2
\end{align}

Vi løser for $M_E$:

\begin{align}
M_E &= \frac{4 \pi^2}{G} \frac{r_m^3}{T^2}
\end{align}

Setter inn verdiene:

\begin{align}
M_E &= \frac{4 \pi^2 \left(3.841 \times 10^8 \, \mathrm{m}\right)^3}{6.673 \times 10^{-11} \, \mathrm{N \cdot m^2/kg^2} \cdot \left(2358720 \, \mathrm{s}\right)^2} \\
M_E &= \doubleunderline{6.02 \times 10^{24} \, \mathrm{kg}}
\end{align}
\end{solution}

\part Sammenlign med tabellverdien:
\begin{solution}
Den beregnede massen til jorda, $6.02 \times 10^{24} \, \mathrm{kg}$, stemmer godt overens med den tabulerte verdien $5.97 \times 10^{24} \, \mathrm{kg}$. Dette viser at beregningene våre er nøyaktige og i tråd med kjente data.
\end{solution}

\end{parts}

\question FILL QUESTION
\begin{parts}

\part Energien til en satellitt med masse $m$ som går i bane rundt jorda er gitt ved:
\begin{solution}
Energien til en satellitt i bane består av kinetisk energi $K$ og potensiell energi $U$. Den totale energien $E$ er gitt ved:
\begin{align}
E &= K + U \\
E &= \frac{1}{2} m v^2 - \frac{G M_E m}{R}
\end{align}
hvor $R$ er avstanden mellom satellitten og jordas sentrum, $G$ er gravitasjonskonstanten, $M_E$ er jordas masse, og $v$ er satellittens hastighet.
\end{solution}

\part Bruker Newtons andre lov for å finne farten $v$ satellitten beveger seg med i bane:
\begin{solution}
Ved å bruke Newtons andre lov, $\sum F = ma$, og uttrykket for sentripetalakselerasjon $a = \frac{v^2}{R}$, får vi:
\begin{align}
F_g &= \frac{m v^2}{R} \\
\frac{G M_E m}{R^2} &= \frac{m v^2}{R} \\
v^2 &= \frac{G M_E}{R}
\end{align}
Dette uttrykket gir oss farten $v$ som satellitten beveger seg med i en sirkulær bane.
\end{solution}

\part Setter likning (II) inn i likning (I) for å finne et uttrykk for totalenergien til satellitten:
\begin{solution}
Ved å sette uttrykket for $v^2$ fra likning (II) inn i likning (I), får vi:
\begin{align}
E &= \frac{1}{2} \frac{G M_E m}{R} - \frac{G M_E m}{R} \\
E &= -\frac{G M_E m}{2R}
\end{align}
Dette uttrykket viser at den totale energien til satellitten er negativ, noe som indikerer en bundet bane.
\end{solution}

\part Finner energiforskjellen til satellitten for de to ulike banene:
\begin{solution}
Energiforskjellen mellom to baner er gitt ved:
\begin{align}
\Delta E &= E_2 - E_1 \\
&= -\frac{G M_E m}{2(R_E + h_2)} + \frac{G M_E m}{2(R_E + h_1)} \\
&= \frac{1}{2} G M_E m \left( \frac{1}{R_E + h_1} - \frac{1}{R_E + h_2} \right)
\end{align}
Ved å sette inn verdiene:
\begin{align}
\Delta E &= 6.67 \times 10^{-11} \, \mathrm{Nm}^2/\mathrm{kg}^2 \times 5.97 \times 10^{24} \, \mathrm{kg} \times 500 \, \mathrm{kg} \left( \frac{1}{6370 \times 10^3 \, \mathrm{m} + 1000 \times 10^2 \, \mathrm{m}} - \frac{1}{6370 \times 10^3 + 35790 \times 10^3 \, \mathrm{m}} \right) \\
\Delta E &= \doubleunderline{1.11 \times 10^{10} \, \mathrm{J}}
\end{align}
Det krever \SI{11.1}{\giga\joule} mer energi for å få satellitten i den geosynkrone satellittbanen enn i den "vanlige" satellittbanen.
\end{solution}

\end{parts}
\end{questions}
\end{document}
