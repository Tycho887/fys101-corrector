\documentclass[answers,a4paper,12pt]{exam}
\input{preamble.tex}

\title{{\bf{FYS101 Mekanikk}} \\ \Large{\answersareprinted Oblig XX}} 
\author{Institutt for Fysikk, REALTEK}
\date{Uke xx}

\begin{document}
\maketitle

\begin{questions}
\question FILL QUESTION
\begin{parts}

\part Part A: Definer vinkelfart (u) og banehastighet (v_t):
\begin{solution}
Vinkelfart, ofte betegnet som \( u \), er hastigheten som et objekt roterer rundt en akse. Den måles vanligvis i radianer per sekund. Banehastighet, \( v_t \), er hastigheten til et punkt på en roterende kropp langs sin bane. Den er relatert til vinkelfarten ved formelen:
\begin{align}
v_{t} &= r \cdot u
\end{align}
hvor \( r \) er radiusen til banen.
\end{solution}

\part Part B: Definer vinkelakselerasjon (\(\alpha\)) og tangentiell baneakselerasjon (\(a_t\)):
\begin{solution}
Vinkelakselerasjon, \(\alpha\), er endringen i vinkelfart over tid. Den beskriver hvor raskt et objekt akselererer i rotasjon og måles i radianer per sekund kvadrat. Tangentiell baneakselerasjon, \(a_t\), er akselerasjonen langs banen til et punkt på en roterende kropp. Den er relatert til vinkelakselerasjonen ved formelen:
\begin{align}
a_{t} &= r \cdot \alpha
\end{align}
hvor \( r \) er radiusen til banen.
\end{solution}

\end{parts}

\question FILL QUESTION
\begin{parts}
\part Forklar Newtons 2. lov for rotasjon:
\begin{solution}
Newtons andre lov for rotasjon sier at summen av de eksterne dreiemomentene (\(\sum_{i} \tau_{i}\)) som virker på et legeme er lik produktet av treghetsmomentet (\(I\)) og vinkelakselerasjonen (\(\alpha\)). Dette kan uttrykkes som:
\begin{align}
\sum_{i} \tau_{i} &= I \alpha
\end{align}
Summen av de eksterne kraftmomentene er altså lik treghetsmomentet multiplisert med vinkelakselerasjonen.
\end{solution}
\end{parts}

\question FILL QUESTION
\begin{parts}

\part Part A: Beregn vinkelfarten
\begin{solution}
Vinkelfarten, også kjent som den vinkelrette hastigheten, kan beregnes ved å bruke formelen for vinkelfart:

\begin{align}
\omega &= \frac{v}{r} \\
       &= \frac{\SI{25}{\meter\per\second}}{\SI{90}{\meter}} \\
\omega &= \doubleunderline{\SI{0.28}{\radian\per\second}}
\end{align}

Vinkelfarten er altså \SI{0.28}{\radian\per\second}.
\end{solution}

\part Part B: Beregn antall omdreininger i løpet av \SI{30}{\second}
\begin{solution}
Antall omdreininger kan beregnes ved å multiplisere vinkelfarten med tiden og deretter konvertere til omdreininger:

\begin{align}
\theta &= \omega \cdot t \\
       &= \SI{0.28}{\radian\per\second} \cdot \SI{30}{\second} \cdot \frac{1 \text{ rev}}{2\pi \text{ rad}} \\
\theta &= \doubleunderline{\SI{1.3}{rev}}
\end{align}

Antall omdreininger i løpet av \SI{30}{\second} er \SI{1.3}{rev}.
\end{solution}

\end{parts}

\question FILL QUESTION
\begin{parts}

\part Beregn den opprinnelige vinkelfarten $w_0$:
\begin{solution}
Vi starter med å bruke formelen for konstant vinkelakselerasjon:

\begin{align}
\omega_1 &= \omega_0 + \alpha \cdot t \\
\alpha &= \frac{\omega_1 - \omega_0}{t} \\
\alpha &= \frac{-\omega_0}{t} \quad \text{(siden $\omega_1 = 0$)}
\end{align}

Sett inn i bevegelseslikningen:

\begin{align}
\theta_1 &= \theta_0 + \omega_0 t + \frac{1}{2} \alpha t^2 \\
\theta_1 &= \theta_0 + \omega_0 t - \frac{1}{2} \frac{\omega_0}{t} t^2 \\
\theta_1 &= \omega_0 t - \frac{1}{2} \omega_0 t \\
\theta_1 &= \frac{1}{2} \omega_0 t
\end{align}

Løs for $\omega_0$:

\begin{align}
\omega_0 &= \frac{2 \theta_1}{t} \\
&= \frac{2 \times 5.0 \, \text{rad}}{2.8 \, \text{s}} \\
\omega_0 &= \doubleunderline{3.6 \, \text{rad/s}}
\end{align}

Vinkelfarten var altså $\SI{3.6}{\radian\per\second}$ før hjulet begynte å bremse.
\end{solution}

\end{parts}

\question FILL QUESTION
\begin{parts}

\part Beregn treghetsmomentet for en solid sfære som roterer om en akse tangent til overflaten:
\begin{solution}
For å finne treghetsmomentet når sfæren roterer om en tangentakse, bruker vi parallellakse-teoremet. Treghetsmomentet om massesenteret er gitt ved:

\[
I_{\mathrm{cm}} = \frac{2}{5} M R^{2}
\]

Ved bruk av parallellakse-teoremet, der \( h = R \), får vi:

\[
I = I_{\mathrm{cm}} + M h^{2}
\]

Setter inn uttrykket for \( I_{\mathrm{cm}} \):

\begin{align}
I &= \frac{2}{5} M R^{2} + M R^{2} \\
  &= \frac{2}{5} M R^{2} + \frac{5}{5} M R^{2} \\
  &= \frac{7}{5} M R^{2}
\end{align}

Derfor er treghetsmomentet når sfæren roterer om en tangentakse \doubleunderline{\(\frac{7}{5} M R^{2}\)}.
\end{solution}

\end{parts}

\question Beregn treghetsmomentet for et hjul:
\begin{parts}
\part Del A: Definer treghetsmoment for ulike komponenter
\begin{solution}
Treghetsmomentet, også kjent som rotasjonsinerti, er et mål på hvor vanskelig det er å endre rotasjonstilstanden til et objekt. For en masse \( M \) som roterer rundt en akse, er treghetsmomentet gitt ved:
\[
I_{\text{hjul}} = M R^2
\]
For en stang med masse \( m \) og lengde \( L \), som roterer rundt en ende, er treghetsmomentet:
\[
I_{\text{eike}} = \frac{1}{3} m L^2
\]
\end{solution}

\part Del B: Beregn hele hjulets treghetsmoment
\begin{solution}
Vi har at \( R = L = \frac{D}{2} \). Hele hjulets treghetsmoment er summen av alle delkomponentenes treghetsmoment:
\begin{align}
I & = I_{\text{hjul}} + 6 \cdot I_{\text{eike}} \\
  & = M R^2 + 6 \cdot \frac{1}{3} m L^2 \\
  & = \frac{1}{4} M D^2 + \frac{1}{2} m D^2 \\
  & = \frac{1}{4}(M + 2m) D^2 \\
  & = \frac{1}{4}(8.00 \, \mathrm{kg} + 2 \cdot 1.20 \, \mathrm{kg}) \cdot (1.00 \, \mathrm{m})^2 \\
  & = \frac{1}{4}(8.00 + 2.40) \, \mathrm{kg} \cdot 1.00 \, \mathrm{m}^2 \\
  & = \frac{1}{4} \cdot 10.40 \, \mathrm{kg} \cdot 1.00 \, \mathrm{m}^2 \\
  & = \doubleunderline{2.60 \, \mathrm{kg \cdot m^2}}
\end{align}
\end{solution}
\end{parts}
\end{questions}
\end{document}
