\documentclass[answers,a4paper,12pt]{exam}
\input{preamble.tex}

\title{{\bf{FYS101 Mekanikk}} \\ \Large{\answersareprinted Oblig XX}} 
\author{Institutt for Fysikk, REALTEK}
\date{Uke xx}

\begin{document}
\maketitle

\begin{questions}
\question Definer nøkkelbegreper i "Del A":
\begin{parts}
\part Definer vinkelhastighet (u) og banehastighet (v_t):
\begin{solution}
Vinkelhastighet, ofte betegnet som \( u \), er hastigheten med hvilken en gjenstand roterer rundt en akse. Banehastighet, \( v_t \), er den lineære hastigheten til et punkt på en roterende gjenstand, som kan uttrykkes som:
\begin{align}
v_{t} &= r \cdot u \cdot t
\end{align}
hvor \( r \) er radiusen til rotasjonen, \( u \) er vinkelhastigheten, og \( t \) er tiden.
\end{solution}

\part Definer vinkelakselerasjon (\(\alpha\)) og tangentiell banakselerasjon (\(a_t\)):
\begin{solution}
Vinkelakselerasjon, \(\alpha\), er endringen i vinkelhastighet over tid. Tangentiell banakselerasjon, \(a_t\), er akselerasjonen langs en bane som er tangent til rotasjonsbanen, og kan uttrykkes som:
\begin{align}
a_{t} &= r \cdot \alpha
\end{align}
hvor \( r \) er radiusen til rotasjonen og \(\alpha\) er vinkelakselerasjonen.
\end{solution}
\end{parts}


\question Definer nøkkelkonsepter i "Del B":
\begin{parts}
\part Forklar Newtons 2. lov for rotasjon:
\begin{solution}
Newtons 2. lov for rotasjon sier at summen av de ytre dreiemomentene er lik produktet av treghetsmomentet og vinkelakselerasjonen. Matematisk uttrykt som:
\begin{align}
\sum_{i} \tau_{i} &= I \alpha
\end{align}
Her er \(\tau_{i}\) det ytre dreiemomentet, \(I\) er treghetsmomentet, og \(\alpha\) er vinkelakselerasjonen.
\end{solution}
\end{parts}


\question FILL QUESTION
\begin{parts}

\part Beregn vinkelfarten:
\begin{solution}
Vinkelfarten \( w \) kan beregnes ved å bruke formelen for vinkelfart, som er forholdet mellom den lineære hastigheten \( v \) og radiusen \( r \) av sirkelen:

\begin{align}
w &= \frac{v}{r} \\
  &= \frac{\SI{25}{\meter\per\second}}{\SI{90}{\meter}} \\
w &= \doubleunderline{\SI{0.28}{\radian\per\second}}
\end{align}

Vinkelfarten er altså \(\SI{0.28}{\radian\per\second}\).
\end{solution}

\part Beregn antall omdreininger i løpet av \( t = \SI{30}{\second} \):
\begin{solution}
Antall omdreininger kan beregnes ved å multiplisere vinkelfarten \( w \) med tiden \( t \), og deretter konvertere fra radianer til omdreininger:

\begin{align}
\theta &= w \cdot t \\
       &= \SI{0.28}{\radian\per\second} \cdot \SI{30}{\second} \cdot \frac{1 \text{ rev}}{2\pi \text{ rad}} \\
\theta &= \doubleunderline{\SI{1.3}{\text{ rev}}}
\end{align}

Antall omdreininger i løpet av \(\SI{30}{\second}\) er \(\SI{1.3}{\text{ rev}}\).
\end{solution}

\end{parts}

\question Beregn vinkelfarten for et hjul:
\begin{parts}
\part Gitt data:
\begin{solution}
Vi har følgende initialbetingelser:
\begin{align*}
\omega_{1} &= 0 \\
\theta_{1} &= \SI{5.0}{\radian} \\
\theta_{0} &= 0 \\
t &= \SI{2.8}{\second}
\end{align*}
\end{solution}

\part Beregn vinkelakselerasjonen:
\begin{solution}
Den konstante vinkelakselerasjonen er gitt ved:
\begin{align}
\omega_{1} &= \omega_{0} + \alpha \cdot t \\
\alpha &= \frac{\omega_{1} - \omega_{0}}{t} \\
\alpha &= \frac{-\omega_{0}}{t} \quad \text{(I)}
\end{align}
\end{solution}

\part Sett inn likning (I) i bevegelseslikningen:
\begin{solution}
Vi bruker bevegelseslikningen for vinkel:
\begin{align}
\theta_{1} &= \theta_{0} + \omega_{0} t + \frac{1}{2} \alpha t^{2}
\end{align}
Setter inn likning (I):
\begin{align}
\theta_{1} &= \theta_{0} + \omega_{0} t - \frac{1}{2} \frac{\omega_{0}}{t} t^{2} \\
&= \omega_{0} t - \frac{1}{2} \omega_{0} t \\
\theta_{1} &= \frac{1}{2} \omega_{0} t
\end{align}
\end{solution}

\part Løs for $\omega_{0}$:
\begin{solution}
Vi løser med hensyn på $\omega_{0}$:
\begin{align}
\omega_{0} &= \frac{2 \theta_{1}}{t} \\
&= \frac{2 \times \SI{5.0}{\radian}}{\SI{2.8}{\second}} \\
\omega_{0} &= \doubleunderline{\SI{3.6}{\radian\per\second}}
\end{align}
\end{solution}
\end{parts}

\question Forklar resultatet:
\begin{solution}
Vinkelfarten var på \SI{3.6}{\radian\per\second} før hjulet begynte å bremse. Dette betyr at hjulet roterte med denne hastigheten før noen bremsekraft ble påført.
\end{solution}


\question Beregn treghetsmomentet for en solid sfære:
\begin{parts}
\part Treghetsmomentet for en solid sfære om sitt eget massesenter er gitt ved:
\begin{solution}
Treghetsmomentet for en solid sfære om sitt eget massesenter er gitt ved formelen:
\[
I_{\mathrm{cm}} = \frac{2}{5} M R^{2}
\]
hvor \( M \) er massen til sfæren og \( R \) er radiusen.
\end{solution}

\part Bruk parallellakse-teoremet for å finne treghetsmomentet når sfæren roterer om en akse tangent til overflaten:
\begin{solution}
Parallellakse-teoremet sier at treghetsmomentet om en akse som er parallell med en akse gjennom massesenteret er gitt ved:
\[
I = I_{\mathrm{cm}} + M h^{2}
\]
hvor \( h = R \) for en tangent akse.

Setter inn uttrykket for \( I_{\mathrm{cm}} \):
\begin{align}
I &= \frac{2}{5} M R^{2} + M R^{2} \\
  &= \frac{2}{5} M R^{2} + \frac{5}{5} M R^{2} \\
  &= \frac{7}{5} M R^{2}
\end{align}
Derfor er treghetsmomentet når sfæren roterer om en tangent akse \doubleunderline{\(\frac{7}{5} M R^{2}\)}.
\end{solution}
\end{parts}



\question Beregn treghetsmomentet for et hjul:
\begin{parts}

\part Gitt data:
\begin{solution}
\centering
\begin{align}
M & = \SI{8.00}{\kilo\gram} \\
m & = \SI{1.20}{\kilo\gram} \\
d & = \SI{1.00}{\meter}
\end{align}
\end{solution}

\part Treghetsmoment for massen M:
\begin{solution}
Treghetsmomentet for massen M (navet) er gitt ved:
\begin{align}
I_{\text{nav}} & = M R^2
\end{align}
\end{solution}

\part Treghetsmoment for massen m:
\begin{solution}
Treghetsmomentet for massen m (eikene) er gitt ved:
\begin{align}
I_{\text{eike}} & = \frac{1}{3} m L^2
\end{align}
\end{solution}

\part Beregn hele hjulets treghetsmoment:
\begin{solution}
Hele hjulets treghetsmoment er summen av alle delkomponentenes treghetsmoment:
\begin{align}
I & = I_{\text{nav}} + 6 \cdot I_{\text{eike}} \\
  & = M R^2 + 6 \cdot \frac{1}{3} m L^2 \\
  & = \frac{1}{4} M D^2 + \frac{1}{2} m D^2 \\
  & = \frac{1}{4}(M + 2m) D^2 \\
  & = \frac{1}{4}(\SI{8.00}{\kilo\gram} + 2 \cdot \SI{1.20}{\kilo\gram}) \cdot (\SI{1.00}{\meter})^2 \\
  & = \doubleunderline{\SI{2.60}{\kilo\gram\meter^2}}
\end{align}
\end{solution}

\part Grafisk representasjon:
\begin{solution}
PLACE GRAPHICS HERE
\end{solution}

\end{parts}

\end{questions}
\end{document}
