\documentclass[answers,a4paper,12pt]{exam}
\input{preamble.tex}

\title{{\bf{FYS101 Mekanikk}} \\ \Large{\answersareprinted Oblig XX}} 
\author{Institutt for Fysikk, REALTEK}
\date{Uke xx}

\begin{document}
\maketitle

\begin{questions}
\question FILL QUESTION
\begin{parts}

\part Del A: Definer vinkelhastighet (u) og banehastighet (v_t):
\begin{solution}
Vinkelhastighet, ofte betegnet som \( u \), er hastigheten med hvilken en gjenstand roterer rundt en akse. Banehastighet, \( v_t \), er hastigheten til et punkt på en roterende gjenstand langs en bane. Forholdet mellom disse er gitt ved formelen:
\begin{align}
v_{t} &= r \cdot u \cdot t
\end{align}
hvor \( r \) er radiusen til banen, \( u \) er vinkelhastigheten, og \( t \) er tiden.
\end{solution}

\part Del B: Definer vinkelakselerasjon (\(\alpha\)) og tangentiell baneakselerasjon (\(a_t\)):
\begin{solution}
Vinkelakselerasjon, \(\alpha\), er endringen i vinkelhastighet over tid. Tangentiell baneakselerasjon, \(a_t\), er akselerasjonen langs banen til et punkt på en roterende gjenstand. Forholdet mellom disse er gitt ved formelen:
\begin{align}
a_{t} &= r \cdot \alpha
\end{align}
hvor \( r \) er radiusen til banen og \(\alpha\) er vinkelakselerasjonen.
\end{solution}

\end{parts}

\question FILL QUESTION
\begin{parts}
\part Forklar begrepet "rotasjonsdynamikk":
\begin{solution}
Rotasjonsdynamikk er studiet av bevegelse av objekter som roterer rundt en akse. Det involverer begreper som dreiemoment, treghetsmoment, og vinkelakselerasjon. Dreiemoment er kraften som forårsaker rotasjon, mens treghetsmoment er et mål på et objekts motstand mot endringer i rotasjonshastighet.
\end{solution}

\part Forklar formelen for rotasjon:
\begin{solution}
Formelen for rotasjon er gitt ved:
\begin{align}
\sum_{i} \tau_{i} &= I \alpha
\end{align}
Her er \(\sum_{i} \tau_{i}\) summen av de eksterne dreiemomentene, \(I\) er treghetsmomentet, og \(\alpha\) er vinkelakselerasjonen. Denne ligningen uttrykker at summen av de eksterne dreiemomentene er lik produktet av treghetsmomentet og vinkelakselerasjonen.
\end{solution}

\part Beregn vinkelakselerasjonen gitt et treghetsmoment og et eksternt dreiemoment:
\begin{solution}
Anta at vi har et treghetsmoment \(I = \SI{5}{\kilogram\meter\squared}\) og et eksternt dreiemoment \(\tau = \SI{10}{\newton\meter}\). Vi ønsker å finne vinkelakselerasjonen \(\alpha\).

\begin{align}
\sum_{i} \tau_{i} &= I \alpha \\
\SI{10}{\newton\meter} &= \SI{5}{\kilogram\meter\squared} \cdot \alpha \\
\alpha &= \frac{\SI{10}{\newton\meter}}{\SI{5}{\kilogram\meter\squared}} \\
\alpha &= \doubleunderline{\SI{2}{\radian\per\second\squared}}
\end{align}
\end{solution}
\end{parts}

\question FILL QUESTION
\begin{parts}

\part Beregn vinkelfarten:
\begin{solution}
Vinkelfarten, også kjent som angular velocity, kan beregnes ved å bruke formelen:
\begin{align}
w &= \frac{v}{r} \\
  &= \frac{\SI{25}{\meter\per\second}}{\SI{90}{\meter}} \\
w &= \doubleunderline{\SI{0.28}{\radian\per\second}}
\end{align}
Vinkelfarten er dermed \SI{0.28}{\radian\per\second}.
\end{solution}

\part Beregn antall omdreininger i løpet av \SI{30}{\second}:
\begin{solution}
Antall omdreininger kan beregnes ved å bruke formelen for vinkelposisjon:
\begin{align}
\theta &= w \cdot t \\
       &= \SI{0.28}{\radian\per\second} \cdot \SI{30}{\second} \\
       &= \SI{8.4}{\radian}
\end{align}
For å konvertere fra radianer til omdreininger, bruker vi:
\begin{align}
\theta &= \frac{\SI{8.4}{\radian}}{2\pi} \cdot \SI{1}{\rev} \\
       &= \doubleunderline{\SI{1.34}{\rev}}
\end{align}
Antall omdreininger i løpet av \SI{30}{\second} er dermed \SI{1.34}{\rev}.
\end{solution}

\end{parts}


\question Beregn vinkelfarten for et hjul:
\begin{parts}

\part Gitt data:
\begin{solution}
\begin{align}
\omega_{1} &= 0 \\
\theta_{1} &= \SI{5.0}{\radian} \\
\theta_{0} &= 0 \\
t &= \SI{2.8}{\second}
\end{align}
\end{solution}

\part Konstant vinkelakselerasjon:
\begin{solution}
Vi har formelen for vinkelakselerasjon:
\begin{align}
\omega_{1} &= \omega_{0} + \alpha \cdot t \\
\alpha &= \frac{\omega_{1} - \omega_{0}}{t} \\
\alpha &= \frac{-\omega_{0}}{t} \quad \text{(I)}
\end{align}
\end{solution}

\part Beregn vinkelposisjonen:
\begin{solution}
Vi bruker formelen for vinkelposisjon:
\begin{align}
\theta_{1} &= \theta_{0} + \omega_{0} t + \frac{1}{2} \alpha t^{2}
\end{align}
Setter inn likning (I):
\begin{align}
\theta_{1} &= \theta_{0} + \omega_{0} t - \frac{1}{2} \frac{\omega_{0}}{t} t^{2} \\
&= \omega_{0} t - \frac{1}{2} \omega_{0} t \\
\theta_{1} &= \frac{1}{2} \omega_{0} t
\end{align}
\end{solution}

\part Løs for $\omega_{0}$:
\begin{solution}
\begin{align}
\omega_{0} &= \frac{2 \theta_{1}}{t} \\
&= \frac{2 \times \SI{5.0}{\radian}}{\SI{2.8}{\second}} \\
\omega_{0} &= \doubleunderline{\SI{3.6}{\radian\per\second}}
\end{align}
\end{solution}

\end{parts}

\question Vinkelfarten var på \SI{3.6}{\radian\per\second} før hjulet begynte å bremse.



\question Finner treghetsmomentet for en solid sfære som roterer om en akse tangent til overflaten:
\begin{parts}
\part Bruk parallellakse-teoremet for å finne treghetsmomentet:
\begin{solution}
Parallellakse-teoremet sier at treghetsmomentet om en akse som er parallell med en akse gjennom massesenteret er gitt ved:

\[
I = I_{\mathrm{cm}} + M h^2
\]

der \( h = R \) for en akse tangent til overflaten av sfæren. Vi setter inn uttrykket for \( I_{\mathrm{cm}} \):

\[
I_{\mathrm{cm}} = \frac{2}{5} M R^2
\]

Setter dette inn i parallellakse-teoremet:

\begin{align}
I &= \frac{2}{5} M R^2 + M R^2 \\
  &= \frac{2}{5} M R^2 + \frac{5}{5} M R^2 \\
  &= \frac{7}{5} M R^2
\end{align}

Derfor er treghetsmomentet for en solid sfære som roterer om en tangent akse \doubleunderline{\(\frac{7}{5} M R^2\)}.
\end{solution}
\end{parts}



\question Beregn treghetsmomentet for et hjul:
\begin{parts}

\part Gitt data:
\begin{solution}
\centering
\begin{align}
M & = \SI{8.00}{\kilo\gram} \\
m & = \SI{1.20}{\kilo\gram} \\
d & = \SI{1.00}{\meter}
\end{align}
\end{solution}

\part Beregn treghetsmomentet for massen M:
\begin{solution}
Treghetsmomentet for massen M er gitt ved:
\begin{align}
I_{\text{hjul}} & = M R^2
\end{align}
\end{solution}

\part Beregn treghetsmomentet for massen m:
\begin{solution}
Treghetsmomentet for massen m er gitt ved:
\begin{align}
I_{\text{eike}} & = \frac{1}{3} m L^2
\end{align}
\end{solution}

\part Beregn hele hjulets treghetsmoment:
\begin{solution}
Hele hjulets treghetsmoment er summen av alle delkomponentenes treghetsmoment:
\begin{align}
I & = I_{\text{hjul}} + 6 \cdot I_{\text{eike}} \\
  & = M R^2 + 6 \cdot \frac{1}{3} m L^2 \\
  & = \frac{1}{4} M D^2 + \frac{1}{2} m D^2 \\
  & = \frac{1}{4}(M + 2m) D^2 \\
  & = \frac{1}{4}(\SI{8.00}{\kilo\gram} + 2 \cdot \SI{1.20}{\kilo\gram}) \cdot (\SI{1.00}{\meter})^2 \\
  & = \doubleunderline{\SI{2.60}{\kilo\gram\meter^2}}
\end{align}
\end{solution}

\end{parts}


\end{questions}
\end{document}
