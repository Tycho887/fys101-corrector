\documentclass[answers,a4paper,12pt]{exam}
\input{preamble.tex}

\title{{\bf{FYS101 Mekanikk}} \\ \Large{\answersareprinted Oblig XX}} 
\author{Institutt for Fysikk, REALTEK}
\date{Uke xx}

\begin{document}
\maketitle

\begin{questions}
\question Definer nøkkelbegreper i "Del A":
\begin{parts}

\part Definer terminalfart:
\begin{solution}
Terminalfart er den største farten et legeme oppnår ved fall gjennom luft fra store høyder. Det skjer når tyngdekraften balanseres av luftmotstanden.
\end{solution}

\part Definer sentripetalkraft:
\begin{solution}
Sentripetalkraft er netto kraft som påvirker et legeme til å følge en sirkulær bane. Den virker mot sentrum av sirkelen.
\end{solution}

\part Definer sentrifugalkraft:
\begin{solution}
Sentrifugalkraft er en fiktiv kraft som brukes til å forklare bevegelse i et akselerert referansesystem. Den oppleves som en kraft som trekker et objekt bort fra sentrum av sirkulær bevegelse.
\end{solution}

\part Definer massesenter:
\begin{solution}
Massesenter, også kjent som tyngdepunktet i et homogent tyngdefelt, er det punktet i (eller utenfor) et legeme som beveger seg som om legemets masse var samlet i det punktet. Det er også det punktet der et legeme kan balanseres.
\end{solution}

\end{parts}


\question Del $B$: Analyser kreftene på en bok i ro på et bord
\begin{parts}

\part Forklar tyngdekraften og normalkraften som virker på boka:
\begin{solution}
Tyngdekraften på boka ($F_g$) er en fjernkraft, som virker mellom legemer som ikke er i fysisk kontakt. Den virker mellom boka og jorda. Normalkraften på boka ($F_n$) er en kontaktkraft, som virker mellom legemer som er i kontakt med hverandre, i dette tilfellet mellom boka og bordet.
\end{solution}

\part Analyser kreftene i $y$-retningen:
\begin{solution}
Ettersom boka er i ro, har vi at $\sum F_y = 0$. For $y$-retningen (der vi har definert nedover som positiv) har vi:
\begin{align}
\sum F_{y} &= 0 \\
F_{g} - F_{n} &= 0 \\
F_{g} &= F_{n}
\end{align}
Dette betyr at kreftene er like store og motsatt rettet fordi boka er i ro. De er imidlertid ulike typer krefter og har sine respektive motkrefter.
\end{solution}

\part Illustrer kreftene med vektorgrafikk:
\begin{solution}
PLACE GRAPHICS HERE
\end{solution}

\part Definer motkreftene til tyngdekraften og normalkraften:
\begin{solution}
Motkraften til normalkraften ($F_{n}^{\prime}$) er den kraften bordet utøver på boka. Motkraften til tyngdekraften ($F_{g}^{\prime}$) er den kraften boka utøver på jorda.
\end{solution}

\end{parts}



\question Analyser systemet og beregn akselerasjonen og sluttfarten til kloss 2:
\begin{parts}

\part Del opp systemet i to delsystemer og definer kreftene:
\begin{solution}
Vi deler opp systemet i to delsystemer:
\begin{itemize}
    \item System 1: Kloss på skrått plan med koordinatsystem parallelt med bevegelsesretningen.
    \item System 2: Kloss som faller, definerer positiv retning nedover.
\end{itemize}
Stram, masseløs snor gir:
\begin{align*}
    T_{1} &= T_{2} = T \quad \text{(forutsetter at snoren er stram)} \\
    a_{1} &= a_{2} = a \quad \text{(retninger er definert som positivt)}
\end{align*}
Vi analyserer hvert delsystem for seg for å finne akselerasjonen. Vi antar at denne er konstant slik at vi kan bruke bevegelsesligningene til å finne kloss 2 sin fart.
\end{solution}

\part Analyser system 1 og finn normalkraften:
\begin{solution}
System 1 - Vi må dekomponere tyngdekraften for å analysere hver retning for seg:
\begin{align*}
    F_{g x} &= F_{g} \sin \theta = m_{1} g \sin \theta \\
    F_{g y} &= F_{g} \cos \theta = m_{1} g \cos \theta \\
    f &= \mu_{k} F_{n}
\end{align*}
Bruker Newtons andre lov, $\Sigma \vec{F} = m \vec{a}$ i $y$-retning:
\begin{align*}
    \Sigma F_{y} &= m a_{y} = 0 \quad \text{(ingen bevegelse i $y$-retning)} \\
    F_{n} + F_{g y} &= 0 \\
    F_{n} - m_{1} g \cos \theta &= 0
\end{align*}
Derfor er normalkraften:
\begin{equation*}
    F_{n} = m_{1} g \cos \theta
\end{equation*}
\end{solution}

\part Analyser system 1 i $x$-retning:
\begin{solution}
I $x$-retning:
\begin{align*}
    \Sigma F_{x} &= m a_{x}, \quad a_{x} = a \\
    T_{1} + F_{g x} + f &= m_{1} a \\
    T_{1} - m_{1} g \sin \theta - \mu_{k} F_{n} &= m_{1} a
\end{align*}
Setter inn for $F_{n}$:
\begin{equation*}
    T_{1} - m_{1} g \sin \theta - \mu_{k} m_{1} g \cos \theta = m_{1} a \tag{*}
\end{equation*}
\end{solution}

\part Analyser system 2 og finn uttrykk for spenningen:
\begin{solution}
Ser kun på $y$-retning, har ingen krefter i $x$-retning:
\begin{align*}
    \Sigma F_{y} &= m_{2} a \\
    T_{2} + F_{g_{2}} &= m_{2} a \\
    -T_{2} + m_{2} g &= m_{2} a \\
    T_{2} &= m_{2} g - m_{2} a
\end{align*}
(fortegn utifra positiv retning nedover)
\end{solution}

\part Sett $T_{1} = T_{2} = T$ og løs for akselerasjonen $a$:
\begin{solution}
Setter $T_{1} = T_{2} = T$ og setter inn i (*):
\begin{align*}
    m_{2} g - m_{2} a - m_{1} g \sin \theta - \mu_{k} m_{1} g \cos \theta &= m a \\
    m_{1} a + m_{2} a &= m_{2} g - m_{1} g \sin \theta - \mu_{k} m_{1} g \cos \theta \\
    a(m_{1} + m_{2}) &= g(m_{2} - m_{1} \sin \theta - \mu_{k} m_{1} g \cos \theta)
\end{align*}
Løser mhp $a$:
\begin{equation*}
    a = \frac{g(m_{2} - m_{1} \sin \theta - \mu_{k} m_{1} g \cos \theta)}{m_{1} + m_{2}}
\end{equation*}
Setter inn verdier:
\begin{align*}
    a &= \frac{9,81 \, \mathrm{m/s^2} \left(0,200 \, \mathrm{kg} - 0,250 \, \mathrm{kg} \cdot \sin 30^{\circ} - 0,100 \cdot 9,81 \, \mathrm{m/s^2} \cdot \cos 30^{\circ}\right)}{0,250 \, \mathrm{kg} + 0,200 \, \mathrm{kg}} \\
    &= \doubleunderline{1,163 \, \mathrm{m/s^2}}
\end{align*}
\end{solution}

\part Beregn sluttfarten til kloss 2:
\begin{solution}
Bruker bevegelsesligning:
\begin{align*}
    v^2 &= v_{0}^2 + 2 a \Delta y, \quad v_{0} = 0 \\
    v &= \sqrt{2 a \Delta y} \\
    &= \sqrt{2 \cdot 1,163 \, \mathrm{m/s^2} \cdot 0,300 \, \mathrm{m}} \\
    &= \doubleunderline{0,84 \, \mathrm{m/s}}
\end{align*}
(To gjeldende siffer i svaret pga $30^{\circ}$ kan være alt fra $29,6^{\circ}$ til $30,4^{\circ}$)
\end{solution}

\end{parts}



\question Beregn sentripetalkraften og dekomponer kreftene for en bil:
\begin{parts}

\part Gitt data for hastighet og masse:
\begin{solution}
\centering
\begin{align}
m & = \SI{750}{\kilo\gram} \\
r & = \SI{160}{\meter} \\
v & = \SI{90}{\kilo\meter\per\hour} \\
  & = \SI{90}{\kilo\meter\per\hour} \cdot \frac{\SI{1000}{\meter}}{\SI{1}{\kilo\meter}} \cdot \frac{\SI{1}{\hour}}{\SI{3600}{\second}} \\
  & = \doubleunderline{\SI{25}{\meter\per\second}}
\end{align}
\end{solution}

\part Beregn sentripetalkraften:
\begin{solution}
\centering
PLACE GRAPHICS HERE \\
\begin{align}
F_c & = m \cdot \frac{v^2}{r} \\
    & = \SI{750}{\kilo\gram} \cdot \frac{\left(\SI{25}{\meter\per\second}\right)^2}{\SI{160}{\meter}} \\
    & = \doubleunderline{\SI{2921.875}{\newton}}
\end{align}
\end{solution}

\part Dekomponer kreftene i $x$- og $y$-retning:
\begin{solution}
\centering
\begin{align}
\sum F_{x} &= m a_{x}, \quad a_{x} = \frac{v^2}{r} \quad \text{(sentripetalakselerasjon)} \\
F_{n x} &= \frac{v^2}{r} \\
F_{n} \sin \theta &= \frac{v^2}{r} \quad (*)
\end{align}
\end{solution}

\end{parts}


\end{questions}
\end{document}
