\documentclass[answers,a4paper,12pt]{exam}
\input{preamble.tex}

\title{{\bf{FYS101 Mekanikk}} \\ \Large{\answersareprinted Oblig XX}} 
\author{Institutt for Fysikk, REALTEK}
\date{Uke xx}

\begin{document}
\maketitle

\begin{questions}
\question Definer nøkkelbegreper i "Del A":
\begin{parts}

\part Definer terminalfart:
\begin{solution}
Terminalfart er den største farten et legeme oppnår ved fall gjennom luft fra store høyder. Det skjer når tyngdekraften er balansert av luftmotstanden.
\end{solution}

\part Definer sentripetalkraft:
\begin{solution}
Sentripetalkraft er netto kraft som påvirker et legeme til å følge en sirkulær bane. Den virker mot sentrum av sirkelen.
\end{solution}

\part Definer sentrifugalkraft:
\begin{solution}
Sentrifugalkraft er en fiktiv kraft som brukes til å forklare bevegelse i et akselerert referansesystem. Den oppleves som en kraft som trekker et legeme bort fra sentrum av en sirkulær bane.
\end{solution}

\part Definer massesenter:
\begin{solution}
Massesenter, også kjent som tyngdepunktet i et homogent tyngdefelt, er det punktet i (eller utenfor) et legeme som beveger seg som om legemets masse var samlet i dette punktet. Det er også det punktet der et legeme kan balanseres.
\end{solution}

\end{parts}


\question Del B: Analyser kreftene på en bok i ro på et bord

\begin{parts}
\part Forklar de involverte kreftene:
\begin{solution}
Tyngdekraften på boka ($F_g$) er en fjernkraft, som virker mellom legemer som ikke er i fysisk kontakt. Den virker mellom boka og jorda. Normalkraften på boka ($F_n$) er en kontaktkraft, som virker mellom legemer som er i kontakt med hverandre, i dette tilfellet mellom boka og bordet. Ettersom boka er i ro, har vi at $\sum F = 0$.
\end{solution}

\part Analyser kreftene i y-retningen:
\begin{solution}
For y-retningen (der vi har definert nedover som positiv) har vi:
\begin{align}
\sum F_{y} &= 0 \\
F_{g} - F_{n} &= 0 \\
F_{g} &= F_{n}
\end{align}
Dette betyr at kreftene er like store og motsatt rettet fordi boka er i ro, men de er ulike typer krefter og har sine respektive motkrefter.
\end{solution}

\part Illustrer kreftene og deres motkrefter:
\begin{solution}
PLACE GRAPHICS HERE

\includegraphics[max width=\textwidth, center]{path/to/image}

\begin{itemize}
    \item $F_{n}$ - Normalkraft
    \item $F_{n}^{\prime}$ - Motkraft til normalkraft
    \item $F_{g}$ - Tyngdekraft
    \item $F_{g}^{\prime}$ - Motkraft til tyngdekraft
\end{itemize}
\end{solution}
\end{parts}



\question Analyser systemet og beregn akselerasjonen og hastigheten til kloss 2:
\begin{parts}

\part Del opp systemet i to delsystemer:
\begin{solution}
Vi deler opp systemet i to delsystemer:
\begin{enumerate}
    \item Kloss på skrått plan - koordinatsystem parallelt med bevegelsesretning.
    \item Kloss som faller - definerer positiv retning nedover.
\end{enumerate}
Stram, masseløs snor gir $T_{1}=T_{2}=T$ og vi forutsetter $a_{1}=a_{2}=a$ med retninger som er definert som positivt.
\end{solution}

\part Analyser delsystem 1:
\begin{solution}
Vi må dekomponere tyngdekraften $F_g$ for å analysere hver retning for seg:
\begin{align*}
F_{g x} &= F_{g} \sin \theta = m_{1} g \sin \theta \\
F_{g y} &= F_{g} \cos \theta = m_{1} g \cos \theta \\
f &= \mu_{k} F_{n}
\end{align*}
Bruker Newtons andre lov, $\Sigma \vec{F} = m \vec{a}$ i $y$-retning:
\begin{align*}
\Sigma F_{y} &= m a_{y} = 0 \quad (\text{ingen bevegelse i } y\text{-retning}) \\
F_{n} + F_{g y} &= 0 \\
F_{n} - m_{1} g \cos \theta &= 0
\end{align*}
Derfor er $F_{n} = m_{1} g \cos \theta$.

I $x$-retning:
\begin{align*}
\Sigma F_{x} &= m a_{x}, \quad a_{x} = a \\
T_{1} + F_{g x} + f &= m_{1} a \\
T_{1} - m_{1} g \sin \theta - \mu_{k} F_{n} &= m_{1} a
\end{align*}
Setter inn for $F_{n}$:
\begin{equation*}
T_{1} - m_{1} g \sin \theta - \mu_{k} m_{1} g \cos \theta = m_{1} a \tag{*}
\end{equation*}
\end{solution}

\part Analyser delsystem 2:
\begin{solution}
Ser kun på $y$-retning, ingen krefter i $x$-retning:
\begin{align*}
\Sigma F_{y} &= m_{2} a \\
T_{2} + F_{g_{2}} &= m_{2} a \\
-T_{2} + m_{2} g &= m_{2} a \\
T_{2} &= m_{2} g - m_{2} a
\end{align*}
(fortegn utifra positiv retning nedover)
Setter $T_{1} = T_{2} = T$ og setter inn i (*):
\begin{align*}
m_{2} g - m_{2} a - m_{1} g \sin \theta - \mu_{k} m_{1} g \cos \theta &= m a \\
m_{1} a + m_{2} a &= m_{2} g - m_{1} g \sin \theta - \mu_{k} m_{1} g \cos \theta \\
a(m_{1} + m_{2}) &= g(m_{2} - m_{1} \sin \theta - \mu_{k} m_{1} g \cos \theta)
\end{align*}
Løser mhp $a$:
\begin{equation*}
a = \frac{g(m_{2} - m_{1} \sin \theta - \mu_{k} m_{1} g \cos \theta)}{m_{1} + m_{2}}
\end{equation*}
Setter inn verdier:
\begin{align*}
a &= \frac{9.81 \, \mathrm{m/s^2} \left(0.200 \, \mathrm{kg} - 0.250 \, \mathrm{kg} \cdot \sin 30^{\circ} - 0.100 \cdot 9.81 \, \mathrm{m/s^2} \cdot \cos 30^{\circ}\right)}{0.250 \, \mathrm{kg} + 0.200 \, \mathrm{kg}} \\
&= \doubleunderline{1.163 \, \mathrm{m/s^2}}
\end{align*}
\end{solution}

\part Beregn hastigheten til kloss 2:
\begin{solution}
Bruker bevegelsesligning:
\begin{align*}
v^{2} &= v_{0}^{2} + 2 a \Delta y, \quad v_{0} = 0 \\
v &= \sqrt{2 a \Delta y} \\
&= \sqrt{2 \cdot 1.163 \, \mathrm{m/s^2} \cdot 0.300 \, \mathrm{m}} \\
&= \doubleunderline{0.84 \, \mathrm{m/s}}
\end{align*}
(To gjeldende siffer i svaret pga $30^{\circ}$ kan være alt fra $29.6^{\circ}$ til $30.4^{\circ}$)
\end{solution}

\end{parts}



\question Beregn kreftene som virker på en bil i en sirkulær bevegelse:
\begin{parts}
\part Gitt data for hastighet og radius:
\begin{solution}
\centering
\begin{align}
m & = \SI{750}{\kilo\gram} \\
r & = \SI{160}{\meter} \\
v & = \SI{90}{\kilo\meter\per\hour} \\
  & = \SI{90}{\kilo\meter\per\hour} \cdot \frac{\SI{1000}{\meter}}{\SI{1}{\kilo\meter}} \cdot \frac{\SI{1}{\hour}}{\SI{3600}{\second}} \\
  & = \doubleunderline{\SI{25}{\meter\per\second}}
\end{align}
\end{solution}

\part Beregn den sentripetale akselerasjonen:
\begin{solution}
\centering
\begin{align}
a_{x} & = \frac{v^2}{r} \\
      & = \frac{\left(\SI{25}{\meter\per\second}\right)^2}{\SI{160}{\meter}} \\
      & = \doubleunderline{\SI{3.90625}{\meter\per\second\squared}}
\end{align}
\end{solution}

\part Beregn normalkraften i $x$-retning:
\begin{solution}
\centering
\begin{align}
F_{n x} & = m \cdot a_{x} \\
        & = \SI{750}{\kilo\gram} \cdot \SI{3.90625}{\meter\per\second\squared} \\
        & = \doubleunderline{\SI{2929.6875}{\newton}}
\end{align}
\end{solution}

\part Beregn normalkraften i $y$-retning:
\begin{solution}
\centering
\begin{align}
F_{n} \sin \theta & = \frac{v^2}{r} \\
\end{align}
\end{solution}
\end{parts}

\begin{center}
PLACE GRAPHICS HERE
\end{center}

\question Forklar kreftene som virker på bilen:
\begin{parts}
\part Forklar hvordan friksjonskraften påvirker bilen:
\begin{solution}
Friksjonskraften må være tilstrekkelig for å holde bilen i sirkulær bevegelse uten å skli. Dette betyr at friksjonskraften må balansere den sentripetale kraften som kreves for å holde bilen på banen.
\end{solution}

\part Forklar hvordan normalkraften virker på bilen:
\begin{solution}
Normalkraften virker vinkelrett på underlaget og balanserer tyngdekraften som virker på bilen. Når bilen beveger seg i en sirkulær bane, vil normalkraften også ha en komponent som bidrar til den sentripetale akselerasjonen.
\end{solution}
\end{parts}

\end{questions}
\end{document}
