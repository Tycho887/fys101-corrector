\documentclass[answers,a4paper,12pt]{exam}
\input{preamble.tex}

\title{{\bf{FYS101 Mekanikk}} \\ \Large{\answersareprinted Oblig XX}} 
\author{Institutt for Fysikk, REALTEK}
\date{Uke xx}

\begin{document}
\maketitle

\begin{questions}
\question Definer nøkkelbegreper i "Del A":
\begin{parts}
\part Definer terminalfart:
\begin{solution}
Terminalfart er den største farten et legeme oppnår ved fall gjennom luft fra store høyder. Det skjer når tyngdekraften er balansert av luftmotstanden.
\end{solution}

\part Definer sentripetalkraft:
\begin{solution}
Sentripetalkraft er netto kraft som påvirker et legeme til å følge en sirkulær bane. Den virker mot sentrum av sirkelen.
\end{solution}

\part Definer sentrifugalkraft:
\begin{solution}
Sentrifugalkraft er en fiktiv kraft som brukes til å forklare bevegelse i et akselerert referansesystem. Den oppleves som en kraft som trekker et legeme bort fra sentrum av en sirkulær bane.
\end{solution}

\part Definer massesenter:
\begin{solution}
Massesenter, også kjent som tyngdepunktet i et homogent tyngdefelt, er det punktet i (eller utenfor) et legeme som beveger seg som om legemets masse var samlet i dette punktet. Det er også det punktet der et legeme kan balanseres.
\end{solution}
\end{parts}

\question Del $B$: Analyser kreftene på en bok i ro på et bord
\begin{parts}

\part Forklar tyngdekraften og normalkraften som virker på boka.
\begin{solution}
Tyngdekraften på boka ($F_g$) er en fjernkraft, som virker mellom legemer som ikke er i fysisk kontakt. Den virker mellom boka og jorda. Normalkraften på boka ($F_n$) er en kontaktkraft, som virker mellom legemer som er i kontakt med hverandre, i dette tilfellet mellom boka og bordet. Ettersom boka er i ro, har vi $\sum F = 0$.
\end{solution}

\part Analyser kreftene i $y$-retningen.
\begin{solution}
For $y$-retningen (der vi har definert nedover som positiv) har vi:
\begin{align}
\sum F_{y} &= 0 \\
F_{g} - F_{n} &= 0 \\
F_{g} &= F_{n}
\end{align}
De er altså like store og motsatt rettet fordi boka er i ro, men de er ulike typer krefter og har sine respektive motkrefter.
\end{solution}

\part Illustrer kreftene og deres motkrefter.
\begin{solution}
Placeholder for graphics: \includegraphics[width=0.5\linewidth]{path/to/image} \\
$F_{n}$ - normalkraft \\
$F_{n}^{\prime}$ - motkraft til normalkraft \\
$F_{g}$ - tyngdekraft \\
$F_{g}^{\prime}$ - motkraft til tyngdekraft
\end{solution}

\end{parts}

\question FILL QUESTION
\begin{parts}

\part Del C: Beregn akselerasjonen og hastigheten til kloss 2
\begin{solution}
Vi deler opp systemet i to delsystemer:
\begin{enumerate}
    \item Kloss på skrått plan - bruker koordinatsystem parallelt med bevegelsesretning.
    \item Kloss i fritt fall - definerer positiv retning nedover.
\end{enumerate}

Vi analyserer hvert delsystem for seg for å finne akselerasjonen. Vi antar at denne er konstant slik at vi kan bruke bevegelsesligningene til å finne kloss 2 sin fart.

\textbf{System 1:} Vi må dekomponere tyngdekraften for å analysere hver retning for seg:
\begin{align*}
    F_{g x} &= F_{g} \sin \theta = m_{1} g \sin \theta \\
    F_{g y} &= F_{g} \cos \theta = m_{1} g \cos \theta \\
    f &= \mu_{k} F_{n}
\end{align*}

Bruker Newtons andre lov, $\Sigma \vec{F} = m \vec{a}$, i $y$-retning:
\begin{align*}
    \Sigma F_{y} &= m_{1} a_{y} = 0 \quad (\text{ingen bevegelse i } y\text{-retning}) \\
    F_{n} - m_{1} g \cos \theta &= 0 \\
    F_{n} &= m_{1} g \cos \theta
\end{align*}

I $x$-retning:
\begin{align*}
    \Sigma F_{x} &= m_{1} a \\
    T_{1} - m_{1} g \sin \theta - \mu_{k} F_{n} &= m_{1} a
\end{align*}

Setter inn for $F_{n}$:
\begin{equation*}
    T_{1} - m_{1} g \sin \theta - \mu_{k} m_{1} g \cos \theta = m_{1} a \tag{*}
\end{equation*}

\textbf{System 2:} Ser kun på $y$-retning, har ingen krefter i $x$-retning:
\begin{align*}
    \Sigma F_{y} &= m_{2} a \\
    -T_{2} + m_{2} g &= m_{2} a \\
    T_{2} &= m_{2} g - m_{2} a
\end{align*}

Setter $T_{1} = T_{2} = T$ og setter inn i (*):
\begin{align*}
    m_{2} g - m_{2} a - m_{1} g \sin \theta - \mu_{k} m_{1} g \cos \theta &= (m_{1} + m_{2}) a
\end{align*}

Alle ledd med $a$ på venstre side og alle ledd med $g$ på høyre side:
\begin{align*}
    a(m_{1} + m_{2}) &= g(m_{2} - m_{1} \sin \theta - \mu_{k} m_{1} g \cos \theta)
\end{align*}

Løser mhp $a$:
\begin{align*}
    a &= \frac{g(m_{2} - m_{1} \sin \theta - \mu_{k} m_{1} g \cos \theta)}{m_{1} + m_{2}}
\end{align*}

Setter inn verdier:
\begin{align*}
    a &= \frac{9.81 \, \mathrm{m/s^2} \left(0.200 \, \mathrm{kg} - 0.250 \, \mathrm{kg} \cdot \sin 30^{\circ} - 0.100 \cdot 9.81 \, \mathrm{m/s^2} \cdot \cos 30^{\circ}\right)}{0.250 \, \mathrm{kg} + 0.200 \, \mathrm{kg}} \\
    &= \doubleunderline{1.163 \, \mathrm{m/s^2}}
\end{align*}

\textbf{Bevegelsesligning:}
\begin{align*}
    v^2 &= v_0^2 + 2a\Delta y, \quad v_0 = 0 \\
    v &= \sqrt{2a\Delta y} \\
    &= \sqrt{2 \cdot 1.163 \, \mathrm{m/s^2} \cdot 0.300 \, \mathrm{m}} \\
    &= \doubleunderline{0.84 \, \mathrm{m/s}}
\end{align*}

(To gjeldende siffer i svaret pga $30^{\circ}$ kan være alt fra $29.6^{\circ}$ til $30.4^{\circ}$)
\end{solution}

\end{parts}


\question FILL QUESTION
\begin{parts}

\part Del A: Beregn hastigheten til bilen:
\begin{solution}
\centering
\begin{align}
v & = \SI{90}{\kilo\meter\per\hour} \\
  & = \SI{90}{\kilo\meter\per\hour} \cdot \frac{\SI{1000}{\meter}}{\SI{1}{\kilo\meter}} \cdot \frac{\SI{1}{\hour}}{\SI{3600}{\second}} \\
  & = \doubleunderline{\SI{25}{\meter\per\second}}
\end{align}
\end{solution}

\part Del B: Beregn den sentripetale kraften:
\begin{solution}
\centering
\begin{align}
F_c & = m \cdot \frac{v^2}{r} \\
    & = \SI{750}{\kilo\gram} \cdot \frac{\left(\SI{25}{\meter\per\second}\right)^2}{\SI{160}{\meter}} \\
    & = \doubleunderline{\SI{2921.875}{\newton}}
\end{align}
\end{solution}

\part Del C: Dekomponer kreftene i $x$- og $y$-retning:
\begin{solution}
\centering
\begin{align}
\sum F_{x} &= m a_{x}, \quad a_{x} = \frac{v^2}{r} \quad \text{(sentripetalakselerasjon)} \\
F_{n x} &= \frac{v^2}{r} \\
F_{n} \sin \theta &= \frac{v^2}{r} \quad (*)
\end{align}
\end{solution}

\end{parts}


Note: The problem statement and the context for the calculations are not fully provided, so placeholders and assumptions have been made based on the given equations and context.
\end{questions}
\end{document}
