\documentclass[answers,a4paper,12pt]{exam}
\input{preamble.tex}

\title{{\bf{FYS101 Mekanikk}} \\ \Large{\answersareprinted Oblig XX}} 
\author{Institutt for Fysikk, REALTEK}
\date{Uke xx}

\begin{document}
\maketitle

\begin{questions}
\question Definer nøkkelbegreper i "Del A":
\begin{parts}
\part Definer terminalfart:
\begin{solution}
Terminalfart er den største farten et legeme oppnår ved fall gjennom luft fra store høyder. Det skjer når tyngdekraften er balansert av luftmotstanden.
\end{solution}

\part Definer sentripetalkraft:
\begin{solution}
Sentripetalkraft er netto kraft som påvirker et legeme til å følge en sirkulær bane. Den virker mot sentrum av sirkelen.
\end{solution}

\part Definer sentrifugalkraft:
\begin{solution}
Sentrifugalkraft er en fiktiv kraft som brukes til å forklare bevegelse i et akselerert referansesystem. Den oppleves som en kraft som trekker et legeme bort fra sentrum av sirkulær bevegelse.
\end{solution}

\part Definer massesenter:
\begin{solution}
Massesenter, også kjent som tyngdepunktet i et homogent tyngdefelt, er:
\begin{itemize}
  \item Det punktet i (eller utenfor) et legeme som beveger seg som om legemets masse var samlet i dette punktet.
  \item Det punktet der et legeme kan balanseres.
\end{itemize}
\end{solution}
\end{parts}

latex
\question Del $B$: Eks-bok i ro på bordet
\begin{parts}

\part Forklar tyngdekraften og normalkraften på boka:
\begin{solution}
Tyngdekraften på boka ($F_g$) er en fjernkraft som virker mellom legemer som ikke er i fysisk kontakt. Den virker mellom boka og jorda. Normalkraften på boka ($F_n$) er en kontaktkraft som virker mellom legemer som er i kontakt med hverandre, i dette tilfellet mellom boka og bordet. Ettersom boka er i ro, har vi $\sum F = 0$.
\end{solution}

\part Analyser kreftene i $y$-retning:
\begin{solution}
For $y$-retning (der vi har definert nedover som positiv) har vi:
\begin{align}
\sum F_{y} &= 0 \\
F_{g} - F_{n} &= 0 \\
F_{g} &= F_{n}
\end{align}
De er altså like store og motsatt rettet fordi boka er i ro, men de er ulike typer krefter og har sine respektive motkrefter.
\end{solution}

\part Illustrer kreftene:
\begin{solution}
\centering
\includegraphics[width=0.5\linewidth]{path/to/image}
\end{solution}

\part Definer motkreftene:
\begin{solution}
\begin{itemize}
    \item $F_{n}$ - Normalkraft
    \item $F_{n}^{\prime}$ - Motkraft til normalkraft
    \item $F_{g}$ - Tyngdekraft
    \item $F_{g}^{\prime}$ - Motkraft til tyngdekraft
\end{itemize}
\end{solution}

\end{parts}


\question Beregn akselerasjonen og hastigheten til kloss 2 i systemet:

\begin{parts}

\part Del A: Systemparametere
\begin{solution}
Vi har følgende parametere for systemet:
\begin{align*}
m_{1} &= \SI{250}{\gram} = \SI{0.250}{\kilo\gram} \\
m_{2} &= \SI{200}{\gram} = \SI{0.200}{\kilo\gram} \\
\theta &= 30^{\circ} \\
\mu_{k} &= 0.100 \\
\Delta y &= \SI{30.0}{\centi\meter} = \SI{0.300}{\meter}
\end{align*}
\end{solution}

\part Del B: Analysering av delsystem 1
\begin{solution}
Vi dekomponerer tyngdekraften for kloss 1:
\begin{align*}
F_{g x} &= F_{g} \sin \theta = m_{1} g \sin \theta \\
F_{g y} &= F_{g} \cos \theta = m_{1} g \cos \theta \\
f &= \mu_{k} F_{n}
\end{align*}

Bruker Newtons andre lov i y-retning:
\begin{align*}
\sum F_{y} &= m_{1} a_{y} = 0 \quad (\text{ingen bevegelse i y-retning}) \\
F_{n} - m_{1} g \cos \theta &= 0 \\
F_{n} &= m_{1} g \cos \theta
\end{align*}

I x-retning:
\begin{align*}
\sum F_{x} &= m_{1} a \\
T_{1} - m_{1} g \sin \theta - \mu_{k} F_{n} &= m_{1} a
\end{align*}
\end{solution}

\part Del C: Analysering av delsystem 2
\begin{solution}
I y-retning for kloss 2:
\begin{align*}
\sum F_{y} &= m_{2} a \\
-T_{2} + m_{2} g &= m_{2} a \\
T_{2} &= m_{2} g - m_{2} a
\end{align*}

Setter $T_{1} = T_{2} = T$ og setter inn i ligningen:
\begin{align*}
m_{2} g - m_{2} a - m_{1} g \sin \theta - \mu_{k} m_{1} g \cos \theta &= m_{1} a
\end{align*}

Løsning for akselerasjon $a$:
\begin{align*}
a &= \frac{g\left(m_{2} - m_{1} \sin \theta - \mu_{k} m_{1} g \cos \theta\right)}{m_{1} + m_{2}}
\end{align*}

Setter inn verdier:
\begin{align*}
a &= \frac{\SI{9.81}{\meter\per\second\squared}\left(\SI{0.200}{\kilo\gram} - \SI{0.250}{\kilo\gram} \cdot \sin 30^{\circ} - 0.100 \cdot \SI{9.81}{\meter\per\second\squared} \cdot \cos 30^{\circ}\right)}{\SI{0.250}{\kilo\gram} + \SI{0.200}{\kilo\gram}} \\
&= \doubleunderline{\SI{1.163}{\meter\per\second\squared}}
\end{align*}
\end{solution}

\part Del D: Beregning av hastighet for kloss 2
\begin{solution}
Bruker bevegelsesligningen:
\begin{align*}
v^{2} &= v_{0}^{2} + 2 a \Delta y, \quad v_{0} = 0 \\
v &= \sqrt{2 a \Delta y} \\
&= \sqrt{2 \cdot \SI{1.163}{\meter\per\second\squared} \cdot \SI{0.300}{\meter}} \\
&= \doubleunderline{\SI{0.84}{\meter\per\second}}
\end{align*}

(To gjeldende siffer i svaret på grunn av usikkerhet i vinkelen $\theta$ som kan variere mellom $29.6^{\circ}$ og $30.4^{\circ}$.)
\end{solution}

\end{parts}

latex
\question Beregn sentripetalkraften for en bil:
\begin{parts}

\part Gitt data for hastighet:
\begin{solution}
\centering
\begin{align}
v & = \SI{90}{\kilo\meter\per\hour} \\
  & = \SI{90}{\kilo\meter\per\hour} \cdot \frac{\SI{1000}{\meter}}{\SI{1}{\kilo\meter}} \cdot \frac{\SI{1}{\hour}}{\SI{3600}{\second}} \\
  & = \doubleunderline{\SI{25}{\meter\per\second}}
\end{align}
\end{solution}

\part Beregn sentripetalkraften:
\begin{solution}
\centering
\includegraphics[width=0.5\linewidth]{path/to/image} \\
\begin{align}
F_c & = m \cdot \frac{v^2}{r} \\
    & = \SI{750}{\kilo\gram} \cdot \frac{\left(\SI{25}{\meter\per\second}\right)^2}{\SI{160}{\meter}} \\
    & = \doubleunderline{\SI{2921.875}{\newton}}
\end{align}
\end{solution}

\part Forklar kreftene i $x$- og $y$-retning:
\begin{solution}
Når kraften fra underlaget skal stå normalt på underlaget, må friksjonskraften på bilen være null. Vi dekomponerer kreftene i $x$- og $y$-retning (se figur - disse er ikke parallelle og normale på skråplanet slik vi pleier).

\begin{align}
& \sum F_{x} = m a_{x}, \quad a_{x} = \frac{v^{2}}{r} \quad \text{(sentripetalakselerasjon)} \\
& F_{n x} = \frac{v^{2}}{r} \\
& F_{n} \sin \theta = \frac{v^{2}}{r} \quad (*)
\end{align}
\end{solution}

\end{parts}


\end{questions}
\end{document}
