\documentclass[answers,a4paper,12pt]{exam}
\input{preamble.tex}

\title{{\bf{FYS101 Mekanikk}} \\ \Large{\answersareprinted Oblig XX}} 
\author{Institutt for Fysikk, REALTEK}
\date{Uke xx}

\begin{document}
\maketitle

\begin{questions}

\question Beregn det totale arealet hvert hjul er i kontakt med bakken:
\begin{parts}
\part Gitt data:
\begin{solution}
\centering
\begin{align}
m &= \SI{1500}{\kilo\gram} \\
P_y &= \SI{200}{\kilo\pascal}
\end{align}
\end{solution}

\part Beregn det totale arealet:
\begin{solution}
For å finne det totale arealet \( A \) som hvert hjul er i kontakt med bakken, må vi først beregne kraften som virker på hvert hjul. Denne kraften er lik vekten av bilen delt på antall hjul. Anta at bilen har fire hjul:

\centering
\begin{align}
F_{\text{total}} &= m \cdot g \\
                 &= \SI{1500}{\kilo\gram} \cdot \SI{9.81}{\meter\per\second\squared} \\
                 &= \SI{14715}{\newton}
\end{align}

Kraften per hjul blir da:
\begin{align}
F_{\text{per hjul}} &= \frac{F_{\text{total}}}{4} \\
                    &= \frac{\SI{14715}{\newton}}{4} \\
                    &= \SI{3678.75}{\newton}
\end{align}

Arealet \( A \) kan beregnes ved å bruke trykkformelen \( P = \frac{F}{A} \), hvor \( P \) er trykket og \( F \) er kraften:
\begin{align}
A &= \frac{F_{\text{per hjul}}}{P_y} \\
  &= \frac{\SI{3678.75}{\newton}}{\SI{200}{\kilo\pascal}} \\
  &= \frac{\SI{3678.75}{\newton}}{\SI{200000}{\pascal}} \\
  &= \doubleunderline{\SI{0.01839}{\meter\squared}}
\end{align}
\end{solution}
\end{parts}



\question Beregn trykket og tettheten for en blokk:
\begin{parts}

\part Beregn trykket på hvert hjul:
\begin{solution}
Kreftene på hvert hjul er gitt ved:
\begin{align*}
F_{\text{ghjul}} &= \frac{F_{\text{gbil}}}{\text{antall hjul}} \\
F_{\text{ghjul}} &= \frac{m \cdot g}{4} \tag{II}
\end{align*}

Trykket er gitt ved krefter per areal:
\begin{align*}
P &= \frac{F_{\text{ghjul}}}{A} \\
P &= \frac{m \cdot g}{4 \cdot A} \\
A &= \frac{m \cdot g}{4 \cdot P} \\
  &= \frac{1500 \cdot 9.8}{4 \cdot 200 \times 10^{3}} \\
  &= \doubleunderline{0.018393 \, \mathrm{m}^2} \\
A &= \doubleunderline{184 \, \mathrm{cm}^2}
\end{align*}

Det totale arealet til den delen av hjulene som er i kontakt med bakken er \SI{184}{\centi\meter\squared}.
\end{solution}

\part Beregn tettheten til blokken:
\begin{solution}
Vi begynner med å finne oppdriften. Blokken er i ro, og vi benytter derfor Newtons 1. lov:
\begin{align*}
\sum F &= 0 \\
\pi + B - F_{g} &= 0 \\
B &= F_{g} - T \\
  &= \SI{5.00}{\newton} - \SI{4.55}{\newton} \\
B &= \doubleunderline{\SI{0.45}{\newton}}
\end{align*}

Oppdriften er det samme som tyngden til den fortrengte væsken (Arkimedes prinsipp):
\begin{align*}
B &= \text{F}_{\text{fortrengt}} \\
  &= m_{\text{fortrengt}} \cdot g \quad \left(m_{\text{fortrengt}} = m_{v} = \rho_{v} \cdot V\right) \\
B &= \rho_{v} \cdot V \cdot g \\
V &= \frac{B}{\rho_{v} \cdot g} \tag{I}
\end{align*}

Tettheten til blokken er gitt ved:
\begin{align*}
\rho &= \frac{m}{V} \\
     &= \frac{\rho_{v} \cdot m}{B} \quad \left(m = \frac{F_{g}}{g}\right) \\
     &= \frac{\rho_{v} \cdot F_{g}}{B} \\
     &= \frac{1000 \cdot \SI{5.00}{\newton}}{\SI{0.45}{\newton}} \\
\rho &= \doubleunderline{\SI{11111.1}{\kilogram\per\meter\cubed}}
\end{align*}

Tettheten til materialet blokken består av er omtrent \SI{11111.1}{\kilogram\per\meter\cubed}, som er rundt samme tetthet som bly.
\end{solution}

\end{parts}



\question FILL QUESTION
\begin{parts}

\part Beregn hastigheten $v_1$ ved bruk av volumstrømning:
\begin{solution}
Volumstrømningen $I_v$ er gitt ved:
\begin{align}
I_v &= A_1 v_1 \\
v_1 &= \frac{4 I_v}{\pi d_1^2} \tag{I}
\end{align}
hvor $A_1 = \pi \left(\frac{d_1}{2}\right)^2$.
\end{solution}

\part Bruk Bernoullis likning for å finne hastigheten etter kompresjon, $v_2$:
\begin{solution}
Bernoullis likning gir:
\begin{align}
P_2 + \frac{1}{2} \rho v_2^2 &= P_1 + \frac{1}{2} \rho v_1^2 \\
\rho v_2^2 &= 2(P_1 - P_2) + \rho v_1^2 \\
v_2 &= \sqrt{\frac{2(P_1 - P_2)}{\rho} + v_1^2}
\end{align}
Ved å sette inn likning (I):
\begin{align}
v_2 &= \sqrt{2(P_1 - P_2) + \left(\frac{4 I_v}{\pi d_1^2}\right)^2}
\end{align}
\end{solution}

\part Beregn diameteren $d_2$ som tillater trykkendringen:
\begin{solution}
Ved å anta at væsken er inkompressibel, bruker vi kontinuitetslikningen:
\begin{align}
A_2 v_2 &= A_1 v_1 \\
\frac{\pi d_2^2}{4} v_2 &= I_v \\
d_2 &= \sqrt{\frac{4 I_v}{\pi v_2}}
\end{align}
Ved å sette inn verdiene:
\begin{align}
d_2 &= \sqrt{\frac{4 \cdot \left(2.80 \frac{\mathrm{L}}{5}\right) \cdot \frac{1030}{100}}{\pi \cdot 12.706 \frac{\mathrm{m}}{\mathrm{s}}}} \\
d_2 &= \doubleunderline{\SI{1.68}{\centi\meter}}
\end{align}
\end{solution}

\end{parts}




\question Beregn volumstrømmen når oljen først forlater tanken:
\begin{parts}

\part Gitt data:
\begin{solution}
\centering
\begin{align}
h &= \SI{250}{\centi\meter} = \SI{2.50}{\meter} \\
l &= \SI{5.00}{\centi\meter} = \SI{0.050}{\meter} \\
r &= \SI{0.75}{\centi\meter} = \SI{0.0075}{\meter} \\
\rho_{\text{olje}} &= \SI{860}{\kilo\gram\per\meter\cubed} \\
\eta_{\text{olje}} &= \SI{180}{\milli\pascal\second}
\end{align}
\end{solution}

\part Beregn trykkforskjellen $P$:
\begin{solution}
\centering
\begin{align}
P &= P_{0} + \rho g h \\
  &= P_{0} + \SI{860}{\kilo\gram\per\meter\cubed} \cdot \SI{9.81}{\meter\per\second\squared} \cdot \SI{2.50}{\meter} \\
  &= P_{0} + \SI{21015}{\pascal}
\end{align}
\end{solution}

\part Bruk Poiseuilles lov for å finne volumstrømmen $Q$:
\begin{solution}
\centering
\begin{align}
\Delta P &= P - P_{0} = \rho g h \\
Q &= \frac{\pi \Delta P r^4}{8 \eta l} \\
  &= \frac{\pi \cdot \SI{21015}{\pascal} \cdot \left(\SI{0.0075}{\meter}\right)^4}{8 \cdot \SI{180}{\milli\pascal\second} \cdot \SI{0.050}{\meter}} \\
  &= \frac{\pi \cdot \SI{21015}{\pascal} \cdot \SI{3.1640625e-10}{\meter^4}}{8 \cdot \SI{0.180}{\pascal\second} \cdot \SI{0.050}{\meter}} \\
  &= \SI{2.91}{\liter\per\second}
\end{align}
\end{solution}

\part Konklusjon:
\begin{solution}
Volumstrømmen når oljen først forlater tanken er \doubleunderline{\SI{2.91}{\liter\per\second}}.
\end{solution}

\end{parts}

\end{questions}
\end{document}
