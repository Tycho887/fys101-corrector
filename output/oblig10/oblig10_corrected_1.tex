\documentclass[answers,a4paper,12pt]{exam}
\input{preamble.tex}

\title{{\bf{FYS101 Mekanikk}} \\ \Large{\answersareprinted Oblig XX}} 
\author{Institutt for Fysikk, REALTEK}
\date{Uke xx}

\begin{document}
\maketitle

\begin{questions}

\question FILL QUESTION
\begin{parts}
\part Beregn det totale arealet hvert hjul er i kontakt med bakken:
\begin{solution}
For å finne det totale arealet som hvert hjul er i kontakt med bakken, må vi først forstå kreftene som virker på hjulene fra bilen og trykkfordelingen på hjulene. 

Gitt:
\begin{align}
m &= \SI{1500}{\kilo\gram} \\
P_y &= \SI{200}{\kilo\pascal}
\end{align}

Først, beregn den totale vekten av bilen:
\begin{align}
F_g &= m \cdot g \\
    &= \SI{1500}{\kilo\gram} \cdot \SI{9.81}{\meter\per\second\squared} \\
    &= \SI{14715}{\newton}
\end{align}

Anta at vekten er jevnt fordelt på de fire hjulene, så kraften på hvert hjul er:
\begin{align}
F_{\text{per hjul}} &= \frac{F_g}{4} \\
                    &= \frac{\SI{14715}{\newton}}{4} \\
                    &= \SI{3678.75}{\newton}
\end{align}

Trykket \( P_y \) er relatert til kraften og arealet ved formelen:
\begin{align}
P_y &= \frac{F_{\text{per hjul}}}{A}
\end{align}

Løs for \( A \):
\begin{align}
A &= \frac{F_{\text{per hjul}}}{P_y} \\
  &= \frac{\SI{3678.75}{\newton}}{\SI{200}{\kilo\pascal}} \\
  &= \frac{\SI{3678.75}{\newton}}{\SI{200000}{\pascal}} \\
  &= \doubleunderline{\SI{0.01839}{\meter\squared}}
\end{align}
\end{solution}
\end{parts}


\question FILL QUESTION
\begin{parts}

\part Kreftene på hvert hjul er gitt ved:
\begin{solution}
\begin{align}
F_{\text{ghjul}} &= \frac{F_{\text{gbil}}}{\text{antall hjul}} \\
F_{\text{ghjul}} &= \frac{m \cdot g}{4} \tag{II}
\end{align}
\end{solution}

\part Bruker trykk er gitt ved krefter per areal og løser for $A$:
\begin{solution}
\begin{align}
P &= \frac{F_{\text{ghjul}}}{A} \\
P &= \frac{m \cdot g}{4 \cdot A} \\
A &= \frac{m \cdot g}{4 \cdot P} \\
  &= \frac{1500 \cdot \SI{9.8}{\newton\per\kilo\gram}}{4 \cdot \SI{200}{\kilo\pascal}} \\
  &= \doubleunderline{\SI{0.018393}{\meter\squared}} \\
A &= \doubleunderline{\SI{184}{\centi\meter\squared}}
\end{align}
Det totale arealet til den delen av hjulene som er i kontakt med bakken er \SI{184}{\centi\meter\squared}.
\end{solution}

\part Skal finne blokkens tetthet, og begynner med å finne oppdriften. Blokken er i ro, og benytter derfor Newtons 1. lov:
\begin{solution}
\begin{align}
\sum F &= 0 \\
\pi + B - F_{g} &= 0 \\
B &= F_{g} - T \\
  &= \SI{5.00}{\newton} - \SI{4.55}{\newton} \\
B &= \doubleunderline{\SI{0.45}{\newton}}
\end{align}
Oppdriften er det samme som tyngden til den fortrengte væsken (Arkimedes prinsipp).
\end{solution}

\part Tettheten til blokken er gitt ved $\rho = \frac{m}{V}$, og setter inn likning:
\begin{solution}
\begin{align}
B &= \rho_{v} \cdot V \cdot g \\
V &= \frac{B}{\rho_{v} \cdot g} \tag{I} \\
\rho &= \frac{m}{V} \\
    &= \frac{\rho_{v} \cdot m}{B} \\
    &= \frac{\rho_{v} \cdot F_{g}}{B} \\
    &= \frac{1000 \cdot \SI{5.00}{\newton}}{\SI{0.45}{\newton}} \\
\rho &= \doubleunderline{\SI{11111.1}{\kilo\gram\per\meter\cubed}}
\end{align}
Tettheten til materialet blokken består av er omtrent \SI{11111.1}{\kilo\gram\per\meter\cubed}, som er rundt samme tetthet som bly.
\end{solution}

\end{parts}

\question FILL QUESTION
\begin{parts}

\part Beregn hastigheten $v_1$ ved hjelp av volumstrømmen:
\begin{solution}
Volumstrømmen $I_v$ kan uttrykkes som:
\begin{align}
I_v &= A_1 v_1 \\
v_1 &= \frac{4 I_v}{\pi d_1^2} \tag{I}
\end{align}
hvor $A_1 = \pi \left(\frac{d_1}{2}\right)^2$ er tverrsnittsarealet.
\end{solution}

\part Bruk Bernoullis likning for å finne hastigheten etter kompresjonen, $v_2$:
\begin{solution}
Bernoullis likning gir oss:
\begin{align}
P_2 + \frac{1}{2} \rho v_2^2 &= P_1 + \frac{1}{2} \rho v_1^2 \\
\rho v_2^2 &= 2(P_1 - P_2) + \rho v_1^2 \\
v_2 &= \sqrt{\frac{2(P_1 - P_2)}{\rho} + v_1^2}
\end{align}
Ved å sette inn likning (I) får vi:
\begin{align}
v_2 &= \sqrt{2(P_1 - P_2) + \left(\frac{4 I_v}{\pi d_1^2}\right)^2}
\end{align}
Etter beregning finner vi:
\begin{align}
v_2 &= \doubleunderline{\SI{12.706}{\meter\per\second}}
\end{align}
\end{solution}

\part Bruk kontinuitetslikningen for å finne diameteren $d_2$:
\begin{solution}
For en inkompressibel væske gjelder kontinuitetslikningen:
\begin{align}
A_2 v_2 &= A_1 v_1 \\
\pi \left(\frac{d_2}{2}\right)^2 v_2 &= I_v \\
d_2 &= \sqrt{\frac{4 I_v}{\pi v_2}}
\end{align}
Ved å sette inn verdiene får vi:
\begin{align}
d_2 &= \sqrt{\frac{4 \cdot 2.80 \cdot \frac{1}{5}}{\pi \cdot 12.706}} \\
d_2 &= \doubleunderline{\SI{1.68}{\centi\meter}}
\end{align}
\end{solution}

\end{parts}

\question FILL QUESTION
\begin{parts}

\part Beregn trykket $P$ ved hjelp av trykkforskjellen:
\begin{solution}
Trykket $P$ kan beregnes ved hjelp av formelen for trykkforskjell:
\begin{align}
P &= P_{0} + \rho g h
\end{align}
Her er $\rho$ tettheten til oljen, $g$ er tyngdeakselerasjonen, og $h$ er høyden.
\end{solution}

\part Bruk Poiseuille's lov for å finne volumstrømmen:
\begin{solution}
Poiseuille's lov gir oss volumstrømmen $Iv$ som:
\begin{align}
\Delta P &= \frac{8 \eta l}{\pi r^{4}} Iv \quad \left(\Delta P = P - P_{0}\right) \\
Iv &= \frac{\pi \left(P - P_{0}\right) r^{4}}{8 \eta l}
\end{align}
Ved å sette inn uttrykket for $\Delta P$ fra trykkforskjellen, får vi:
\begin{align}
Iv &= \frac{\pi \rho g h r^{4}}{8 \eta l} \\
   &= \frac{\pi \cdot \SI{860}{\kilogram\per\cubic\meter} \cdot \SI{9.81}{\meter\per\second\squared} \cdot \SI{2.50}{\meter} \cdot \left(\SI{0.0075}{\meter}\right)^{4}}{8 \cdot \SI{0.180}{\pascal\second} \cdot \SI{0.05}{\meter}} \\
   &= \doubleunderline{\SI{2.91}{\liter\per\second}}
\end{align}
Volumstrømmen når oljen først forlater tanken er \SI{2.91}{\liter\per\second}.
\end{solution}

\end{parts}
\end{questions}
\end{document}
