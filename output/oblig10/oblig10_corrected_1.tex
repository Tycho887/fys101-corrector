\documentclass[answers,a4paper,12pt]{exam}
\input{preamble.tex}

\title{{\bf{FYS101 Mekanikk}} \\ \Large{\answersareprinted Oblig XX}} 
\author{Institutt for Fysikk, REALTEK}
\date{Uke xx}

\begin{document}
\maketitle

\begin{questions}

\question Beregn det totale arealet hvert hjul er i kontakt med bakken:
\begin{parts}
\part Gitt data:
\begin{solution}
\begin{align}
m & = \SI{1500}{\kilo\gram} \\
P_y & = \SI{200}{\kilo\pascal}
\end{align}
\end{solution}

\part Forklar hvordan man finner det totale arealet:
\begin{solution}
For å finne det totale arealet som hvert hjul er i kontakt med bakken, må vi vurdere kreftene som hjulene opplever fra bilen og trykkfordelingen på hjulene. Trykket \( P \) er definert som kraft per areal, så vi kan bruke formelen:
\[
P = \frac{F}{A}
\]
hvor \( F \) er kraften som virker på hjulene, og \( A \) er arealet. Kraften \( F \) kan beregnes ved å bruke tyngdekraften som virker på bilen:
\[
F = m \cdot g
\]
hvor \( g \) er tyngdeakselerasjonen, \SI{9.81}{\meter\per\second\squared}.
\end{solution}

\part Beregn det totale arealet:
\begin{solution}
\centering
\begin{align}
F & = m \cdot g \\
  & = \SI{1500}{\kilo\gram} \cdot \SI{9.81}{\meter\per\second\squared} \\
  & = \SI{14715}{\newton}
\end{align}
Deretter bruker vi trykkformelen for å finne arealet:
\begin{align}
P_y & = \frac{F}{A} \\
A & = \frac{F}{P_y} \\
  & = \frac{\SI{14715}{\newton}}{\SI{200}{\kilo\pascal}} \\
  & = \frac{\SI{14715}{\newton}}{\SI{200000}{\pascal}} \\
  & = \doubleunderline{\SI{0.073575}{\meter\squared}}
\end{align}
\end{solution}
\end{parts}



\question Beregn trykket og tettheten for en blokk:
\begin{parts}

\part Beregn trykket på hvert hjul:
\begin{solution}
Kreftene på hvert hjul er gitt ved:
\begin{align*}
F_{\text{ghjul}} &= \frac{F_{\text{gbil}}}{\text{antall hjul}} \\
F_{\text{ghjul}} &= \frac{m \cdot g}{4} \tag{II}
\end{align*}

Bruker trykk som er gitt ved krefter per areal og løser for $A$:
\begin{align*}
P &= \frac{F_{\text{ghjul}}}{A} \\
P &= \frac{m \cdot g}{4 \cdot A} \\
A &= \frac{m \cdot g}{4 \cdot P} \\
  &= \frac{1500 \, \mathrm{kg} \cdot 9.8 \, \mathrm{m/s^2}}{4 \cdot 200 \cdot 10^{3} \, \mathrm{Pa}} \\
  &= \SI{0.018393}{\meter\squared} \\
A &= \doubleunderline{\SI{184}{\centi\meter\squared}}
\end{align*}
Det totale arealet til den delen av hjulene som er i kontakt med bakken er \SI{184}{\centi\meter\squared}.
\end{solution}

\part Beregn tettheten til blokken:
\begin{solution}
\begin{align*}
F_g &= \SI{5.00}{\newton} \\
T &= \SI{4.55}{\newton}
\end{align*}

Skal finne blokkens tetthet, og begynner med å finne oppdriften. Blokken er i ro, og benytter derfor Newtons 1. lov:
\begin{align*}
\sum F &= 0 \\
B - F_g + T &= 0 \\
B &= F_g - T \\
  &= \SI{5.00}{\newton} - \SI{4.55}{\newton} \\
B &= \SI{0.45}{\newton}
\end{align*}

Oppdriften er det samme som tyngden til den fortrengte væsken (Arkimedes prinsipp):
\begin{align*}
B &= \rho_v \cdot V \cdot g \\
V &= \frac{B}{\rho_v \cdot g} \tag{I}
\end{align*}

Tettheten til blokken er gitt ved $\rho = \frac{m}{V}$, og setter inn i likning:
\begin{align*}
\rho &= \frac{m}{V} \\
     &= \frac{\rho_v \cdot F_g}{B \cdot g} \\
     &= \frac{1000 \, \mathrm{kg/m^3} \cdot \SI{5.00}{\newton}}{\SI{0.45}{\newton}} \\
     &= \doubleunderline{\SI{11111}{\kilogram\per\meter\cubed}}
\end{align*}
Tettheten til materialet blokken består av er omtrent \SI{11111}{\kilogram\per\meter\cubed}, som er rundt samme tetthet som bly.
\end{solution}

\end{parts}



\question FILL QUESTION
\begin{parts}

\part Beregn diameteren $d_2$ som tillater trykkendringen:
\begin{solution}
Vi begynner med å uttrykke volumstrømmen $I_v$ ved hjelp av tverrsnittsarealet $A_1$ og hastigheten $v_1$:
\begin{align}
I_v &= A_1 v_1 \\
v_1 &= \frac{4 I_v}{\pi d_1^2} \tag{I}
\end{align}
hvor $A_1 = \pi \left(\frac{d_1}{2}\right)^2$.

Deretter bruker vi Bernoullis likning for å finne et uttrykk for hastigheten etter kompresjonen, $v_2$:
\begin{align}
P_2 + \frac{1}{2} \rho v_2^2 &= P_1 + \frac{1}{2} \rho v_1^2 \\
\rho v_2^2 &= 2(P_1 - P_2) + \rho v_1^2 \\
v_2 &= \sqrt{\frac{2(P_1 - P_2)}{\rho} + v_1^2}
\end{align}
Ved å sette inn likning (I) får vi:
\begin{align}
v_2 &= \sqrt{2\left(\frac{P_1 - P_2}{\rho}\right) + \left(\frac{4 I_v}{\pi d_1^2}\right)^2}
\end{align}
\begin{align}
v_2 &= \SI{12.706}{\meter\per\second}
\end{align}

Vi antar væsken som inkompressibel og bruker kontinuitetslikningen for å finne et uttrykk for $d_2$:
\begin{align}
A_2 v_2 &= A_1 v_1 \\
\pi \left(\frac{d_2}{2}\right)^2 v_2 &= I_v \\
d_2 &= \sqrt{\frac{4 I_v}{\pi v_2}}
\end{align}
Ved å sette inn de gitte verdiene:
\begin{align}
d_2 &= \sqrt{\frac{4 \cdot \SI{2.80}{\liter\per\second} \cdot \frac{1}{5}}{\pi \cdot \SI{12.706}{\meter\per\second}}} \\
d_2 &= \doubleunderline{\SI{1.68}{\centi\meter}}
\end{align}
Dette er diameteren som tillater trykkendringen.
\end{solution}

\end{parts}



\question FILL QUESTION
\begin{parts}
\part Beregn trykket $P$ ved hjelp av trykkforskjellen:
\begin{solution}
Trykket $P$ er gitt ved formelen:
\begin{align}
P &= P_{0} + \rho g h
\end{align}
hvor:
\begin{itemize}
  \item $P_{0}$ er det atmosfæriske trykket,
  \item $\rho$ er tettheten til oljen, \SI{860}{\kilogram\per\meter\cubed},
  \item $g$ er tyngdeakselerasjonen, \SI{9.81}{\meter\per\second\squared},
  \item $h$ er høyden, \SI{250}{\centi\meter} = \SI{2.50}{\meter}.
\end{itemize}
\end{solution}

\part Bruk Poiseuilles lov for å finne volumstrømmen:
\begin{solution}
Poiseuilles lov gir oss volumstrømmen $I_v$ som:
\begin{align}
\Delta P &= \frac{8 \eta l}{\pi r^4} I_v \quad \left(\Delta P = P - P_{0}\right) \\
I_v &= \frac{\pi (P - P_{0}) r^4}{8 \eta l}
\end{align}
Ved å sette inn for $\Delta P = \rho g h$, får vi:
\begin{align}
I_v &= \frac{\pi \rho g h r^4}{8 \eta l}
\end{align}
Ved å sette inn de gitte verdiene:
\begin{align}
I_v &= \frac{\pi \cdot \SI{860}{\kilogram\per\meter\cubed} \cdot \SI{9.81}{\meter\per\second\squared} \cdot \SI{2.50}{\meter} \cdot \left(\SI{0.75}{\centi\meter}\right)^4}{8 \cdot \SI{180}{\milli\pascal\second} \cdot \SI{5.00}{\centi\meter}} \\
&= \frac{\pi \cdot 860 \cdot 9.81 \cdot 2.50 \cdot (0.0075)^4}{8 \cdot 0.180 \cdot 0.05} \\
&= \doubleunderline{\SI{2.91}{\liter\per\second}}
\end{align}
Volumstrømmen når oljen først forlater tanken er \SI{2.91}{\liter\per\second}.
\end{solution}
\end{parts}

\end{questions}
\end{document}
