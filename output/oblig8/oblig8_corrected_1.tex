\documentclass[answers,a4paper,12pt]{exam}
\input{preamble.tex}

\title{{\bf{FYS101 Mekanikk}} \\ \Large{\answersareprinted Oblig XX}} 
\author{Institutt for Fysikk, REALTEK}
\date{Uke xx}

\begin{document}
\maketitle

\begin{questions}
\question FILL QUESTION
\begin{parts}

\part Bruk Newtons andre lov for rotasjon:
\begin{solution}
Vi starter med å bruke Newtons andre lov for rotasjon, som sier at summen av dreiemomentene (\(\Sigma \tau\)) er lik treghetsmomentet (\(I\)) multiplisert med vinkelakselerasjonen (\(\alpha\)):

\[
\Sigma \tau = I \alpha
\]

For en sirkelbevegelse, er lineær akselerasjon (\(a\)) relatert til vinkelakselerasjon ved \(a = \alpha R\). Dermed kan vi skrive:

\[
T R = \frac{2}{5} M R^2 \cdot \frac{a}{R}
\]

Dette forenkler til:

\[
T = \frac{2}{5} M a
\]

Dette er likning (I).
\end{solution}

\part Bruk Newtons andre lov for klossen i \(y\)-retning:
\begin{solution}
For klossen som beveger seg i \(y\)-retning, bruker vi Newtons andre lov:

\[
\sum F_y = m a_y
\]

Siden all bevegelse skjer i \(y\)-retning, er \(a_y = a\). Vi har:

\[
E_g - T = m a
\]

Dette er likning (II).
\end{solution}

\part Sett inn likning (I) i likning (II) og løs for akselerasjon:
\begin{solution}
Vi setter inn likning (I) i likning (II):

\[
m g - \frac{2}{5} M a = m a
\]

For å løse for \(a\), omorganiserer vi:

\[
a \left(m + \frac{2}{5} M\right) = m g
\]

Dermed er akselerasjonen:

\[
a = \frac{m}{m + \frac{2}{5} M} g
\]

Dette er likning (III).
\end{solution}

\part Sett inn likning (III) i likning (I) for å finne spenningen \(T\):
\begin{solution}
Vi setter inn uttrykket for \(a\) fra likning (III) i likning (I):

\[
T = \frac{2}{5} M \frac{m}{m + \frac{2}{5} M} g
\]

For å forenkle uttrykket, multipliserer vi:

\[
T = \frac{2 m M}{5 m + 2 M} g
\]

Dette gir oss uttrykket for spenningen \(T\).
\end{solution}

\end{parts}


\question FILL QUESTION
\begin{parts}

\part Finner spinnet når hjulet roterer:
\begin{solution}
\centering
\begin{align}
L & = m_h \cdot r \cdot \omega \cdot r^2 \\
  & = \frac{F_{gh}}{g} \cdot \omega \cdot r^2 \quad \left(F_{gh} = m_h \cdot g \rightarrow m_h = \frac{F_{gh}}{g}\right) \\
  & = \frac{30 \, \mathrm{N}}{9.81 \, \mathrm{m/s^2}} \cdot 12 \, \mathrm{rad/s} \cdot \left(28 \times 10^{-2} \, \mathrm{m}\right)^2 \\
  & = \doubleunderline{18 \, \mathrm{kg \cdot m^2/s}}
\end{align}
\end{solution}

\part Utrykket for presesjonsvinkelhastighet er gitt ved:
\begin{solution}
\centering
\begin{align}
\omega_p & = \frac{M_g \cdot D}{2 \cdot I \cdot \omega} \\
         & = \frac{F_{gh} \cdot D}{2 \cdot L} \\
         & = \frac{30 \, \mathrm{N} \cdot 50 \times 10^{-2} \, \mathrm{m}}{2 \cdot 18 \, \mathrm{kg \cdot m^2/s}} \\
         & = \doubleunderline{0.42 \, \mathrm{rad/s}}
\end{align}
\end{solution}

\part Finner tiden $T$ ved:
\begin{solution}
\centering
\begin{align}
T & = \frac{1}{f} \quad \left(f = \frac{\omega}{2\pi}\right) \\
  & = \frac{2\pi}{\omega} \\
  & = \frac{2\pi}{0.42 \, \mathrm{rad/s}} \\
  & = \doubleunderline{15 \, \mathrm{s}}
\end{align}
\end{solution}

\part Spinnet assosiert med bevegelse til massesenteret blir gitt ved:
\begin{solution}
\centering
\begin{align}
L_p & = I_{cm} \cdot \omega_p \quad \left(I_{cm} = m \left(\frac{D}{2}\right)^2\right) \\
    & = m \left(\frac{D}{2}\right)^2 \cdot \omega_p \\
    & = \frac{F_{gh} \cdot D^2 \cdot \omega_p}{4 \cdot g} \\
    & = \frac{30 \, \mathrm{N} \cdot (0.50 \, \mathrm{m})^2 \cdot 0.42 \, \mathrm{rad/s}}{4 \cdot 9.81 \, \mathrm{m/s^2}} \\
    & = \doubleunderline{0.080 \, \mathrm{kg \cdot m^2/s}}
\end{align}
\end{solution}

\end{parts}


\section*{Del B}

\question FILL QUESTION

\begin{parts}

\part Forklar hva kinetisk energi er:
\begin{solution}
Kinetisk energi er energien et objekt har på grunn av sin bevegelse. Den er gitt ved formelen:
\[
E_k = \frac{1}{2} m v^2
\]
hvor \(m\) er massen til objektet og \(v\) er hastigheten.
\end{solution}

\part Beregn den kinetiske energien for et objekt med masse \SI{10}{\kilo\gram} og hastighet \SI{5}{\meter\per\second}:
\begin{solution}
\centering
\begin{align}
E_k &= \frac{1}{2} \cdot \SI{10}{\kilo\gram} \cdot \left(\SI{5}{\meter\per\second}\right)^2 \\
    &= \frac{1}{2} \cdot \SI{10}{\kilo\gram} \cdot \SI{25}{\meter\squared\per\second\squared} \\
    &= \SI{125}{\joule} \\
    &= \doubleunderline{\SI{125}{\joule}}
\end{align}
\end{solution}

\part Forklar Python-funksjonen gitt i oppgaven:
\begin{solution}
Funksjonen \texttt{linear\_kinetic\_energy(m, v)} beregner den kinetiske energien til et objekt med masse \(m\) og hastighet \(v\). Den returnerer verdien \(\frac{1}{2} \times m \times v^2\), som er den matematiske formelen for kinetisk energi.
\end{solution}

\end{parts}

\question Beregn treghetsmomentet for en solid sylinder:
\begin{parts}

\part Del A: Definer treghetsmoment for en solid sylinder:
\begin{solution}
Treghetsmomentet for en solid sylinder er et mål på hvor mye motstand en sylinder har mot rotasjon rundt sin akse. For en solid sylinder med masse \(M\) og radius \(R\), er treghetsmomentet gitt ved formelen:
\[
I = \frac{1}{2} M R^2
\]
\end{solution}

\part Del B: Beregn treghetsmomentet for en sylinder med masse \SI{10}{\kilo\gram} og radius \SI{0.5}{\meter}:
\begin{solution}
\centering
\begin{align}
I & = \frac{1}{2} M R^2 \\
  & = \frac{1}{2} \times \SI{10}{\kilo\gram} \times \left(\SI{0.5}{\meter}\right)^2 \\
  & = \frac{1}{2} \times \SI{10}{\kilo\gram} \times \SI{0.25}{\meter\squared} \\
  & = \SI{1.25}{\kilo\gram\meter\squared} \\
  & = \doubleunderline{\SI{1.25}{\kilo\gram\meter\squared}}
\end{align}
\end{solution}

\end{parts}

\question FILL QUESTION
\begin{parts}

\part Forklar hva angulær kinetisk energi er:
\begin{solution}
Angulær kinetisk energi er energien et roterende objekt har på grunn av sin rotasjon. Den er avhengig av objektets treghetsmoment og vinkelhastighet. Formelen for angulær kinetisk energi er gitt ved:
\[
E_k = \frac{1}{2} I \omega^2
\]
hvor \( I \) er treghetsmomentet og \( \omega \) er vinkelhastigheten.
\end{solution}

\part Korriger koden for å beregne angulær kinetisk energi:
\begin{solution}
Den korrigerte Python-koden for å beregne angulær kinetisk energi er som følger:
\begin{verbatim}
def angular_kinetic_energy(I, omega):
    return 0.5 * I * omega**2
\end{verbatim}
Denne funksjonen tar inn treghetsmomentet \( I \) og vinkelhastigheten \( \omega \), og returnerer den angulære kinetiske energien.
\end{solution}

\part Beregn den angulære kinetiske energien for et objekt med \( I = \SI{10}{\kilogram\meter\squared} \) og \( \omega = \SI{5}{\radian\per\second} \):
\begin{solution}
\centering
\begin{align}
E_k &= \frac{1}{2} \cdot I \cdot \omega^2 \\
    &= \frac{1}{2} \cdot \SI{10}{\kilogram\meter\squared} \cdot \left(\SI{5}{\radian\per\second}\right)^2 \\
    &= \frac{1}{2} \cdot \SI{10}{\kilogram\meter\squared} \cdot \SI{25}{\radian\squared\per\second\squared} \\
    &= \SI{125}{\joule} \\
    &= \doubleunderline{\SI{125}{\joule}}
\end{align}
\end{solution}

\end{parts}

\question FILL QUESTION
\begin{parts}

\part Beregn massen $M$ av svinghjulet som kreves for å stoppe bilen.
\begin{solution}
Vi har følgende data for bilen:
\begin{align}
m &= \SI{1200}{\kilo\gram} \\
v &= \SI{120}{\kilo\meter\per\hour} = \SI{120}{\kilo\meter\per\hour} \cdot \frac{\SI{1000}{\meter}}{\SI{1}{\kilo\meter}} \cdot \frac{\SI{1}{\hour}}{\SI{3600}{\second}} = \SI{33.33}{\meter\per\second}
\end{align}

For svinghjulet:
\begin{align}
R &= \SI{0.12}{\meter} \\
\omega &= \SI{60000}{\revolutions\per\minute} = \SI{60000}{\revolutions\per\minute} \cdot \frac{2\pi \, \mathrm{rad}}{\SI{1}{\revolution}} \cdot \frac{\SI{1}{\minute}}{\SI{60}{\second}} = \SI{6283.19}{\radian\per\second}
\end{align}

Kinetisk energi for bilen er lik kinetisk rotasjonsenergi for svinghjulet:
\begin{align}
\frac{1}{2} m v^{2} &= \frac{1}{2} I \omega^{2}
\end{align}

Med treghetsmomentet $I = \frac{1}{2} M R^{2}$, setter vi inn og løser for $M$:
\begin{align}
\frac{1}{2} m v^{2} &= \frac{1}{2} \cdot \frac{1}{2} M R^{2} \omega^{2} \\
M &= \frac{2 m v^{2}}{R^{2} \omega^{2}} \\
  &= \frac{2 \cdot \SI{1200}{\kilo\gram} \cdot \left(\SI{33.33}{\meter\per\second}\right)^{2}}{\left(\SI{0.12}{\meter}\right)^{2} \cdot \left(\SI{6283.19}{\radian\per\second}\right)^{2}} \\
  &= \doubleunderline{\SI{4.7}{\kilo\gram}}
\end{align}
\end{solution}

\end{parts}
\end{questions}
\end{document}
