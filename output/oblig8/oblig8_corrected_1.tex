\documentclass[answers,a4paper,12pt]{exam}
\input{preamble.tex}

\title{{\bf{FYS101 Mekanikk}} \\ \Large{\answersareprinted Oblig XX}} 
\author{Institutt for Fysikk, REALTEK}
\date{Uke xx}

\begin{document}
\maketitle

\begin{questions}

\question FILL QUESTION
\begin{parts}

\part Bruk Newtons andre lov for rotasjon:
\begin{solution}
Vi starter med uttrykket for dreiemomentet:
\begin{align}
\Sigma \tau &= I \alpha \quad \text{hvor} \quad a = \alpha R \\
T R &= \frac{2}{5} M R^{2} \cdot \frac{a}{R} \\
T &= \frac{2}{5} M a \tag{I}
\end{align}
\end{solution}

\part Bruk Newtons andre lov for klossen i $y$-retning:
\begin{solution}
Vi ser på kreftene som virker på klossen:
\begin{align}
\sum F_y &= m a_y \quad \text{(hvor $a_y = a$, all bevegelse skjer i $y$-retning)} \\
E_g - T &= m a \tag{II}
\end{align}
Setter likning (I) inn i likning (II):
\begin{align}
m g - \frac{2}{5} M a &= m a \\
a \left(m + \frac{2}{5} M\right) &= m g \\
a &= \frac{m}{m + \frac{2}{5} M} g \tag{III}
\end{align}
\end{solution}

\part Beregn spenningen $T$ ved å sette likning (III) inn i likning (I):
\begin{solution}
\begin{align}
T &= \frac{2}{5} M \frac{m}{m + \frac{2}{5} M} g \\
T &= \frac{2 m M}{5 m + 2 M} g \\
T &= \frac{2 g}{\frac{5}{M} + \frac{2}{m}}
\end{align}
\end{solution}

\end{parts}



\question FILL QUESTION
\begin{parts}

\part Finner spinnet når hjulet roterer:
\begin{solution}
For å finne spinnet \(L\) når hjulet roterer, bruker vi formelen:
\begin{align}
L &= m_h r \omega r \\
  &= \frac{F_{gh} \omega r^2}{g} \quad \left(F_{gh} = m_h g \rightarrow m_h = \frac{F_{gh}}{g}\right) \\
  &= \frac{30 \mathrm{~N}}{9.81 \frac{\mathrm{m}}{\mathrm{s}^2}} \cdot 12 \frac{\mathrm{rad}}{\mathrm{s}} \cdot \left(28 \cdot 10^{-2} \mathrm{~m}\right)^2 \\
  &= \doubleunderline{18 \mathrm{~kg} \cdot \mathrm{m}^2/\mathrm{s}}
\end{align}
Spinnet er \(18 \mathrm{~kg} \cdot \mathrm{m}^2/\mathrm{s}\) som et resultat av at hjulet roteres.
\end{solution}

\part Utrykket for presesjonsvinkelhastighet:
\begin{solution}
Presesjonsvinkelhastigheten \(\omega_p\) er gitt ved:
\begin{align}
\omega_p &= \frac{M_g D}{2 I_w} \\
         &= \frac{F_{gh} D}{2 L} \\
         &= \frac{30 \mathrm{~N} \cdot 50 \cdot 10^{-2} \mathrm{~m}}{2 \cdot 18 \mathrm{~kg} \cdot \mathrm{m}^2/\mathrm{s}} \\
         &= \doubleunderline{0.42 \mathrm{~rad/s}}
\end{align}
\end{solution}

\part Finner tiden \(T\):
\begin{solution}
Tiden \(T\) er gitt ved:
\begin{align}
T &= \frac{1}{f} \quad \left(f = \frac{\omega}{2 \pi}\right) \\
  &= \frac{2 \pi}{\omega} \\
  &= \frac{2 \pi}{0.42 \mathrm{~rad/s}} \\
  &= \doubleunderline{15 \mathrm{~s}}
\end{align}
\end{solution}

\part Spinnet assosiert med bevegelse til massesenteret:
\begin{solution}
Spinnet \(L_p\) assosiert med bevegelse til massesenteret er gitt ved:
\begin{align}
L_p &= I_{cm} \omega_p \quad \left(I_{cm} = m \left(\frac{D}{2}\right)^2\right) \\
    &= \frac{F_{gh} D^2 \omega_p}{4 g} \\
    &= \frac{30 \mathrm{~N} \cdot (0.50 \mathrm{~m})^2 \cdot 0.42 \mathrm{~rad/s}}{4 \cdot 9.81 \mathrm{~m/s}^2} \\
    &= \doubleunderline{0.080 \mathrm{~kg} \cdot \mathrm{m}^2/\mathrm{s}}
\end{align}
Spinnet grunnet presesjon er \(0.080 \mathrm{~kg} \cdot \mathrm{m}^2/\mathrm{s}\).
\end{solution}

\end{parts}


\question FILL QUESTION
\begin{parts}
\part Forklar hva lineær kinetisk energi er:
\begin{solution}
Lineær kinetisk energi er energien et objekt har på grunn av sin bevegelse langs en rett linje. Den beregnes ved hjelp av formelen:
\[
KE = \frac{1}{2} m v^2
\]
hvor \( m \) er massen til objektet og \( v \) er hastigheten.
\end{solution}

\part Beregn den lineære kinetiske energien for et objekt med masse \SI{5}{\kilo\gram} og hastighet \SI{10}{\meter\per\second}:
\begin{solution}
\centering
\begin{align}
KE & = \frac{1}{2} \times \SI{5}{\kilo\gram} \times \left(\SI{10}{\meter\per\second}\right)^2 \\
   & = \frac{1}{2} \times \SI{5}{\kilo\gram} \times \SI{100}{\meter\squared\per\second\squared} \\
   & = \SI{2.5}{\kilo\gram} \times \SI{100}{\meter\squared\per\second\squared} \\
   & = \doubleunderline{\SI{250}{\joule}}
\end{align}
\end{solution}
\end{parts}

\question Beregn treghetsmomentet for en solid sylinder:
\begin{parts}
\part Forklar hva treghetsmomentet er:
\begin{solution}
Treghetsmomentet, også kjent som rotasjonsinerti, er et mål på hvor mye motstand et objekt har mot endringer i sin rotasjonsbevegelse. For en solid sylinder, er treghetsmomentet gitt ved formelen:
\[
I = \frac{1}{2} M R^2
\]
hvor \(M\) er massen til sylinderen og \(R\) er radiusen.
\end{solution}

\part Beregn treghetsmomentet for en sylinder med masse \SI{10}{\kilo\gram} og radius \SI{0.5}{\meter}:
\begin{solution}
\centering
\begin{align}
I & = \frac{1}{2} \cdot M \cdot R^2 \\
  & = \frac{1}{2} \cdot \SI{10}{\kilo\gram} \cdot \left(\SI{0.5}{\meter}\right)^2 \\
  & = \frac{1}{2} \cdot \SI{10}{\kilo\gram} \cdot \SI{0.25}{\meter\squared} \\
  & = \SI{1.25}{\kilo\gram\meter\squared} \\
  & = \doubleunderline{\SI{1.25}{\kilo\gram\meter\squared}}
\end{align}
\end{solution}
\end{parts}

\question FILL QUESTION
\begin{parts}

\part Define angular kinetic energy:
\begin{solution}
Angular kinetic energy is the energy possessed by a rotating object due to its motion. It is given by the formula:
\[
E_k = \frac{1}{2} I \omega^2
\]
where \(I\) is the moment of inertia and \(\omega\) is the angular velocity.
\end{solution}

\part Correct the given Python function for calculating angular kinetic energy:
\begin{solution}
The provided Python function calculates the angular kinetic energy using the formula:
\begin{verbatim}
def angular_kinetic_energy(I, omega):
    return 0.5 * I * omega**2
\end{verbatim}
This function correctly implements the formula for angular kinetic energy, where \(I\) is the moment of inertia and \(\omega\) is the angular velocity.
\end{solution}

\part Calculate the angular kinetic energy for a given moment of inertia and angular velocity:
\begin{solution}
Given:
\begin{align*}
I & = \SI{10}{\kilogram\meter\squared} \\
\omega & = \SI{5}{\radian\per\second}
\end{align*}

The angular kinetic energy is calculated as follows:
\begin{align}
E_k &= \frac{1}{2} I \omega^2 \\
    &= \frac{1}{2} \times \SI{10}{\kilogram\meter\squared} \times \left(\SI{5}{\radian\per\second}\right)^2 \\
    &= \frac{1}{2} \times \SI{10}{\kilogram\meter\squared} \times \SI{25}{\radian\squared\per\second\squared} \\
    &= \SI{125}{\joule}
\end{align}
Therefore, the angular kinetic energy is \doubleunderline{\SI{125}{\joule}}.
\end{solution}

\end{parts}


\question Beregn massen til svinghjulet som kreves for å stoppe bilen ved å overføre all kinetisk energi fra bilen til svinghjulet:
\begin{parts}

\part Definer de nødvendige formlene:
\begin{solution}
For å beregne massen til svinghjulet, må vi bruke formelen for kinetisk energi og moment of inertia. Den kinetiske energien til bilen er gitt ved:
\begin{align}
K_{\text{bil}} &= \frac{1}{2} m v^2
\end{align}
Den kinetiske rotasjonsenergien til svinghjulet er:
\begin{align}
K_{\text{svinghjul}} &= \frac{1}{2} I \omega^2
\end{align}
Moment of inertia for et svinghjul er:
\begin{align}
I &= \frac{1}{2} M R^2
\end{align}
\end{solution}

\part Beregn massen til svinghjulet:
\begin{solution}
Vi setter opp likningen for energibevaring:
\begin{align}
K_{\text{bil}} &= K_{\text{svinghjul}} \\
\frac{1}{2} m v^2 &= \frac{1}{2} \left(\frac{1}{2} M R^2\right) \omega^2
\end{align}
Løs for \( M \):
\begin{align}
M &= \frac{2 m v^2}{R^2 \omega^2}
\end{align}
Sett inn verdiene:
\begin{align}
M &= \frac{2 \cdot \SI{1200}{\kilogram} \cdot \left(\SI{120}{\kilo\meter\per\hour} \cdot \frac{\SI{1000}{\meter}}{\SI{1}{\kilo\meter}} \cdot \frac{\SI{1}{\hour}}{\SI{3600}{\second}}\right)^2}{\left(\SI{12e-2}{\meter}\right)^2 \cdot \left(\SI{60000}{\per\minute} \cdot \frac{2\pi}{\SI{60}{\second}}\right)^2} \\
M &= \doubleunderline{\SI{4.7}{\kilogram}}
\end{align}
\end{solution}

\part Visualiser resultatene:
\begin{solution}
PLACE GRAPHICS HERE
\end{solution}

\end{parts}

\end{questions}
\end{document}
