\documentclass[answers,a4paper,12pt]{exam}
\input{preamble.tex}

\title{{\bf{FYS101 Mekanikk}} \\ \Large{\answersareprinted Oblig XX}} 
\author{Institutt for Fysikk, REALTEK}
\date{Uke xx}

\begin{document}
\maketitle

\begin{questions}
\question Ser på superposisjoner av de to bølgene:
\begin{parts}

\part Finner amplituden til resultantbølgen ved $\delta=\frac{\pi}{6}$:
\begin{solution}
Vi starter med uttrykket for amplituden til resultantbølgen:
\begin{align}
A &= 2 y_{0} \cos \left(\frac{1}{2} \delta\right) \\
  &= 2 \cdot \SI{0.020}{\meter} \cdot \cos \left(\frac{\pi}{12}\right) \\
  &= \SI{0.039}{\meter} \\
  &= \doubleunderline{\SI{3.9}{\centi\meter}}
\end{align}
Amplituden til resultantbølgen ved $\delta=\frac{\pi}{6}$ er derfor \SI{3.9}{\centi\meter}.
\end{solution}

\part Finner amplituden til resultantbølgen ved $\delta=\frac{\pi}{3}$:
\begin{solution}
Vi bruker samme formel for amplituden:
\begin{align}
A &= 2 y_{0} \cos \left(\frac{1}{2} \delta\right) \\
  &= 2 \cdot \SI{0.020}{\meter} \cdot \cos \left(\frac{\pi}{6}\right) \\
  &= \SI{0.035}{\meter} \\
  &= \doubleunderline{\SI{3.5}{\centi\meter}}
\end{align}
Amplituden til resultantbølgen ved $\delta=\frac{\pi}{3}$ er derfor \SI{3.5}{\centi\meter}.
\end{solution}

\end{parts}

\question FILL QUESTION
\begin{parts}

\part Finner fase differansen: punkt $P$
\begin{solution}
Vi starter med å finne bølgelengden $\lambda$ ved hjelp av formelen for bølgefart:
\begin{align}
v &= f \cdot \lambda \\
\lambda &= \frac{v}{f} \\
&= \frac{\SI{343}{\meter\per\second}}{\SI{100}{\hertz}} \\
\lambda &= \SI{3.43}{\meter}
\end{align}

Deretter beregner vi faseforskjellen $\delta$:
\begin{align}
\delta &= 2 \pi \frac{\Delta x}{\lambda} \\
&= 2 \pi \frac{(x_{2} - x_{1})}{\lambda} \\
&= 2 \pi \frac{(\SI{5.85}{\meter} - \SI{5.00}{\meter})}{\SI{3.43}{\meter}} \\
&= 2 \pi \frac{\SI{0.85}{\meter}}{\SI{3.43}{\meter}} \\
&= 1.557 \, \text{rad} \\
\delta &\approx \doubleunderline{1.6 \, \text{rad}}
\end{align}
\end{solution}

\part Finner amplituden til resultantbølgen i punktet $P$
\begin{solution}
Amplituden $A$ til resultantbølgen kan beregnes ved:
\begin{align}
A &= 2 y_{0} \cos \left(\frac{1}{2} \delta\right) \\
&= 2 y_{0} \cos \left(\frac{1}{2} \cdot 1.557 \, \text{rad}\right) \\
A &\approx 1.4 y_{0}
\end{align}
Amplituden til resultantbølgen er dermed \doubleunderline{1.4 $y_{0}$}.
\end{solution}

\end{parts}

\question Beregn frekvensen og diskuter endringer i frekvens for en lydkilde:
\begin{parts}

\part Beregn den andre frekvensen $f_2$ gitt at $f_{\text{beat}} = \SI{4.0}{\hertz}$ og $f_1 = \SI{500}{\hertz}$:
\begin{solution}
\centering
\begin{align}
f_{\text{beat}} &= \Delta f \\
f_{\text{beat}} &= \pm\left(f_2 - f_1\right) \\
f_2 &= f_1 \pm f_{\text{beat}} \\
    &= \SI{500}{\hertz} \pm \SI{4.0}{\hertz} \\
f_2 &= \doubleunderline{\SI{504}{\hertz} \text{ eller } \SI{496}{\hertz}}
\end{align}
\end{solution}

\part Diskuter hva som skjer når $f_1$ endres:
\begin{solution}
Dersom $f_{\text{beat}} < \SI{4.0}{\hertz}$, vil $f_1$ ha nærmet seg verdien til $f_2$, noe som betyr at $f_1 = \SI{496}{\hertz}$. Dette indikerer at frekvensen til lydkilden har blitt lavere.

Dersom $f_{\text{beat}} > \SI{4.0}{\hertz}$, vil $f_1$ ha fjernet seg fra verdien til $f_2$, noe som betyr at $f_1 = \SI{504}{\hertz}$. Dette indikerer at frekvensen til lydkilden har blitt høyere.
\end{solution}

\end{parts}


\question Analyser bølgefunksjonen $y(x, t)=4.20 \sin (0.200 x) \cdot \cos (300 t)$:
\begin{parts}

\part Beregn bølgelengden $\lambda$:
\begin{solution}
Bølgelengden $\lambda$ kan beregnes ved å bruke bølgetallet $k$:
\begin{align}
\lambda &= \frac{2 \pi}{k} \\
        &= \frac{2 \pi}{0.200 \cdot 10^{-2} \frac{1}{\mathrm{~m}}} \\
        &= \doubleunderline{\SI{31.4}{\centi\meter}}
\end{align}
\end{solution}

\part Beregn frekvensen $f$:
\begin{solution}
Frekvensen $f$ kan beregnes ved å bruke vinkelfrekvensen $w$:
\begin{align}
f &= \frac{w}{2 \pi} \\
  &= \frac{300 \frac{1}{\mathrm{s}}}{2 \pi} \\
  &= \doubleunderline{\SI{47.7}{\hertz}}
\end{align}
\end{solution}

\part Beregn bølgefarten $v$:
\begin{solution}
Bølgefarten $v$ kan beregnes ved å bruke forholdet mellom vinkelfrekvensen $w$ og bølgetallet $k$:
\begin{align}
v &= \frac{w}{k} \\
  &= \frac{300 \frac{1}{\mathrm{s}}}{0.200 \cdot 10^{-2} \frac{1}{\mathrm{m}}} \\
  &= \doubleunderline{\SI{15}{\meter\per\second}}
\end{align}
\end{solution}

\part Beregn lengden $L$ for $n=4$:
\begin{solution}
Lengden $L$ for en stående bølge med $n$ antall noder er gitt ved:
\begin{align}
L &= \frac{n \cdot \lambda}{2}, \quad n=1,2,3,\ldots \\
L_4 &= \frac{4 \cdot \lambda}{2} \\
    &= 2 \cdot \lambda \\
    &= 2 \cdot \SI{31.4}{\centi\meter} \\
    &= \doubleunderline{\SI{62.8}{\centi\meter}}
\end{align}
\end{solution}

\end{parts}


\question Beregn plasseringen av fingeren for å spille forskjellige toner på en streng:
\begin{parts}

\part Beregn lengden for å spille en A:
\begin{solution}
Vi bruker formelen for bølgelengde i en streng:
\begin{align}
L &= \frac{v}{2f}
\end{align}
Farten i strengen er uavhengig av lengden, og vi bruker:
\begin{align}
L_A &= \frac{f_G \cdot L_G}{f_A} \\
    &= \frac{\SI{196}{\hertz} \cdot \SI{0.300}{\meter}}{\SI{220}{\hertz}} \\
    &= \SI{0.267}{\meter}
\end{align}
Forskjellen i lengde er:
\begin{align}
L_G - L_A &= \SI{0.300}{\meter} - \SI{0.267}{\meter} \\
          &= \doubleunderline{\SI{3.27}{\centi\meter}}
\end{align}
Fingeren må plasseres \SI{3.27}{\centi\meter} fra enden for å spille en A.
\end{solution}

\part Beregn lengden for å spille en B:
\begin{solution}
\begin{align}
L_B &= \frac{f_G \cdot L_G}{f_B} \\
    &= \frac{\SI{196}{\hertz} \cdot \SI{0.300}{\meter}}{\SI{247}{\hertz}} \\
    &= \SI{0.238}{\meter}
\end{align}
Forskjellen i lengde er:
\begin{align}
L_G - L_B &= \SI{0.300}{\meter} - \SI{0.238}{\meter} \\
          &= \doubleunderline{\SI{6.19}{\centi\meter}}
\end{align}
Fingeren må plasseres \SI{6.19}{\centi\meter} fra enden for å spille en B.
\end{solution}

\part Beregn lengden for å spille en C:
\begin{solution}
\begin{align}
L_C &= \frac{f_G \cdot L_G}{f_C} \\
    &= \frac{\SI{196}{\hertz} \cdot \SI{0.300}{\meter}}{\SI{262}{\hertz}} \\
    &= \SI{0.224}{\meter}
\end{align}
Forskjellen i lengde er:
\begin{align}
L_G - L_C &= \SI{0.300}{\meter} - \SI{0.224}{\meter} \\
          &= \doubleunderline{\SI{7.56}{\centi\meter}}
\end{align}
Fingeren må plasseres \SI{7.56}{\centi\meter} fra enden for å spille en C.
\end{solution}

\part Beregn lengden for å spille en D:
\begin{solution}
\begin{align}
L_0 &= \frac{f_G \cdot L_G}{f_0} \\
    &= \frac{\SI{196}{\hertz} \cdot \SI{0.300}{\meter}}{\SI{294}{\hertz}} \\
    &= \SI{0.200}{\meter}
\end{align}
Forskjellen i lengde er:
\begin{align}
L_G - L_0 &= \SI{0.300}{\meter} - \SI{0.200}{\meter} \\
          &= \doubleunderline{\SI{10.0}{\centi\meter}}
\end{align}
Fingeren må plasseres \SI{10.0}{\centi\meter} fra enden for å spille en D.
\end{solution}

\end{parts}
\end{questions}
\end{document}
