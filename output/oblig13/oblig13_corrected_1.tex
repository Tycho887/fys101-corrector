\documentclass[answers,a4paper,12pt]{exam}
\input{preamble.tex}

\title{{\bf{FYS101 Mekanikk}} \\ \Large{\answersareprinted Oblig XX}} 
\author{Institutt for Fysikk, REALTEK}
\date{Uke xx}

\begin{document}
\maketitle

\begin{questions}
\question Ser på superposisjoner av de to bølgene
\begin{parts}

\part Finner amplituden til resultantbølgen ved $\delta=\frac{\pi}{6}$:
\begin{solution}
Vi har uttrykket for amplituden til resultantbølgen som:
\begin{align}
A &= 2 y_{0} \cos \left(\frac{1}{2} \delta\right) \\
  &= 2 \cdot \SI{0.020}{\meter} \cdot \cos \left(\frac{\pi}{12}\right) \\
  &= \doubleunderline{\SI{0.039}{\meter}}
\end{align}
Her har vi brukt verdien $\delta = \frac{\pi}{6}$ for å finne amplituden. Resultatet er \SI{3.9}{\centi\meter}.
\end{solution}

\part Finner amplituden til resultantbølgen ved $\delta=\frac{\pi}{3}$:
\begin{solution}
På samme måte bruker vi uttrykket for amplituden:
\begin{align}
A &= 2 y_{0} \cos \left(\frac{1}{2} \delta\right) \\
  &= 2 \cdot \SI{0.020}{\meter} \cdot \cos \left(\frac{\pi}{6}\right) \\
  &= \doubleunderline{\SI{0.035}{\meter}}
\end{align}
Her har vi brukt verdien $\delta = \frac{\pi}{3}$ for å finne amplituden. Resultatet er \SI{3.5}{\centi\meter}.
\end{solution}

\end{parts}

\question FILL QUESTION
\begin{parts}

\part Finner fase differansen: purikt $p$
\begin{solution}
Vi skal finne faseforskjellen mellom to punkter i en bølge. Først beregner vi bølgelengden $\lambda$ ved hjelp av formelen for bølgefart $v = f \cdot \lambda$:

\centering
\begin{align}
v &= f \cdot \lambda \\
\lambda &= \frac{v}{f} \\
&= \frac{\SI{343}{\meter\per\second}}{\SI{100}{\hertz}} \\
\lambda &= \SI{3.43}{\meter}
\end{align}

Nå kan vi beregne faseforskjellen $\delta$:

\begin{align}
\delta &= 2 \pi \frac{\Delta x}{\lambda} \\
&= 2 \pi \frac{(x_{2} - x_{1})}{\lambda} \\
&= 2 \pi \frac{(\SI{5.85}{\meter} - \SI{5.00}{\meter})}{\SI{3.43}{\meter}} \\
&= 2 \pi \frac{\SI{0.85}{\meter}}{\SI{3.43}{\meter}} \\
&= 1.557 \, \text{rad} \\
\delta &\approx \doubleunderline{\SI{1.6}{\radian}}
\end{align}
\end{solution}

\part Finner amplituden til resultantbølgen i punktet $P$
\begin{solution}
Amplituden til resultantbølgen kan beregnes ved hjelp av formelen:

\centering
\begin{align}
A &= 2 y_{0} \cos \left(\frac{1}{2} \delta\right) \\
&= 2 y_{0} \cos \left(\frac{1}{2} \cdot \SI{1.557}{\radian}\right) \\
A &\approx 1.4 y_{0}
\end{align}
\end{solution}

\end{parts}

\question FILL QUESTION
\begin{parts}

\part Part A: Beregn frekvensen $f_2$ gitt at $f_{\text{beat}} = \SI{4.0}{\hertz}$ og $f_1 = \SI{500}{\hertz}$.
\begin{solution}
Vi starter med formelen for beat-frekvensen:
\begin{align}
f_{\text{beat}} &= \Delta f \\
f_{\text{beat}} &= \pm\left(f_2 - f_1\right)
\end{align}
Vi ønsker å finne $f_2$, så vi omorganiserer ligningen:
\begin{align}
f_2 &= f_1 \pm f_{\text{beat}} \\
    &= \SI{500}{\hertz} \pm \SI{4.0}{\hertz} \\
    &= \SI{504}{\hertz} \text{ eller } \SI{496}{\hertz}
\end{align}
Dermed kan $f_2$ være enten \doubleunderline{\SI{504}{\hertz}} eller \doubleunderline{\SI{496}{\hertz}}.
\end{solution}

\part Part B: Diskuter hvordan endringer i $f_{\text{beat}}$ påvirker $f_1$.
\begin{solution}
Dersom $f_{\text{beat}} < \SI{4.0}{\hertz}$, vil $f_1$ nærme seg verdien til $f_2$. Dette betyr at $f_1$ kan være \SI{496}{\hertz}.

På den annen side, dersom $f_{\text{beat}} > \SI{4.0}{\hertz}$, vil $f_1$ bevege seg bort fra verdien til $f_2$. Dette indikerer at $f_1$ kan være \SI{504}{\hertz}.

Dette viser hvordan endringer i beat-frekvensen kan gi innsikt i forholdet mellom de to frekvensene $f_1$ og $f_2$.
\end{solution}

\end{parts}

\question FILL QUESTION
\begin{parts}

\part Part A: Beregn bølgelengden
\begin{solution}
Bølgelengden $\lambda$ kan beregnes ved hjelp av bølgetallet $k$. Formelen for bølgelengde er gitt ved:
\begin{align}
\lambda &= \frac{2 \pi}{k} \\
        &= \frac{2 \pi}{0.200 \cdot 10^{-2} \, \si{\per\meter}} \\
        &= \doubleunderline{\SI{31.4}{\centi\meter}}
\end{align}
\end{solution}

\part Part B: Beregn frekvensen
\begin{solution}
Frekvensen $f$ kan beregnes ved hjelp av vinkelfrekvensen $w$. Formelen for frekvens er:
\begin{align}
f &= \frac{w}{2 \pi} \\
  &= \frac{300 \, \si{\per\second}}{2 \pi} \\
  &= \doubleunderline{\SI{47.7}{\hertz}}
\end{align}
\end{solution}

\part Part C: Beregn bølgefarten
\begin{solution}
Bølgefarten $v$ kan beregnes ved å bruke forholdet mellom vinkelfrekvens $w$ og bølgetall $k$:
\begin{align}
v &= \frac{w}{k} \\
  &= \frac{300 \, \si{\per\second}}{0.200 \cdot 10^{-2} \, \si{\per\meter}} \\
  &= \doubleunderline{\SI{15}{\meter\per\second}}
\end{align}
\end{solution}

\part Part D: Beregn lengden for $n=4$
\begin{solution}
Lengden $L$ for en stående bølge med $n$ antall noder er gitt ved:
\begin{align}
L &= \frac{n \cdot \lambda}{2}, \quad n=1,2,3,\ldots \\
L_4 &= \frac{4 \cdot \lambda}{2} \\
    &= 2 \cdot \lambda \\
    &= 2 \cdot \SI{31.4}{\centi\meter} \\
    &= \doubleunderline{\SI{62.8}{\centi\meter}}
\end{align}
\end{solution}

\end{parts}

\question FILL QUESTION
\begin{parts}

\part Beregn lengden for å spille en A:
\begin{solution}
Vi bruker formelen for lengden av strengen når frekvensen endres:
\begin{align}
L_A &= \frac{f_G \cdot L_G}{f_A} \\
    &= \frac{\SI{196}{\hertz} \cdot \SI{0.300}{\meter}}{\SI{220}{\hertz}} \\
    &= \SI{0.267}{\meter}
\end{align}
Forskjellen i lengde er:
\begin{align}
L_G - L_A &= \SI{0.300}{\meter} - \SI{0.267}{\meter} \\
          &= \doubleunderline{\SI{3.27}{\centi\meter}}
\end{align}
Fingeren må plasseres \SI{3.27}{\centi\meter} fra enden for å spille en A.
\end{solution}

\part Beregn lengden for å spille en B:
\begin{solution}
\begin{align}
L_B &= \frac{f_G \cdot L_G}{f_B} \\
    &= \frac{\SI{196}{\hertz} \cdot \SI{0.300}{\meter}}{\SI{247}{\hertz}} \\
    &= \SI{0.238}{\meter}
\end{align}
Forskjellen i lengde er:
\begin{align}
L_G - L_B &= \SI{0.300}{\meter} - \SI{0.238}{\meter} \\
          &= \doubleunderline{\SI{6.19}{\centi\meter}}
\end{align}
Fingeren må plasseres \SI{6.19}{\centi\meter} fra enden for å spille en B.
\end{solution}

\part Beregn lengden for å spille en C:
\begin{solution}
\begin{align}
L_C &= \frac{f_G \cdot L_G}{f_C} \\
    &= \frac{\SI{196}{\hertz} \cdot \SI{0.300}{\meter}}{\SI{262}{\hertz}} \\
    &= \SI{0.224}{\meter}
\end{align}
Forskjellen i lengde er:
\begin{align}
L_G - L_C &= \SI{0.300}{\meter} - \SI{0.224}{\meter} \\
          &= \doubleunderline{\SI{7.56}{\centi\meter}}
\end{align}
Fingeren må plasseres \SI{7.56}{\centi\meter} fra enden for å spille en C.
\end{solution}

\part Beregn lengden for å spille en D:
\begin{solution}
\begin{align}
L_0 &= \frac{f_G \cdot L_G}{f_0} \\
    &= \frac{\SI{196}{\hertz} \cdot \SI{0.300}{\meter}}{\SI{294}{\hertz}} \\
    &= \SI{0.200}{\meter}
\end{align}
Forskjellen i lengde er:
\begin{align}
L_G - L_0 &= \SI{0.300}{\meter} - \SI{0.200}{\meter} \\
          &= \doubleunderline{\SI{10.0}{\centi\meter}}
\end{align}
Fingeren må plasseres \SI{10.0}{\centi\meter} fra enden for å spille en D.
\end{solution}

\end{parts}
\end{questions}
\end{document}
