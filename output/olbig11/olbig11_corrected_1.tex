\documentclass[answers,a4paper,12pt]{exam}
\input{preamble.tex}

\title{{\bf{FYS101 Mekanikk}} \\ \Large{\answersareprinted Oblig XX}} 
\author{Institutt for Fysikk, REALTEK}
\date{Uke xx}

\begin{document}
\maketitle

\begin{questions}
\question FILL QUESTION
\begin{parts}

\part Definer fasekonstanten i en svingning:
\begin{solution}
Fasekonstanten, ofte betegnet som $\delta$, er en konstant som angir fasen til en svingning ved tiden $t=0$. Den bestemmer hvor i svingningssyklusen bevegelsen starter.
\end{solution}

\part Forklar fenomenet resonans:
\begin{solution}
Resonans er et fenomen som oppstår når frekvensen til en ytre, periodisk kraft er lik egenfrekvensen til systemet. Dette fører til at systemet får en større amplitude i svingningene, noe som kan resultere i betydelige vibrasjoner.
\end{solution}

\part Definer Q-verdi i dempede svingninger:
\begin{solution}
Q-verdi beskriver kvaliteten av dempede svingninger og er relatert til hvor mye energi som tapes i en svingesyklus. Den er gitt ved:
\begin{align}
Q = \frac{2 \pi}{\left(\frac{\Delta E}{E}\right) \text{ syklus}}
\end{align}
hvor $\Delta E$ er energitapet per syklus og $E$ er den totale energien.
\end{solution}

\part Forklar tidskonstanten $\tau$:
\begin{solution}
Tidskonstanten $\tau$ er tiden det tar for energien i svingningen å bli redusert med en faktor $e^{-1}$, som tilsvarer omtrent 37\% av den opprinnelige energien.
\end{solution}

\end{parts}

\question FILL QUESTION
\begin{parts}

\part Del A: Finn frekvensen til partikkelen
\begin{solution}
Frekvensen kan finnes ved å bruke formelen for vinkelfrekvens:
\begin{align}
\omega &= 2 \pi f \\
f &= \frac{\omega}{2 \pi} \\
  &= \frac{6 \pi}{2 \pi} \\
  &= \doubleunderline{\SI{3.0}{\hertz}}
\end{align}
Frekvensen til partikkelen er altså \SI{3.0}{\hertz}.
\end{solution}

\part Del B: Finn perioden til partikkelen
\begin{solution}
Perioden \( T \) er den inverse av frekvensen:
\begin{align}
T &= \frac{1}{f} \\
  &= \frac{1}{\SI{3.0}{\hertz}} \\
  &= \doubleunderline{\SI{0.33}{\second}}
\end{align}
Partikkelen bruker \SI{0.33}{\second} på én full svingning.
\end{solution}

\part Del C: Finn amplituden fra funksjonen som beskriver posisjonen til partikkelen
\begin{solution}
Amplituden \( A \) kan leses direkte fra posisjonsfunksjonen:
\begin{align}
x &= A \cos (\omega t + \phi) \\
A &= \doubleunderline{\SI{7.0}{\centi\meter}}
\end{align}
Amplituden for svingningen er \SI{7.0}{\centi\meter} fra likevektsposisjonen.
\end{solution}

\part Del D: Bestem når partikkelen er ved likevekt for første gang etter \( t=0 \)
\begin{solution}
Partikkelen er ved likevekt når \(\cos(6 \pi t) = 0\). Dette skjer når:
\begin{align}
\cos\left(n \pi + \frac{\pi}{2}\right) &= 0, \quad \text{der } n = 0, \pm 1, \pm 2, \ldots \\
6 \pi t &= \frac{\pi}{2} \\
t &= \frac{1}{12} \\
t &= \doubleunderline{\SI{0.083}{\second}}
\end{align}
Partikkelen er i likevekt \SI{0.083}{\second} etter \( t=0 \), og beveger seg i negativ \( x \)-retning.
\end{solution}

\end{parts}


\question FILL QUESTION
\begin{parts}

\part Finner vinkelfrekvensen til bevegelsen når klossen slippes:
\begin{solution}
Vinkelfrekvensen, $\omega$, kan beregnes ved hjelp av formelen:
\begin{align}
\omega &= \sqrt{\frac{k}{m}} \\
       &= \sqrt{\frac{\SI{700}{\newton\per\meter}}{\SI{5.00}{\kilogram}}} \\
       &= \doubleunderline{\SI{11.832}{\radian\per\second}}
\end{align}
Frekvensen, $f$, kan deretter beregnes som:
\begin{align}
f &= \frac{\omega}{2\pi} \\
  &= \frac{\SI{11.832}{\radian\per\second}}{2\pi} \\
  &= \doubleunderline{\SI{1.8831}{\hertz}}
\end{align}
Frekvensen når klossen slippes er altså \SI{1.88}{\hertz}, og siden underlaget er friksjonsløst, vil frekvensen forbli konstant.
\end{solution}

\part Finner perioden:
\begin{solution}
Perioden, $T$, er den inverse av frekvensen:
\begin{align}
T &= \frac{1}{f} \\
  &= \frac{1}{\SI{1.8831}{\hertz}} \\
  &= \doubleunderline{\SI{0.531}{\second}}
\end{align}
Klossen bruker \SI{0.531}{\second} på én full svingning.
\end{solution}

\part Finner amplituden ved å benytte likningen for posisjonen:
\begin{solution}
For å finne amplituden, $A$, benytter vi posisjonslikningen:
\begin{align}
v &= \frac{dx}{dt} = -\omega A \sin(\omega t + \delta)
\end{align}
Siden $v(0) = v_0 = 0$, har vi:
\begin{align}
\sin(\delta) &= 0 \\
\delta &= 0
\end{align}
Dermed er:
\begin{align}
x_0 &= A \cos(\omega \cdot 0 + 0) \\
A &= x_0 \\
A &= \doubleunderline{\SI{8.00}{\centi\meter}}
\end{align}
Amplituden blir bestemt av hvor langt klossen strekkes, og siden underlaget er friksjonsløst, blir amplituden \SI{8.00}{\centi\meter}.
\end{solution}

\part Finner maksimalhastigheten:
\begin{solution}
Maksimalhastigheten, $v_{\text{max}}$, kan beregnes som:
\begin{align}
v_{\text{max}} &= \omega A \\
               &= \SI{11.832}{\radian\per\second} \cdot \SI{8.00e-2}{\meter} \\
               &= \doubleunderline{\SI{0.947}{\meter\per\second}}
\end{align}
Maksimalhastigheten til klossen er \SI{0.947}{\meter\per\second}.
\end{solution}

\part Finner maksimalakselerasjonen:
\begin{solution}
Maksimalakselerasjonen, $a_{\text{max}}$, kan beregnes som:
\begin{align}
a_{\text{max}} &= \omega^2 A \\
               &= \left(\SI{11.832}{\radian\per\second}\right)^2 \cdot \SI{8.00e-2}{\meter} \\
               &= \doubleunderline{\SI{11.2}{\meter\per\second\squared}}
\end{align}
\end{solution}

\part Finner et uttrykk for tid ved $x=0$:
\begin{solution}
Posisjonen $x$ kan uttrykkes som:
\begin{align}
x &= A \cos(\omega t) \\
0 &= \SI{8.00e-2}{\meter} \cdot \cos(\SI{11.832}{\radian\per\second} \cdot t)
\end{align}
Løsning for $t$ når $\cos(\omega t) = 0$:
\begin{align}
\omega t &= \frac{\pi}{2} \\
t &= \frac{\pi}{2 \cdot \SI{11.832}{\radian\per\second}} \\
t &= \doubleunderline{\SI{0.133}{\second}}
\end{align}
\end{solution}

\end{parts}



\question Finner resonansfrekvens for elastisk pendel:
\begin{parts}

\part Definer resonansfrekvens for en elastisk pendel:
\begin{solution}
Resonansfrekvensen for en elastisk pendel er frekvensen ved hvilken systemet naturlig oscillerer når det ikke påvirkes av eksterne krefter. Den kan beregnes ved hjelp av fjærkonstanten \( k \) og massen \( m \) i systemet.
\end{solution}

\part Beregn resonansfrekvensen når \( k = \SI{400}{\newton\per\meter} \) og \( m = \SI{10}{\kilo\gram} \):
\begin{solution}
\centering
\begin{align}
f &= \frac{1}{2 \pi} \sqrt{\frac{k}{m}} \\
  &= \frac{1}{2 \pi} \sqrt{\frac{\SI{400}{\newton\per\meter}}{\SI{10}{\kilo\gram}}} \\
  &= \frac{1}{2 \pi} \sqrt{\SI{40}{\per\second\squared}} \\
  &= \frac{1}{2 \pi} \cdot \SI{6.3246}{\per\second} \\
  &= \doubleunderline{\SI{1.0}{\hertz}}
\end{align}
\end{solution}

\part Beregn resonansfrekvensen for en elastisk pendel der \( k = \SI{800}{\newton\per\meter} \) og \( m = \SI{5.0}{\kilo\gram} \):
\begin{solution}
\centering
\begin{align}
f &= \frac{1}{2 \pi} \sqrt{\frac{k}{m}} \\
  &= \frac{1}{2 \pi} \sqrt{\frac{\SI{800}{\newton\per\meter}}{\SI{5.0}{\kilo\gram}}} \\
  &= \frac{1}{2 \pi} \sqrt{\SI{160}{\per\second\squared}} \\
  &= \frac{1}{2 \pi} \cdot \SI{12.6491}{\per\second} \\
  &= \doubleunderline{\SI{2.0}{\hertz}}
\end{align}
\end{solution}

\part Beregn resonansfrekvensen for en matematisk pendel med \( L = \SI{2.0}{\meter} \) og \( m = \SI{40}{\kilo\gram} \):
\begin{solution}
For en matematisk pendel er resonansfrekvensen gitt ved:
\begin{align}
f &= \frac{1}{2 \pi} \sqrt{\frac{g}{L}} \\
  &= \frac{1}{2 \pi} \sqrt{\frac{\SI{9.81}{\meter\per\second\squared}}{\SI{2.0}{\meter}}} \\
  &= \frac{1}{2 \pi} \sqrt{\SI{4.905}{\per\second\squared}} \\
  &= \frac{1}{2 \pi} \cdot \SI{2.2147}{\per\second} \\
  &= \doubleunderline{\SI{0.35}{\hertz}}
\end{align}
\end{solution}

\end{parts}



\question FILL QUESTION
\begin{parts}

\part Del A: Finn ukjent frekvens for blokkens svingning
\begin{solution}
Vi starter med å finne den naturlige vinkelfrekvensen til blokkens svingning. Den naturlige vinkelfrekvensen, $\omega_0$, er gitt ved:

\centering
\begin{align}
\omega_0 &= \sqrt{\frac{k}{m}} \\
         &= \sqrt{\frac{\SI{400}{\newton\per\meter}}{\SI{200}{\kilo\gram}}} \\
         &= \doubleunderline{\SI{1.414}{\radian\per\second}}
\end{align}

Ved å bruke denne frekvensen kan vi finne amplituden til svingningene:

\centering
\begin{align}
A &= \SI{4.98}{\meter}
\end{align}
\end{solution}

\part Del B: Resonans ved naturlig frekvens
\begin{solution}
Resonans oppstår ved den naturlige frekvensen $\omega_0 = \SI{1.414}{\radian\per\second}$. Ved en vinkel på \SI{14}{\radian} vil svingningene nå en resonansfrekvens. Amplituden ved resonans kan beregnes ved:

\centering
\begin{align}
A &= \frac{F_a}{\sqrt{m^2\left(\omega_0^2 - \omega_d^2\right)^2 + (b \omega \sqrt{a})^2}}
\end{align}
\end{solution}

\begin{center}
\includegraphics[width=0.5\linewidth]{path/to/image}
\end{center}

\end{parts}

\end{questions}
\end{document}
