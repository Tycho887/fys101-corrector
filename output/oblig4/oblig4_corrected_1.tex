\documentclass[answers,a4paper,12pt]{exam}
\input{preamble.tex}

\title{{\bf{FYS101 Mekanikk}} \\ \Large{\answersareprinted Oblig XX}} 
\author{Institutt for Fysikk, REALTEK}
\date{Uke xx}

\begin{document}
\maketitle

\begin{questions}
\question FYS101 Løsningsforslag oblig 4

\begin{parts}

\part Del A: Arbeid - generelt
\begin{solution}
Arbeid er definert som produktet av kraften som virker på et objekt og forflytningen av objektet i kraftens retning. Matematisk uttrykkes dette som:
\begin{align}
W &= F \cdot d \cdot \cos(\theta)
\end{align}
hvor \(W\) er arbeidet, \(F\) er kraften, \(d\) er forflytningen, og \(\theta\) er vinkelen mellom kraften og forflytningen.
\end{solution}

\part PLACE GRAPHICS HERE
\begin{solution}
Placeholder for graphics: \includegraphics[max width=\textwidth, center]{path/to/image}
\end{solution}

\end{parts}

\question Potensiell energi - generelt
\begin{parts}
\part Forklar hva potensiell energi er:
\begin{solution}
Potensiell energi er energien et objekt har på grunn av sin posisjon eller tilstand. For eksempel, et objekt som er løftet over bakken har potensiell energi på grunn av tyngdekraften. Denne energien kan omdannes til kinetisk energi når objektet faller.
\end{solution}

\part Utled uttrykket for endring i potensiell energi:
\begin{solution}
Endringen i potensiell energi, \(\Delta U\), mellom to punkter kan uttrykkes som:
\begin{align}
\Delta U &= U_{2} - U_{1} \\
        &= -\int_{1}^{2} \vec{F} \cdot d \vec{U}
\end{align}
Her representerer \(\vec{F}\) kraften som virker på objektet, og \(d \vec{U}\) er den differensielle forflytningen. Integralet beregner arbeidet utført av kraften når objektet beveger seg fra posisjon 1 til posisjon 2.
\end{solution}

\part Illustrer konseptet med en grafisk fremstilling:
\begin{solution}
PLACE GRAPHICS HERE
\end{solution}
\end{parts}


\question FILL QUESTION
\begin{parts}

\part Del A: Bevaring av energi
\begin{solution}
Vi har bevaring av energi ettersom det ikke virker noen friksjonskraft på klossen. Vi bruker arbeid-energi-loven:

\begin{align}
E_{1} &= E_{0} \\
K_{1} + U_{g_{1}} + U_{f_{1}} &= K_{0} + U_{g_{0}} + U_{f_{0}} \\
\frac{1}{2} m v_{1}^{2} + m g y_{1} + \frac{1}{2} k x_{1}^{2} &= \frac{1}{2} m v_{0}^{2} + m g y_{0} + \frac{1}{2} k x_{0}^{2}
\end{align}

Siden klossen starter i ro og slutter i ro, har vi $v_{0} = v_{1} = 0$, og vi sitter igjen med:

\begin{align}
\frac{1}{2} k x_{1}^{2} &= m g y_{0}
\end{align}

Vi løser for $x_{1}$:

\begin{align}
x_{1} &= \sqrt{\frac{2 m g y_{0}}{k}} \\
      &= \sqrt{\frac{2 \cdot \SI{3.00}{\kilo\gram} \cdot \SI{9.81}{\meter\per\second\squared} \cdot \SI{5.00}{\meter}}{\SI{400}{\newton\per\meter}}} \\
x_{1} &= \doubleunderline{\SI{0.858}{\meter}}
\end{align}

Fjæren ble komprimert omtrent \SI{85.8}{\centi\meter}.
\end{solution}

\part Del B: Arbeid fra fjæren
\begin{solution}
Blokken vil bli tilført arbeid fra fjæren som vil skyve den i negativ $x$-retning. Siden energien er bevart, vil klossen skyves tilbake til opprinnelig høyde.
\end{solution}

\end{parts}



\question FILL QUESTION
\begin{parts}

\part Part A: Beregn hastigheten $v_1$ ved bunnen av rampen.
\begin{solution}
Vi bruker arbeid-energi-prinsippet:
\begin{align}
E_1 &= E_0 \\
K_1 + U_1 &= K_0 + U_0 \\
\frac{1}{2} m v_1^2 + m g y_1 &= \frac{1}{2} m v_0^2 + m g y_0 \\
\frac{1}{2} m v_1^2 &= m g y_0 \\
v_1 &= \sqrt{2 g y_0} \\
    &= \sqrt{2 \cdot 9.81 \, \frac{\mathrm{N}}{\mathrm{kg}} \cdot 3.0 \, \mathrm{m}} \\
    &= \doubleunderline{\SI{7.7}{\meter\per\second}}
\end{align}
Klossen har en hastighet på \SI{7.7}{\meter\per\second} ved bunnen av rampen.
\end{solution}

\part Part B: Beregn friksjonsarbeidet $W_R$ som virker på klossen i prosess (2).
\begin{solution}
Vi bruker arbeid-energi-prinsippet igjen:
\begin{align}
E_2 &= E_1 + W_R \\
K_2 + U_2 &= K_1 + U_1 + W_R \\
\frac{1}{2} m v_2^2 + m g y_2 &= \frac{1}{2} m v_1^2 + m g y_1 + W_R
\end{align}
Løs for friksjonsarbeidet $W_R$:
\begin{align}
W_R &= -\frac{1}{2} m v_1^2 \\
    &= -\frac{1}{2} \cdot 2.0 \, \mathrm{kg} \cdot \left(7.7 \, \frac{\mathrm{m}}{\mathrm{s}}\right)^2 \\
    &= \doubleunderline{-\SI{59}{\joule}}
\end{align}
Et arbeid på \SI{59}{\joule} forsvinner som resultat av friksjon.
\end{solution}

\part Part C: Beregn friksjonskoeffisienten $\mu$ mellom klossen og det horisontale underlaget.
\begin{solution}
Vi bruker Newtons 2. lov i $y$-retningen:
\begin{align}
\sum F_y &= m a_y \quad (a_y = 0, \text{ all bevegelse skjer i } x\text{-retning}) \\
F_n - F_g &= 0 \\
F_n &= m g \quad \text{(I)} \\
f &= \mu F_n \quad \text{(II)}
\end{align}
Sett (I) inn i (II):
\begin{align}
f &= \mu m g \quad \text{(III)}
\end{align}
Friksjonsarbeidet er gitt ved:
\begin{align}
W_R &= f s \cdot \cos 180^\circ \quad \text{(IV)}
\end{align}
Sett (III) inn i (IV) og løs for $\mu$:
\begin{align}
W_R &= -\mu m g s \\
\mu &= -\frac{W_R}{m g s} \\
    &= -\frac{-\SI{59}{\joule}}{2.0 \, \mathrm{kg} \cdot 9.81 \, \mathrm{m/s^2} \cdot 9.0 \, \mathrm{m}} \\
    &= \doubleunderline{0.33}
\end{align}
Friksjonskoeffisienten mellom klossen og det horisontale underlaget er 0.33.
\end{solution}

\end{parts}


\question FILL QUESTION
\begin{parts}

\part Definer en funksjon med to parametere som returnerer kraften utført av friksjon med en friksjonskoeffisient $\mu$ på et horisontalt plan, basert på den normale kraften.
\begin{solution}
Friksjonskraften $F_f$ kan defineres som produktet av den normale kraften $N$ og friksjonskoeffisienten $\mu$. Dette kan uttrykkes som:
\begin{align}
F_f &= \mu \cdot N
\end{align}
\end{solution}

\part Lag en funksjon kalt 'kinetisk energi' som returnerer den kinetiske energien. Funksjonen skal ha to parametere: masse og hastighet.
\begin{solution}
Den kinetiske energien $K$ til en gjenstand kan uttrykkes som:
\begin{align}
K &= \frac{1}{2} \cdot m \cdot v^2
\end{align}
hvor $m$ er massen og $v$ er hastigheten til gjenstanden.
\end{solution}

\part En curlingstein med en hastighet på \SI{1.0}{\meter\per\second} reiser nøyaktig \SI{6.4}{\meter} og stopper i sentrum av curlingmålet. Finn friksjonskoeffisienten $\mu$.
\begin{solution}
Vi vet at den mekaniske energien $E_{\text{mech}}$ er lik den kinetiske energien $K$ siden potensiell energi $U$ ikke endres. Ved bruk av energiprinsippet $\Delta E_{\text{mech}} = -W_{\text{nonconservative}}$, kan vi finne friksjonskoeffisienten $\mu$:
\begin{align}
\Delta E_{\text{mech}} &= K \\
W_{\text{nonconservative}} &= \mu \cdot m \cdot g \cdot d \\
\mu &= \frac{K}{m \cdot g \cdot d} \\
\mu &= \frac{\frac{1}{2} \cdot m \cdot v^2}{m \cdot g \cdot \SI{6.4}{\meter}} \\
\mu &= \frac{\frac{1}{2} \cdot \SI{20}{\kilo\gram} \cdot (\SI{1.0}{\meter\per\second})^2}{\SI{20}{\kilo\gram} \cdot \SI{9.81}{\meter\per\second\squared} \cdot \SI{6.4}{\meter}} \\
\mu &= \doubleunderline{0.0080}
\end{align}
\end{solution}

\part Plot hastigheten til curlingsteinen mens den beveger seg fra startlinjen (hog line) til sentrum av målet, gitt friksjonskoeffisienten $\mu$.
\begin{solution}
Hastigheten $v$ til curlingsteinen som en funksjon av avstanden $L$ kan uttrykkes som:
\begin{align}
v &= \sqrt{\frac{2K}{m} - 2\mu g L}
\end{align}
Vi definerer en avstandsarray fra \SI{0}{\meter} til \SI{6.4}{\meter} og plotter hastigheten ved hjelp av denne funksjonen.
\begin{center}
PLACE GRAPHICS HERE
\end{center}
\end{solution}

\end{parts}
\end{questions}
\end{document}
