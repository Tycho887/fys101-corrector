\documentclass[answers,a4paper,12pt]{exam}
\input{preamble.tex}

\title{{\bf{FYS101 Mekanikk}} \\ \Large{\answersareprinted Oblig XX}} 
\author{Institutt for Fysikk, REALTEK}
\date{Uke xx}

\begin{document}
\maketitle

\begin{questions}
\question FILL QUESTION
\begin{parts}
\part Del A: Arbeid - generelt
\begin{solution}
Arbeid utført av en kraft på et objekt er definert som produktet av kraftens komponent i bevegelsesretningen og forflytningen av objektet. Matematisk uttrykkes dette som:
\begin{align}
W &= F \cdot d \cdot \cos(\theta)
\end{align}
hvor \( W \) er arbeidet, \( F \) er kraften, \( d \) er forflytningen, og \( \theta \) er vinkelen mellom kraften og forflytningen.
\end{solution}

\part Placeholder for graphics
\begin{solution}
Placeholder for graphics: \includegraphics[max width=\textwidth, center]{path/to/image}
\end{solution}
\end{parts}

\question FILL QUESTION
\begin{parts}
\part Forklar potensiell energi generelt:
\begin{solution}
Potensiell energi er energien som et objekt har på grunn av sin posisjon i et kraftfelt, som for eksempel et gravitasjonsfelt. Det er energien som kan omdannes til kinetisk energi når objektet beveger seg. Den generelle formelen for endring i potensiell energi er gitt ved:
\begin{align}
\Delta U &= U_{2} - U_{1} \\
        &= -\int_{1}^{2} \vec{F} \cdot d \vec{U}
\end{align}
Her representerer \( \vec{F} \) kraften som virker på objektet, og \( d \vec{U} \) er en infinitesimal forflytning i kraftfeltet.
\end{solution}

\part Beregn endringen i potensiell energi for et objekt som beveger seg i et kraftfelt:
\begin{solution}
Anta at vi har en konstant kraft \( \vec{F} \) som virker på et objekt. Da kan vi beregne endringen i potensiell energi som følger:
\begin{align}
\Delta U &= -\int_{1}^{2} \vec{F} \cdot d \vec{U} \\
        &= -\vec{F} \cdot \int_{1}^{2} d \vec{U} \\
        &= -\vec{F} \cdot (\vec{U}_{2} - \vec{U}_{1})
\end{align}
Hvis kraften er konstant og i samme retning som forflytningen, kan vi forenkle dette til:
\begin{align}
\Delta U &= -F \cdot (U_{2} - U_{1})
\end{align}
Dette gir oss endringen i potensiell energi som følge av forflytningen i kraftfeltet.
\end{solution}
\end{parts}

\question FILL QUESTION
\begin{parts}

\part Del A: Bevaring av energi
\begin{solution}
Vi har bevaring av energi ettersom det ikke virker noen friksjonskraft på klossen. Vi bruker arbeid-energi-loven:

\begin{align}
E_{1} &= E_{0} \\
K_{1} + U_{g_{1}} + U_{f_{1}} &= K_{0} + U_{g_{0}} + U_{f_{0}} \\
\frac{1}{2} m v_{1}^{2} + m g y_{1} + \frac{1}{2} k x_{1}^{2} &= \frac{1}{2} m v_{0}^{2} + m g y_{0} + \frac{1}{2} k x_{0}^{2}
\end{align}

Klossen starter i ro og slutter i ro, som gir $v_{0} = v_{1} = 0$, og vi sitter igjen med:

\begin{align}
\frac{1}{2} k x_{1}^{2} &= m g y_{0}
\end{align}

Vi løser for $x_{1}$:

\begin{align}
x_{1} &= \sqrt{\frac{2 m g y_{0}}{k}} \\
      &= \sqrt{\frac{2 \cdot \SI{3.00}{\kilo\gram} \cdot \SI{9.81}{\meter\per\second\squared} \cdot \SI{5.00}{\meter}}{\SI{400}{\newton\per\meter}}} \\
x_{1} &= \doubleunderline{\SI{0.858}{\meter}}
\end{align}

Fjæren ble komprimert omtrent \SI{85.8}{\centimeter}.
\end{solution}

\part Del B: Arbeid utført av fjæren
\begin{solution}
Blokken vil bli tilført arbeid fra fjæren som vil skyve den i negativ $x$-retning. Siden energien er bevart, vil klossen skyves tilbake til opprinnelig høyde.
\end{solution}

\end{parts}


\question FILL QUESTION

\begin{parts}

\part Ser på prosessen i figur (1) og beregn $v_{1}$:

\begin{solution}
Vi bruker arbeid-energi-loven:
\begin{align}
E_{1} &= E_{0} \\
K_{1} + U_{1} &= K_{0} + U_{0} \\
\frac{1}{2} m v_{1}^{2} + m g y_{1} &= \frac{1}{2} m v_{0}^{2} + m g y_{0} \\
\frac{1}{2} m v_{1}^{2} &= m g y_{0} \\
v_{1} &= \sqrt{2 g y_{0}} \\
     &= \sqrt{2 \cdot 9.81 \frac{\mathrm{N}}{\mathrm{kg}} \cdot 3.0 \mathrm{m}} \\
     &= \doubleunderline{\SI{7.67}{\meter\per\second}}
\end{align}
Blokken har en fart på \SI{7.67}{\meter\per\second} ved bunnen av rampen.
\end{solution}

\part Beregn friksjonsarbeidet som virker på blokken i prosess (2):

\begin{solution}
Vi bruker arbeid-energi-loven igjen:
\begin{align}
E_{2} &= E_{1} + W_{R} \\
K_{2} + U_{2} &= K_{1} + U_{1} + W_{R} \\
\frac{1}{2} m v_{2}^{2} + m g y_{2} &= \frac{1}{2} m v_{1}^{2} + m g y_{1} + W_{R}
\end{align}
Løser for friksjonsarbeidet $W_{R}$:
\begin{align}
W_{R} &= -\frac{1}{2} m v_{1}^{2} \\
      &= -\frac{1}{2} \cdot 2.0 \mathrm{~kg} \cdot \left(\SI{7.67}{\meter\per\second}\right)^{2} \\
      &= \doubleunderline{-\SI{59}{\joule}}
\end{align}
Et arbeid på \SI{59}{\joule} forsvinner som resultat av friksjon.
\end{solution}

\part Finn friksjonskoeffisienten ved å bruke Newtons 2. lov i $y$-retning:

\begin{solution}
\begin{align}
\sum F_{y} &= m a_{y} \quad (a_{y} = 0, \text{ all bevegelse skjer i } x\text{-retning}) \\
F_{n} - F_{g} &= 0 \\
F_{n} &= m g \quad \text{(I)} \\
f &= \mu F_{n} \quad \text{(II)}
\end{align}
Setter (I) inn i (II):
\begin{align}
f &= \mu m g \quad \text{(III)}
\end{align}
Friksjonsarbeidet er gitt ved:
\begin{align}
W_{R} &= f S \cdot \cos 180^{\circ} \quad \text{(IV)}
\end{align}
Setter (III) inn i (IV) og løser for $\mu$:
\begin{align}
W_{R} &= \mu m g S \\
\mu &= -\frac{W_{R}}{m g S} \\
    &= -\frac{\SI{-59}{\joule}}{2.0 \mathrm{~kg} \cdot 9.81 \frac{\mathrm{N}}{\mathrm{kg}} \cdot 9.0 \mathrm{~m}} \\
    &= \doubleunderline{0.33}
\end{align}
Friksjonskoeffisienten mellom blokken og det horisontale underlaget er \doubleunderline{0.33}.
\end{solution}

\end{parts}


\question FILL QUESTION
\begin{parts}

\part Exercise 1: Definer en funksjon med to parametere som returnerer kraften utført av friksjon med en koeffisient $\mu$ på et horisontalt plan, basert på den normale kraften.
\begin{solution}
Friksjonskraften på et horisontalt plan kan beregnes ved hjelp av friksjonskoeffisienten $\mu$ og den normale kraften $N$. Formelen for friksjonskraften $f$ er gitt ved:
\begin{align}
f &= \mu \cdot N
\end{align}
Her er $N$ den normale kraften som kan beregnes som produktet av massen og tyngdeakselerasjonen $g$.
\end{solution}

\part Exercise 2: Lag en funksjon kalt 'kinetic_energy' som returnerer den kinetiske energien. Funksjonen skal ha to parametere.
\begin{solution}
Den kinetiske energien $K$ til en gjenstand med masse $m$ og hastighet $v$ er gitt ved:
\begin{align}
K &= \frac{1}{2} m v^2
\end{align}
Denne formelen uttrykker energien som en gjenstand har på grunn av sin bevegelse.
\end{solution}

\part Exercise 3: Vi vet at en curlingstein med en hastighet på \SI{1.0}{\meter\per\second} reiser nøyaktig \SI{6.4}{\meter} og stopper rett ved sentrum av curlingmålet. Finn friksjonskoeffisienten $\mu$.
\begin{solution}
Vi bruker prinsippet om energibevaring, hvor den mekaniske energien $E_{\text{mech}}$ er lik den kinetiske energien $K$ siden den potensielle energien $U$ ikke endres. Arbeidet utført av ikke-konservative krefter $W_{\text{nonconservative}}$ er relatert til endringen i mekanisk energi:
\begin{align}
\Delta E_{\text{mech}} &= -W_{\text{nonconservative}} \\
K &= \mu \cdot m \cdot g \cdot L
\end{align}
Løsning for $\mu$ gir:
\begin{align}
\mu &= \frac{K}{m \cdot g \cdot L} \\
    &= \frac{\frac{1}{2} \cdot \SI{20}{\kilo\gram} \cdot (\SI{1}{\meter\per\second})^2}{\SI{20}{\kilo\gram} \cdot \SI{9.81}{\meter\per\second\squared} \cdot \SI{6.4}{\meter}} \\
    &= \doubleunderline{0.0080}
\end{align}
Friksjonskoeffisienten $\mu$ er dermed \doubleunderline{0.0080}.
\end{solution}

\part Exercise 4: Plot hastigheten til curlingsteinen mens den beveger seg fra startlinjen med en startfart på \SI{1}{\meter\per\second} til sentrum av målet (ved \SI{6.4}{\meter}).
\begin{solution}
Hastigheten $v$ til curlingsteinen som funksjon av avstanden $L$ kan uttrykkes som:
\begin{align}
v &= \sqrt{2 \cdot K / m - 2 \cdot \mu \cdot g \cdot L}
\end{align}
Her definerer vi en avstandsarray fra \SI{0}{\meter} til \SI{6.4}{\meter} ved bruk av \texttt{np.linspace} med 100 datapunkter, og plottet viser hastigheten som en funksjon av avstanden.
\end{solution}

\end{parts}
\end{questions}
\end{document}
