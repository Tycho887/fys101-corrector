\documentclass[answers,a4paper,12pt]{exam}
\input{preamble.tex}

\title{{\bf{FYS101 Mekanikk}} \\ \Large{\answersareprinted Oblig XX}} 
\author{Institutt for Fysikk, REALTEK}
\date{Uke xx}

\begin{document}
\maketitle

\begin{questions}

\question Beregn krefter og moment for en blokk og trommel:
\begin{parts}

\part Bruker Newtons andre lov (N2L) for blokken i $y$-retning:
\begin{solution}
Siden hastigheten $v$ er konstant, er akselerasjonen $a = 0$. Derfor kan vi skrive:
\begin{align}
\sum F_{y} &= m \cdot a \\
T - F_{g} &= 0 \\
T &= m \cdot g \\
  &= 2000 \, \mathrm{kg} \cdot 9.81 \, \mathrm{m/s^2} \\
  &= \SI{19620}{\newton} \\
  &= \doubleunderline{\SI{19.62}{\kilo\newton}}
\end{align}
\end{solution}

\part Finner kraftmomentet fra tauet på trommelen:
\begin{solution}
\begin{align}
\tau &= T \cdot r \\
     &= \SI{19620}{\newton} \cdot \SI{0.30}{\meter} \\
     &= \SI{5886}{\newton\meter} \\
     &= \doubleunderline{\SI{5.89}{\kilo\newton\meter}}
\end{align}
\end{solution}

\part Vinkelhastigheten til trommelen har samme hastighet som tauet som ligger på overflaten:
\begin{solution}
\begin{align}
v &= \omega \cdot r \\
\omega &= \frac{v}{r} \\
       &= \frac{\SI{0.08}{\meter\per\second}}{\SI{0.30}{\meter}} \\
       &= \SI{0.2667}{\radian\per\second} \\
       &= \doubleunderline{\SI{0.27}{\radian\per\second}}
\end{align}
\end{solution}

\part Beregn effekten $P$:
\begin{solution}
\begin{align}
P &= \tau \cdot \omega \\
  &= \SI{5886}{\newton\meter} \cdot \SI{0.2667}{\radian\per\second} \\
  &= \SI{1569.8}{\watt} \\
  &= \doubleunderline{\SI{1.57}{\kilo\watt}}
\end{align}
\end{solution}

\end{parts}



\question FILL QUESTION
\begin{parts}

\part Del A: Finn et uttrykk for friksjonskraften $f_s$ ved bruk av Newtons andre lov for rotasjon.
\begin{solution}
Vi starter med å bruke summen av momentene $\sum \tau = I \alpha$ hvor $\alpha = \frac{a}{r}$ og treghetsmomentet $I = \frac{1}{2} m r^2$ for en sylindrisk gjenstand (se tabell 9.1 i Tipler). 

\begin{align*}
f_s r &= \frac{1}{2} m r^2 \frac{a}{r} \\
f_s &= \frac{1}{2} m a \tag{I}
\end{align*}
\end{solution}

\part Del B: Finn akselerasjonen til massesenteret til sylinderen ved bruk av Newtons andre lov i $x$-retningen.
\begin{solution}
Vi bruker Newtons andre lov $\sum F_x = m a_x$ hvor $a_x = a$.

\begin{align*}
F_{gx} - f_s &= m a \\
F_g \sin \theta - \frac{1}{2} m a &= m a \\
\frac{3}{2} m a &= m g \sin \theta \\
a &= \frac{2}{3} g \sin \theta
\end{align*}
\end{solution}

\part Del C: Finn et uttrykk for friksjonskraften ved å bruke likning (I) og resultatet fra Del B for akselerasjonen.
\begin{solution}
Ved å sette inn akselerasjonen fra Del B i likning (I):

\begin{align*}
f_s &= \frac{1}{2} m a \\
&= \frac{1}{2} m \cdot \frac{2}{3} g \sin \theta \\
f_s &= \frac{1}{3} m g \sin \theta
\end{align*}
\end{solution}

\part Del D: Finn uttrykket for den maksimale friksjonskraften før flatene begynner å gli mot hverandre.
\begin{solution}
Den maksimale statiske friksjonskraften er gitt ved $f_{s, \max} = \mu_s \cdot F_n$. Vi bruker Newtons andre lov i $y$-retningen for å finne $F_n$.

\begin{align*}
\sum F_y &= m a_y \quad (a_y = 0) \\
F_{gy} - F_n &= 0 \\
F_n &= F_g \cos \theta \\
F_n &= m g \cos \theta
\end{align*}

Dermed får vi uttrykket:

\begin{align*}
f_{s, \max} &= \mu_s m g \cos \theta
\end{align*}
\end{solution}

\part Del E: Finn den maksimale vinkelen $\theta_{\text{max}}$ før sylinderen begynner å gli.
\begin{solution}
Resultatet fra Del C gir friksjonskraften like før sylinderen begynner å gli. Vi setter dette lik $f_{s, \max}$ for å finne maksimal vinkel $\theta_{\text{max}}$.

\begin{align*}
f_s &= f_{s, \max} \\
\frac{1}{3} m g \sin \theta_{\text{max}} &= \mu_s m g \cos \theta_{\text{max}} \\
\frac{1}{3} \tan \theta_{\text{max}} &= \mu_s \\
\tan \theta_{\text{max}} &= 3 \mu_s \\
\theta_{\text{max}} &= \tan^{-1}(3 \mu_s)
\end{align*}
\end{solution}

\end{parts}


\question FILL QUESTION
\begin{parts}

\part Beregn spinnet om sirkelens senter:
\begin{solution}
Spinnet, eller det angulære momentet, kan beregnes ved hjelp av formelen:
\begin{align}
L &= m \cdot v \cdot r \\
  &= \SI{2.0}{\kilo\gram} \cdot \SI{3.5}{\meter\per\second} \cdot \SI{4.0}{\meter} \\
  &= \doubleunderline{\SI{28}{\kilo\gram\meter\squared\per\second}}
\end{align}
Spinnet har en størrelse på \SI{28}{\kilo\gram\meter\squared\per\second} og er vinkelrett på sirkelbanen partikkelen beveger seg langs, med retning vekk fra oss.
\end{solution}

\part Beregn treghetsmomentet til partikkelen om en akse vinkelrett på sirkelbanen, gjennom sirkelbanens senter:
\begin{solution}
Treghetsmomentet \( I \) kan beregnes fra det angulære momentet \( L \) og vinkelhastigheten \( \omega \), hvor \( \omega = \frac{v}{r} \):
\begin{align}
L &= I \cdot \omega \\
I &= \frac{L}{\omega} \\
  &= \frac{L \cdot r}{v} \\
  &= \frac{\SI{28}{\kilo\gram\meter\squared\per\second} \cdot \SI{4.0}{\meter}}{\SI{3.5}{\meter\per\second}} \\
  &= \doubleunderline{\SI{32}{\kilo\gram\meter\squared}}
\end{align}
\end{solution}

\part Beregn vinkelhastigheten:
\begin{solution}
Vinkelhastigheten \( \omega \) kan beregnes som:
\begin{align}
\omega &= \frac{v}{r} \\
       &= \frac{\SI{3.5}{\meter\per\second}}{\SI{4.0}{\meter}} \\
       &= \doubleunderline{\SI{0.88}{\radian\per\second}}
\end{align}
\end{solution}

\end{parts}


\question FILL QUESTION
\begin{parts}

\part Plattformen er friksjonsløs, som betyr at spinnet er bevart:
\begin{solution}
\centering
\begin{align}
L_{1} &= L_{2} \\
I_{1} \cdot w_{1} &= I_{2} \cdot w_{2} \\
w_{2} &= \frac{I_{1}}{I_{2}} \cdot w_{1} \\
      &= \frac{6.0 \, \mathrm{kg \, m}^2}{1.8 \, \mathrm{kg \, m}^2} \cdot \frac{1.5 \, \mathrm{rev}}{3} \\
      &= \frac{6.0}{1.8} \cdot \frac{1.5}{3} \\
      &= 5.0 \, \frac{\mathrm{rev}}{\mathrm{s}}
\end{align}
\end{solution}

\part Beregn endringen i kinetisk energi:
\begin{solution}
\centering
\begin{align}
\Delta K &= K_{2} - K_{1} \\
        &= \frac{1}{2} I_{2} w_{2}^2 - \frac{1}{2} I_{1} w_{1}^2 \\
        &= \frac{1}{2} \cdot 1.8 \, \mathrm{kg \, m}^2 \cdot \left(5.0 \, \frac{\mathrm{rev}}{\mathrm{s}} \cdot \frac{2 \pi \, \mathrm{rad}}{\mathrm{rev}}\right)^2 \\
        &\quad - \frac{1}{2} \cdot 6.0 \, \mathrm{kg \, m}^2 \cdot \left(1.5 \, \frac{\mathrm{rev}}{3} \cdot \frac{2 \pi \, \mathrm{rad}}{\mathrm{rev}}\right)^2 \\
        &= 0.62 \, \mathrm{kJ}
\end{align}
\end{solution}

\part Forklar hvorfor den kinetiske energien endres:
\begin{solution}
Endringen i den kinetiske energien kommer fra at radiusen reduseres når vektene blir dratt mot kroppen. På grunn av bevaring av spinn (plattformen er friksjonsløs), må vinkelfarten øke når radiusen reduseres siden massen er konstant. Kvadreringen av vinkelfarten når endringen i kinetisk energi regnes ut, gjør at endringen i vinkelfart ofte får mest å si for den resulterende kinetiske energien. Vinkelfarten øker, og det gjør også den kinetiske energien.
\end{solution}

\end{parts}

\end{questions}
\end{document}
