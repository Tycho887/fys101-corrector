\documentclass[answers,a4paper,12pt]{exam}
\input{preamble.tex}

\title{{\bf{FYS101 Mekanikk}} \\ \Large{\answersareprinted Oblig XX}} 
\author{Institutt for Fysikk, REALTEK}
\date{Uke xx}

\begin{document}
\maketitle

\begin{questions}
\question Definer viktige konsepter i "Del A":
\begin{parts}
\part Definer refraksjon:
\begin{solution}
Refraksjon er brytning av bølger når de passerer en grenseflate mellom to medier med forskjellig tetthet. Dette fører til at bølgene endrer retning.
\end{solution}

\part Definer diffraksjon:
\begin{solution}
Diffraksjon er et fenomen som oppstår når bølger treffer en hindring eller passerer gjennom en åpning. Den ikke-blokkerte delen vil spres på den andre siden.
\end{solution}

\part Definer overlagring:
\begin{solution}
Overlagring er summen av ulike bølger, noe som resulterer i en resultantbølge som er summen av de individuelle bølgene.
\end{solution}

\part Definer Dopplereffekt:
\begin{solution}
Dopplereffekt er effekten der oppfattet bølgelengde (og dermed frekvens) avhenger av mottakerens posisjon og/eller bevegelse i forhold til bølgekilden.
\end{solution}

\part Definer sjokkbølge:
\begin{solution}
Sjokkbølge oppstår dersom kildehastigheten er større enn bølgehastigheten. Da vil det ikke sendes ut noen bølger foran kilden, men i stedet vil bølgene samles opp bak i form av sjokkbølger.
\end{solution}
\end{parts}

\question Løsningsforslag for oblig 12:
\begin{parts}
\part PLACE GRAPHICS HERE
\begin{solution}
Inkluder grafikk for å illustrere refraksjon og diffraksjon.
\end{solution}
\end{parts}


\question Analyser bølgefunksjonen for en harmonisk bølge:
\begin{parts}

\part Del A: Bestem retningen og farten til bølgen
\begin{solution}
Bølgefunksjonen er gitt ved:
\[
y(x, t) = (1.00 \, \mathrm{mm}) \cdot \sin(62.8 \, \mathrm{m}^{-1} x + 314 \, \mathrm{s}^{-1} t)
\]

For en bølge som beveger seg i positiv $x$-retning, er bølgefunksjonen $y(x, t) = A \sin(kx - \omega t)$. Her ser vi at vår funksjon har formen $y(x, t) = A \sin(kx + \omega t)$, som indikerer at bølgen beveger seg i negativ $x$-retning.

Farten til bølgen er gitt ved:
\begin{align}
v &= \frac{\omega}{k} = \frac{314 \, \mathrm{s}^{-1}}{62.8 \, \mathrm{m}^{-1}} \\
v &= \doubleunderline{\SI{5.00}{\meter\per\second}}
\end{align}
Bølgen beveger seg med en fart på \SI{5.00}{\meter\per\second}.
\end{solution}

\part Del B: Beregn bølgelengde, frekvens og periode
\begin{solution}
Bølgelengden $\lambda$ er gitt ved:
\begin{align}
\lambda &= \frac{2\pi}{k} = \frac{2\pi}{62.8 \, \mathrm{m}^{-1}} \\
\lambda &= \doubleunderline{\SI{0.10}{\meter}}
\end{align}
Bølgelengden er \SI{10.0}{\centi\meter}.

Frekvensen $f$ er gitt ved:
\begin{align}
f &= \frac{\omega}{2\pi} = \frac{314 \, \mathrm{s}^{-1}}{2\pi} \\
f &= \doubleunderline{\SI{50.0}{\hertz}}
\end{align}
Frekvensen er \SI{50.0}{\hertz}.

Perioden $T$ er gitt ved:
\begin{align}
T &= \frac{1}{f} = \frac{1}{\SI{50.0}{\hertz}} \\
T &= \doubleunderline{\SI{0.0200}{\second}}
\end{align}
Perioden er \SI{0.0200}{\second}.
\end{solution}

\part Del C: Finn maksimalfarten til partikkelen
\begin{solution}
Maksimalfarten $v_{\max}$ til partikkelen er gitt ved å derivere posisjonen med hensyn på tid:
\begin{align}
v_{\max} &= A \omega \\
&= (1.00 \times 10^{-3} \, \mathrm{m}) \cdot 314 \, \mathrm{s}^{-1} \\
v_{\max} &= \doubleunderline{\SI{0.314}{\meter\per\second}}
\end{align}
Maksimalfarten langs bølgen er \SI{0.314}{\meter\per\second}.
\end{solution}

\end{parts}



\question Beregn effekten levert fra en lydkilde og avstanden der lydintensiteten endres:
\begin{parts}

\part Beregn avstanden der lydintensiteten endres:
\begin{solution}
Effekten levert fra kilden er uavhengig av posisjonens lydintensitet, altså, effekten levert er den samme. Vi bruker derfor at
\begin{align}
P_{1} &= P_{2}, \quad \text{og at} \\
I &= \frac{P_{av}}{A}.
\end{align}

Dette gir oss
\begin{align}
I_{1} A_{1} &= I_{2} A_{2} \quad \left(A = 4 \pi r^{2}\right \text{ for en sfærisk overflate}) \\
I_{1} 4 \pi r_{1}^{2} &= I_{2} 4 \pi r_{2}^{2}.
\end{align}

Ved å løse for \( r_{2} \), får vi
\begin{align}
r_{2} &= r_{1} \sqrt{\frac{I_{1}}{I_{2}}} \\
      &= \SI{10.0}{\meter} \sqrt{\frac{1.00 \cdot 10^{-4} \frac{\mathrm{W}}{\mathrm{m}^{2}}}{1.00 \cdot 10^{-6} \frac{\mathrm{W}}{\mathrm{m}^{2}}}} \\
      &= \doubleunderline{\SI{100}{\meter}}.
\end{align}

Ved en avstand på \SI{100}{\meter} vil lydintensiteten være \SI{1.00e-6}{\watt\per\meter\squared}.
\end{solution}

\part Beregn effekten avgitt av kilden:
\begin{solution}
Effekten avgitt av kilden kan beregnes ved
\begin{align}
P_{av} &= 4 \pi r_{1}^{2} I_{1} \\
       &= 4 \pi (\SI{10.0}{\meter})^{2} \cdot \SI{1.00e-4}{\watt\per\meter\squared} \\
       &= \doubleunderline{\SI{126}{\milli\watt}}.
\end{align}

Effekten levert av kilden er \SI{126}{\milli\watt}.
\end{solution}

\end{parts}


\question Beregn reduksjonen i levert effekt fra en lydkilde når lydintensitetsnivået reduseres fra \SI{90}{\decibel} til \SI{70}{\decibel}.

\begin{parts}

\part Beregn forskjellen i lydintensitetsnivå:
\begin{solution}
Lydintensitetsnivået er gitt ved formelen:
\[
\beta = (10 \, \mathrm{dB}) \log \left(\frac{I}{I_{0}}\right)
\]

Forskjellen i lydintensitetsnivå er:
\begin{align}
\Delta \beta &= \beta_{1} - \beta_{2} \\
             &= \SI{90}{\decibel} - \SI{70}{\decibel} \\
             &= \SI{20}{\decibel}
\end{align}
\end{solution}

\part Beregn forholdet mellom intensitetene:
\begin{solution}
Ved å bruke formelen for lydintensitetsnivå, kan vi skrive:
\begin{align}
\Delta \beta &= (10 \, \mathrm{dB}) \cdot \log \left(\frac{I_{90}}{I_{70}}\right) \\
\SI{20}{\decibel} &= (10 \, \mathrm{dB}) \cdot \log \left(\frac{I_{90}}{I_{70}}\right)
\end{align}

Løsning for intensitetsforholdet:
\begin{align}
\log \left(\frac{I_{90}}{I_{70}}\right) &= 2 \\
\frac{I_{90}}{I_{70}} &= 10^{2} \\
I_{90} &= 100 \cdot I_{70}
\end{align}
\end{solution}

\part Beregn prosentvis reduksjon i levert effekt:
\begin{solution}
Reduksjonen i intensitet kan uttrykkes som:
\begin{align}
\frac{I_{90} - I_{70}}{I_{90}} &= \frac{100 \cdot I_{70} - I_{70}}{100 \cdot I_{70}} \\
                               &= \frac{99 \cdot I_{70}}{100 \cdot I_{70}} \\
                               &= 0.99
\end{align}

Dette tilsvarer en \doubleunderline{\SI{99}{\percent}} reduksjon i levert effekt fra lydkilden.
\end{solution}

\end{parts}


\question Bruk Dopplereffekten for å beregne frekvensene mottatt av regndråpene:
\begin{parts}

\part Forklar Dopplereffekten:
\begin{solution}
Dopplereffekten beskriver endringen i frekvens eller bølgelengde av en bølge i forhold til en observatør som beveger seg i forhold til bølgekilden. Når kilden nærmer seg observatøren, øker frekvensen, og når den beveger seg bort, reduseres frekvensen.
\end{solution}

\part Beregn den mottatte frekvensen $f_{r}$:
\begin{solution}
\centering
\begin{align}
f_{r} &= \frac{c + u_{r}}{c} \cdot f_{s} \\
&= \frac{3.00 \times 10^8 \, \text{m/s} + u_{r}}{3.00 \times 10^8 \, \text{m/s}} \cdot 625 \times 10^6 \, \text{Hz}
\end{align}
\end{solution}

\part Beregn den reflekterte frekvensen $f_{r}^{\prime}$:
\begin{solution}
\centering
\begin{align}
f_{r}^{\prime} &= \frac{c}{c - u_{s}} \cdot f_{r} \\
&= \frac{3.00 \times 10^8 \, \text{m/s}}{3.00 \times 10^8 \, \text{m/s} - u_{s}} \cdot f_{r}
\end{align}
\end{solution}

\part Sett likning (I) inn i likning (II) og forenkle:
\begin{solution}
\centering
\begin{align}
f_{r}^{\prime} &= \left(\frac{c}{c-u_{s}}\right)\left(\frac{c+u_{r}}{c}\right) f_{s} \\
&= \frac{c+u_{r}}{c-u_{s}} \cdot f_{s} \\
f_{r}^{\prime} &= \left(1+\frac{u_{r}}{c}\right)\left(1-\frac{u_{s}}{c}\right)^{-1} f_{s}
\end{align}
\end{solution}

\part Beregn frekvensforskjellen $\Delta f$:
\begin{solution}
\centering
\begin{align}
\Delta f &= f_{r}^{\prime} - f_{s} \\
&= \left(1 + \frac{2u_{r}}{c}\right)f_{s} - f_{s} \\
&= \frac{2u_{r}}{c} \cdot f_{s}
\end{align}
\end{solution}

\part Beregn hastigheten til regndråpene $u_{r}$:
\begin{solution}
\centering
\begin{align}
u_{r} &= \frac{c}{2f_{s}} \Delta f \\
&= \frac{3.00 \times 10^8 \, \text{m/s} \cdot 325 \, \text{Hz}}{2 \cdot 625 \times 10^6 \, \text{Hz}} \\
&= \doubleunderline{\SI{78.0}{\meter\per\second}}
\end{align}
\end{solution}

\end{parts}

\end{questions}
\end{document}
