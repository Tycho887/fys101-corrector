\documentclass[answers,a4paper,12pt]{exam}
\input{preamble.tex}

\title{{\bf{FYS101 Mekanikk}} \\ \Large{\answersareprinted Oblig XX}} 
\author{Institutt for Fysikk, REALTEK}
\date{Uke xx}

\begin{document}
\maketitle

\begin{questions}
\question FILL QUESTION
\begin{parts}

\part Forklar refraksjon:
\begin{solution}
Refraksjon er brytning av stråler eller bølger når de passerer en grenseflate mellom to medier med forskjellig tetthet. Dette fører til at bølgene endrer retning.
\end{solution}

\part Forklar diffraksjon:
\begin{solution}
Diffraksjon er et fenomen som oppstår når bølger treffer en hindring eller passerer gjennom en åpning. Den ikke-blokkerte delen vil bøyes på den andre siden, noe som resulterer i en spredning av bølgene.
\end{solution}

\part Forklar overlagring:
\begin{solution}
Overlagring er summen av ulike bølger, som resulterer i en resultantbølge som er summen av de individuelle bølgene. Dette kan føre til konstruktiv eller destruktiv interferens avhengig av bølgenes fase.
\end{solution}

\part Forklar Dopplereffekten:
\begin{solution}
Dopplereffekten er en effekt der den observerte bølgelengden (og dermed frekvensen) avhenger av mottakerens posisjon og/eller bevegelse i forhold til bølgekilden. Dette kan føre til at lyden oppfattes som høyere eller lavere i frekvens avhengig av om kilden beveger seg mot eller fra observatøren.
\end{solution}

\part Forklar sjokkbølge:
\begin{solution}
Sjokkbølge oppstår dersom kildehastigheten er større enn bølgehastigheten. Da vil det ikke sendes ut noen bølger foran kilden, i stedet vil bølgene samles opp bak i form av sjokkbølger.
\end{solution}

\end{parts}

\question Løsningsforslag for oblig 12:
\begin{solution}
PLACE GRAPHICS HERE
\end{solution}


\question Analyser bølgefunksjonen for en harmonisk bølge:
\begin{parts}

\part Del A: Bestem retningen og farten til bølgen
\begin{solution}
Bølgefunksjonen er gitt ved:
\[
y(x, t) = (1.00 \, \text{mm}) \cdot \sin(62.8 \, \text{m}^{-1} x + 314 \, \text{s}^{-1} t)
\]

\begin{itemize}
  \item Funksjonen $y(x, t) = A \sin(kx + \omega t)$ beskriver en bølge som beveger seg i negativ $x$-retning.
  \item Derfor beveger den oppgitte bølgefunksjonen seg i negativ $x$-retning.
  \item Farten til bølgen er gitt ved:
\end{itemize}

\begin{align}
v &= \frac{\omega}{k} = \frac{314 \, \text{s}^{-1}}{62.8 \, \text{m}^{-1}} \\
v &= \doubleunderline{\SI{5.00}{\meter\per\second}}
\end{align}

Bølgen beveger seg med en fart på \SI{5.00}{\meter\per\second}.
\end{solution}

\part Del B: Bestem bølgelengde, frekvens og periode
\begin{solution}
\begin{itemize}
  \item Bølgelengden er gitt ved:
\end{itemize}

\begin{align}
\lambda &= \frac{2\pi}{k} = \frac{2\pi}{62.8 \, \text{m}^{-1}} \\
\lambda &= \doubleunderline{\SI{0.10}{\meter}}
\end{align}

Bølgelengden er \SI{10.0}{\centi\meter}.

\begin{itemize}
  \item Frekvensen til bølgen er gitt ved:
\end{itemize}

\begin{align}
f &= \frac{\omega}{2\pi} = \frac{314 \, \text{s}^{-1}}{2\pi} \\
f &= \doubleunderline{\SI{50.0}{\hertz}}
\end{align}

Frekvensen er \SI{50.0}{\hertz}.

\begin{itemize}
  \item Perioden er gitt ved:
\end{itemize}

\begin{align}
T &= \frac{1}{f} = \frac{1}{\SI{50.0}{\hertz}} \\
T &= \doubleunderline{\SI{0.0200}{\second}}
\end{align}

Perioden er \SI{0.0200}{\second}.
\end{solution}

\part Del C: Bestem maksimalfarten langs bølgen
\begin{solution}
For å finne maksimalfarten, deriverer vi posisjonen med hensyn til tid:

\begin{align}
v_{\max} &= A \omega \\
v_{\max} &= 1.00 \times 10^{-3} \, \text{m} \cdot 314 \, \text{s}^{-1} \\
v_{\max} &= \doubleunderline{\SI{0.314}{\meter\per\second}}
\end{align}

Maksimalfarten langs bølgen er \SI{0.314}{\meter\per\second}.
\end{solution}

\end{parts}



\question Beregn effekten levert fra en kilde og avstanden hvor lydintensiteten endres:
\begin{parts}

\part Del A: Beregn avstanden $r_2$ hvor lydintensiteten er $I_2$:
\begin{solution}
Effekten levert fra kilden er uavhengig av posisjonens lydintensitet, altså er effekten levert den samme. Vi bruker derfor at:
\begin{align}
P_1 &= P_2, \quad \text{og at} \\
I &= \frac{P_{\text{av}}}{A}.
\end{align}

Dette gir oss:
\begin{align}
I_1 A_1 &= I_2 A_2 \quad \left(A = 4 \pi r^2 \text{ for en sfærisk overflate}\right).
\end{align}

Ved å sette inn verdiene får vi:
\begin{align}
I_1 \cdot 4 \pi r_1^2 &= I_2 \cdot 4 \pi r_2^2 \\
r_2 &= r_1 \sqrt{\frac{I_1}{I_2}} \\
r_2 &= \SI{10.0}{\meter} \sqrt{\frac{1.00 \cdot 10^{-4} \, \frac{\mathrm{W}}{\mathrm{m}^2}}{1.00 \cdot 10^{-6} \, \frac{\mathrm{W}}{\mathrm{m}^2}}} \\
r_2 &= \SI{100}{\meter}.
\end{align}

Ved en avstand på \SI{100}{\meter} vil lydintensiteten være $1.00 \cdot 10^{-6} \, \frac{\mathrm{W}}{\mathrm{m}^2}$.
\end{solution}

\part Del B: Beregn effekten avgitt av kilden:
\begin{solution}
Effekten avgitt av kilden kan beregnes ved:
\begin{align}
P_{\text{av}} &= 4 \pi r_1^2 I_1 \\
&= 4 \pi (\SI{10.0}{\meter})^2 \cdot 1.00 \cdot 10^{-4} \, \frac{\mathrm{W}}{\mathrm{m}^2} \\
P_{\text{av}} &= \SI{126}{\milli\watt}.
\end{align}

Effekten levert av kilden er \doubleunderline{\SI{126}{\milli\watt}}.
\end{solution}

\end{parts}



\question Beregn reduksjonen i levert effekt fra en lydkilde:
\begin{parts}
\part Gitt data for lydintensitetsnivåer:
\begin{solution}
\centering
\begin{align}
\beta_{1} &= \SI{90}{\decibel} \\
\beta_{2} &= \SI{70}{\decibel}
\end{align}
\end{solution}

\part Forklar lydintensitetsnivå:
\begin{solution}
Lydintensitetsnivået, \(\beta\), er et mål på lydens styrke og er gitt ved formelen:
\[
\beta = (10 \, \mathrm{dB}) \log \left(\frac{I}{I_{0}}\right)
\]
hvor \(I\) er lydintensiteten og \(I_{0}\) er referanseintensiteten.
\end{solution}

\part Beregn differansen i lydintensitetsnivå for å finne andelen av effekten som må reduseres:
\begin{solution}
\centering
\begin{align}
\Delta \beta &= \beta_{1} - \beta_{2} \\
             &= \SI{20}{\decibel}
\end{align}
\end{solution}

\part Beregn forholdet mellom intensitetene:
\begin{solution}
\centering
\begin{align}
\Delta \beta &= (10 \, \mathrm{dB}) \cdot \log \left(\frac{I_{90}}{I_{70}}\right) \\
\SI{20}{\decibel} &= (10 \, \mathrm{dB}) \cdot \log \left(\frac{I_{90}}{I_{70}}\right) \\
2 &= \log \left(\frac{I_{90}}{I_{70}}\right) \\
I_{90} &= 10^{2} \cdot I_{70} \\
I_{90} &= 100 \cdot I_{70}
\end{align}
\end{solution}

\part Beregn reduksjonen i levert effekt:
\begin{solution}
\centering
\begin{align}
\frac{I_{90} - I_{70}}{I_{90}} &= \frac{100 \cdot I_{70} - I_{70}}{100 \cdot I_{70}} \\
                               &= \frac{99 \cdot I_{70}}{100 \cdot I_{70}} \\
                               &= 0.99 \\
                               &= \doubleunderline{99\%}
\end{align}
\end{solution}
\end{parts}




\question Bruk Dopplereffekten til å beregne frekvensen mottatt av regndråpene:
\begin{parts}
\part Del A: Forklar Dopplereffekten.
\begin{solution}
Dopplereffekten er fenomenet der frekvensen av en bølge endres for en observatør som beveger seg i forhold til bølgekilden. Når kilden og observatøren nærmer seg hverandre, øker frekvensen, og når de beveger seg fra hverandre, reduseres frekvensen.
\end{solution}

\part Del B: Beregn mottatt frekvens ved bruk av Dopplereffekten.
\begin{solution}
Vi har følgende data:
\begin{align*}
f_{s} &= \SI{625}{\mega\hertz} \\
d &= \SI{50}{\kilo\meter} \\
\Delta f &= \SI{325}{\hertz}
\end{align*}

Bruker likningen for Dopplereffekten:
\begin{align}
f_{r} &= \frac{c + u_{r}}{c} \cdot f_{s} \tag{I}
\end{align}

Reflektert frekvens fra dråpene:
\begin{align}
f_{r}^{\prime} &= \frac{c}{c - u_{s}} \cdot f_{r} \tag{II}
\end{align}

Setter likning (I) inn i likning (II):
\begin{align}
f_{r}^{\prime} &= \left(\frac{c}{c-u_{s}}\right)\left(\frac{c+u_{r}}{c}\right) f_{s} \\
&= \frac{c+u_{r}}{c-u_{s}} \cdot f_{s} \\
f_{r}^{\prime} &= \left(1+\frac{u_{r}}{c}\right)\left(1-\frac{u_{s}}{c}\right)^{-1} f_{s}
\end{align}

Siden $u_{r} \ll c$ og $u_{s} \ll c$, kan vi tilnærme:
\begin{align}
1-\frac{u_{s}}{c} &\approx 1-\frac{u_{r}}{c} \\
\left(1-\frac{u_{r}}{c}\right)^{-1} &\approx 1+\frac{u_{r}}{c}
\end{align}

Dermed:
\begin{align}
f_{r}^{\prime} &= \left(1+\frac{2 u_{r}}{c}\right) f_{s} \\
\Delta f &= f_{r}^{\prime} - f_{s} \\
\Delta f &= \left(1+\frac{2 u_{r}}{c}\right) f_{s} - f_{s}
\end{align}

Løsning for $u_{r}$:
\begin{align}
u_{r} &= \frac{c}{2 f_{s}} \Delta f \\
&= \frac{\SI{3.00e8}{\meter\per\second} \cdot \SI{325}{\hertz}}{2 \cdot \SI{625e6}{\hertz}} \\
&= \doubleunderline{\SI{78.0}{\meter\per\second}}
\end{align}

Farten til vinden er \SI{78.0}{\meter\per\second}.
\end{solution}
\end{parts}

\end{questions}
\end{document}
