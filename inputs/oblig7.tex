\documentclass[10pt]{article}
\usepackage[norsk]{babel}
\usepackage[utf8]{inputenc}
\usepackage[T1]{fontenc}
\usepackage{graphicx}
\usepackage[export]{adjustbox}
\graphicspath{ {./images/} }
\usepackage{amsmath}
\usepackage{amsfonts}
\usepackage{amssymb}
\usepackage[version=4]{mhchem}
\usepackage{stmaryrd}

\begin{document}

MARKPROB

\begin{center}
\includegraphics[max width=\textwidth]{2025_01_16_5a36e17822561bf0a616g-1}
\end{center}

$$
\begin{aligned}
& m=2000 \mathrm{~kg} \\
& v=8.0 \frac{\mathrm{~cm}}{3} \\
& r=30 \mathrm{~cm}
\end{aligned}
$$

a). Bruker N2L for blokken: $y$-rening:

$$
\begin{aligned}
& \sum F_{y}=m a \quad(a=0 \text { siden } v \text { er konstant }) \\
& T-F_{g}=0 . \\
& T=m g \\
& =2000 \mathrm{~kg} \cdot 9.81 \frac{\mathrm{Ng}}{\mathrm{~g}}=19.62 \mathrm{kN} \\
& T=19.6 \mathrm{kN}
\end{aligned}
$$

b). Finner kraftmomentet fra tauet pá trommeler

$$
\begin{aligned}
\tau & =T \cdot r \\
& =19.62 \cdot 10^{3} \mathrm{~N} \cdot 30 \cdot 10^{-2} \mathrm{~m}=5.886 \mathrm{kNm} \\
\tau & =5.9 \mathrm{kNm}
\end{aligned}
$$

c). Vinkelfesta til trommelen has samme fast som tavet som ligger pa overflater

$$
\begin{aligned}
u & =\omega \cdot r \\
\omega & =\frac{u}{r} \\
& =\frac{8.0 \cdot 10^{-2} \frac{\mathrm{~m}}{\mathrm{~s}}}{30 \cdot 10^{-2} \mathrm{~m}}=0.2667 \frac{\mathrm{rod}}{\mathrm{~s}} \\
\omega & =0.27 \frac{\mathrm{cad}}{\mathrm{~s}}
\end{aligned}
$$

d)

$$
\begin{aligned}
P & =\tau \omega \\
& =5.886 \cdot 10^{3} \mathrm{Nm} \cdot 0.2667 \mathrm{cad} \\
P & =1.6 \mathrm{kD}
\end{aligned}
$$

MARKPROB

9.91)\\
\includegraphics[max width=\textwidth, center]{2025_01_16_5a36e17822561bf0a616g-2}

$$
\begin{aligned}
& R=M_{s} N \\
& G x=G \sin \theta \\
& G y=G \cos \theta \\
& G=m g
\end{aligned}
$$

a) Den ereste kratta som har er arm somkar bidra til kraftmement er friksjonskrafta Rs,

\begin{itemize}
  \item Bruker dette, og N2L for rotajjon til à finne et uttryck for fos
\end{itemize}


\begin{align*}
& \sum \tau=I \alpha\left(\alpha=\frac{a}{r}, I=\frac{1}{2} m r^{2}(\text { tabell } 9.1, \text { Tipler) })\right. \\
& \left.f_{5} r=\frac{1}{2} m r^{2} \frac{a}{r} \right\rvert\, \frac{1}{r} \\
& f_{10}=\frac{1}{2} m a \text { (I) } \tag{I}
\end{align*}


\begin{itemize}
  \item Bruker N2L ; $x$-retning for à finne akselerajjones til massesenteret til sylinderen
\end{itemize}

$$
\begin{aligned}
& \sum F_{x}=m a_{x}\left(a_{x}=a\right) \\
& F_{g x}-f_{5}=m a(\text { setter inn }(I)) \\
& F_{g} \sin \theta-\frac{1}{2} m a=m a \\
& \left.\frac{3}{2} m a=m g \sin \theta \right\rvert\, \frac{1}{m} \\
& a=\frac{2}{3} g \sin \theta
\end{aligned}
$$

b) Finner et uHtijkk for friksjonskafta ved a benyte likning (I) og sette inn resultatet fra a) for cikselerasjonen

$$
\begin{aligned}
f_{5} & =\frac{1}{2} m a \\
& =\frac{1}{2} m \cdot \frac{x}{3} g \sin \theta \\
f & =\frac{1}{3} m g \sin \theta
\end{aligned}
$$

c) Mä finne et uttryke for den maksinale friksjonsleratten der statise friksjonen kan ha for flatene begynner a. gli mot hverandre. Friksjonsterafter for durne betingelsen er gitt ved

$$
f_{5, \max }=\mu_{0} \cdot F_{n}
$$

\begin{itemize}
  \item Bruker N2L i y-retring for a finne et uttrykk for $F_{n}$
\end{itemize}

$$
\begin{aligned}
& \sum F_{y}=\operatorname{may}\left(a_{y}=0\right) \\
& F_{g y}-F_{n}=0 \\
& F_{n}=F_{g} \cos \theta \\
& F_{n}=m g \cos \theta
\end{aligned}
$$

og fär urtykket:

$$
f_{s, m a x}=\mu_{s} m g \cos \theta
$$

\begin{itemize}
  \item Resultatet fra b) gir friksjonstrafter like for sylinderen pegynner à gli, og setter derfor dunne lik fo, max for a finne maksimal vinkel $\theta=\theta_{\text {max }}$
\end{itemize}

$$
\begin{aligned}
& f_{s}=f_{s}, \text { max } \\
& \left.\frac{1}{3} m g \sin \theta_{\text {max }}=\mu_{s} m g \cos \theta_{\text {max }} \right\rvert\, \frac{1}{m g \cos \theta_{\text {max }}} \\
& \frac{1}{3} \tan \theta_{\text {max }}=\mu_{s} \\
& \tan \theta_{\text {max }}=3 \mu_{s} \\
& \theta_{\text {max }}=\tan ^{-1}\left(3 \mu_{s}\right)
\end{aligned}
$$

MARKPROB

\begin{center}
\includegraphics[max width=\textwidth]{2025_01_16_5a36e17822561bf0a616g-4}
\end{center}

$$
\begin{aligned}
m & =2.0 \mathrm{~kg} \\
v & =3.5 \mathrm{mg} \\
r & =4.0 \mathrm{~m}
\end{aligned}
$$

a) Spinn am sirkelers senter

$$
\begin{aligned}
L & =m v r \\
& =2.0 \mathrm{~kg} \cdot 3.5 \frac{\mathrm{~m}}{3} \cdot 4.0 \mathrm{~m} \\
L & =28 \mathrm{~kg} \frac{\mathrm{~m}^{2}}{3}
\end{aligned}
$$

Spinnet har en starrelse pà $20 \mathrm{~kg} \frac{\mathrm{~m}^{2}}{\mathrm{~s}}$ vinkulret pa sirkelbanen partikkeler bevegerseg kngs med retning velk fra oss.\\
b) Treghetsmomentet til partikklen om en akse vinkelret pà sirkelbarer, gjernom virkelbanens senter

$$
\begin{aligned}
L & =I w \quad\left(w=\frac{v}{r}\right) \\
L & =I \frac{v}{r} \\
I & =\frac{L r}{v} \\
& =\frac{28 \mathrm{~kg} \frac{\mathrm{~m}^{2}}{5} \cdot 4.0 \mathrm{~m}}{3.5} \mathrm{~s}^{\mathrm{s}} \\
I & =32 \mathrm{~kg} \mathrm{~m}^{2}
\end{aligned}
$$

c)

$$
\begin{aligned}
w & =\frac{v}{r} \\
& =\frac{3.5 \frac{\mathrm{~m}}{\mathrm{~s}}}{4 \mathrm{~m}} \\
w & =0.88 \frac{\mathrm{rad}}{\mathrm{~s}}
\end{aligned}
$$

MARKPROB

10.49)\\
\includegraphics[max width=\textwidth, center]{2025_01_16_5a36e17822561bf0a616g-5}

$$
\begin{aligned}
& w_{1}=1.5 \frac{\mathrm{rev}}{3} \\
& I_{1}=6.0 \mathrm{~kg} \mathrm{~m}^{2} \\
& I_{2}=1.8 \mathrm{~kg} \mathrm{~m}^{2}
\end{aligned}
$$

a) Platformer er friksjonolcost, som vil si at spinnet er bevart

$$
\begin{array}{rl}
L_{1} & =L_{2} \\
I_{1} w_{1} & =I_{2} w_{2} \\
w_{2} & =\frac{I_{1}}{I_{2}} w_{1} \\
& =\frac{6.0 \mathrm{~kg} \mathrm{~m}}{} \\
1.8 \mathrm{kgm}^{2} & 1.5 \frac{\mathrm{rev}}{3} \\
w_{2} & =5.0 \frac{\mathrm{rg}}{3}
\end{array}
$$

vinkel farter etter at vectene er dratt in mot kopper er $5.0 \frac{\mathrm{rev}}{\mathrm{s}}$\\
b)

$$
\begin{aligned}
\Delta K & =K_{2}-K_{1} \\
& =\frac{1}{2} I_{2} w_{2}-\frac{1}{2} I_{1} w_{1} \\
& =\frac{1}{2} 1.8 \mathrm{kgm}{ }^{2} \cdot\left(5.0 \frac{\mathrm{rgv}}{3} \frac{2 \pi \mathrm{rad}}{\mathrm{rev}}\right)^{2}-\frac{1}{2} 6.0 \mathrm{kgm}^{2}\left(1.8 \frac{\mathrm{rev}}{5} \frac{2 \mathrm{rrew}}{} \mathrm{rev}^{2}\right) \\
\Delta K & =0.62 \mathrm{~kJ}
\end{aligned}
$$

C) Dlaningen i der kinetiske erergien konmer fra at radiusen reduseres när veltere blir dratt mot tropper.\\
Pa grunn aur bevaring our spim (platformen er friksjonslas) má vinkeffarter dee nair radiusen reduseres siden massen er konstant.

Kuadrering au vinculfarter nar endringen i kinetisk evergi regnes ut, gjor at endring i vinkelfart ofte fär mest à s' for din resulterende kinetisice energien\\
Vinkelfarten uker, og det gjor ogsá den kipetiske energien


\end{document}