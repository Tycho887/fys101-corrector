\documentclass[10pt]{article}
\usepackage[norsk]{babel}
\usepackage[utf8]{inputenc}
\usepackage[T1]{fontenc}
\usepackage{graphicx}
\usepackage[export]{adjustbox}
\graphicspath{ {./images/} }
\usepackage{amsmath}
\usepackage{amsfonts}
\usepackage{amssymb}
\usepackage[version=4]{mhchem}
\usepackage{stmaryrd}

\begin{document}
Losning oblig 12-FYS101\\

MARKPROB

Refraksjon er brything /awbaying our stralerlbolger scm passerer en grenseflate\\
\includegraphics[max width=\textwidth, center]{2025_01_16_b7ece2a39f2eaf32433fg-1(1)}

Diffralesjer er et fencmen som oppstös nár bolger treffer en hindring eccer passerer en diphing.\\
Den ithe-blokkerte delen vil ourbages pä den andre siden\\
\includegraphics[max width=\textwidth, center]{2025_01_16_b7ece2a39f2eaf32433fg-1}\\
overlagring er summeing our ulike bolger - detle resultereri en resultantbage sam er summen ar in divaluelle bager.

Dopplereffekt - effekt der applerd bokg lengde (og dermed freluens) ourhenger our mottahers posisjon or ev-bevegelse relatiot til bodgekilde.\\
sjoklebdge oppstä dersom kildehastighet er storre enn bolgehasinghet. Da il det illke sendes ut noen bolger foran kilden isteder lil bolgene sames opp bak i form our syollebdger

Losningsforslag oblig 12\\

MARKPROB

15.41) Bolgefunksjonen for en harmonisk bolge er gitt ved

$$
y(x, t)=(1.00 \mathrm{~mm}) \cdot \sin \left(62.8 \mathrm{~m}^{-1} x+314 \mathrm{~s}^{-1} t\right)
$$

a). $y(x, t)=A \sin (k x-w t)$ beskiver en bolge som beveger seg i positiv $x$-retning

\begin{itemize}
  \item $y(x, t)=A \sin (k x+w t)$ beskriver en bolge som beveger seg ; negativ $x$-retning.
  \item Ser derfor at der oppgitte bolgefunkjonen beskriver er partikkel som beveger seg: negativ $x$-retning.
  \item Farten til bolgen er gitt ved
\end{itemize}

$$
\begin{aligned}
& v=\frac{w}{k}=\frac{314 \mathrm{~s}^{-1}}{62.8 \mathrm{~m}^{-1}} \\
& v=5.00 \mathrm{~m}
\end{aligned}
$$

partikkelen beveger seg med en fart pä 5.00 3\\
b)- Bolgelengder er gitt ved

$$
\begin{aligned}
& \lambda=\frac{2 \pi}{k}=\frac{2 \pi}{62.8 \mathrm{~m}^{-1}} \\
& \lambda=10.0 \mathrm{~cm}
\end{aligned}
$$

bolgelengder er 10.0 cm

\begin{itemize}
  \item Frekuenser til bolgen er gitt ved
\end{itemize}

$$
\begin{aligned}
& f=\frac{w}{2 \pi}=\frac{314 \mathrm{~Hz}}{2 \pi} \\
& f=50.0 \mathrm{~Hz}
\end{aligned}
$$

frekuensen er 50.0 Hz

\begin{itemize}
  \item Perioder er gitt ved
\end{itemize}

$$
\begin{aligned}
& T=\frac{1}{f}=\frac{1}{50.0 \mathrm{~Hz}} \\
& T=0.0200 \mathrm{~S}
\end{aligned}
$$

perioden er 0.0200 s\\
c) Finner maksimalforta ved $\dot{a}$ derivere posisjonen mhp tid og finner

$$
\begin{aligned}
v_{\max } & =A w \\
& =1.00 \cdot 10^{-3} \mathrm{~m} \cdot 314 \mathrm{~s}^{-1} \\
v_{\max } & =0.314 \frac{\mathrm{~m}}{\mathrm{~s}}
\end{aligned}
$$

maksimalfarta langs bolgen er $0.314 \frac{\mathrm{3}}{3}$\\

MARKPROB

15.54)\\
\includegraphics[max width=\textwidth, center]{2025_01_16_b7ece2a39f2eaf32433fg-4}

$$
\begin{aligned}
& r_{1}=10.0 \mathrm{~m} \\
& I_{1}=1.00 \cdot 10^{-4} \frac{\mathrm{x}}{\mathrm{~N}^{2}} \\
& I_{2}=1.00 \cdot 10^{-6} \frac{\mathrm{~m}^{2}}{\mathrm{~m}^{2}}
\end{aligned}
$$

a) Effekten levert fra kilden (s) er uauhengig ou posisjonens lydintensitet, altsä, eflekt levert er det samme. Bruker dermed at

$$
\begin{aligned}
& P_{1}=P_{2}, \quad \text { og at } \\
& I=\frac{P_{a v}}{A} .
\end{aligned}
$$

Dette gir oss\\
$I_{1} A_{1}=I_{2} A_{2} \quad\left(A=4 \pi r^{2}\right.$ for en sfarisk overflate)

$$
\begin{aligned}
& I_{1} 4 \pi r_{1}^{2}=I_{2} 4 \pi r_{2}^{2} \\
& r_{2}=r_{1} \sqrt{\frac{I_{1}}{I_{2}}}=10.0 \mathrm{~m} \sqrt{\frac{1}{4 \pi}} \frac{1.00 \cdot 10^{-4} \frac{y x}{\mathrm{~m}^{2}}}{1.00 \cdot 10^{-6} \frac{\mathrm{~m}}{\mathrm{~m}^{2}}} \\
& r_{2}=100 \mathrm{~m}
\end{aligned}
$$

ved en austand pä 100 m vil lydintensiteter vere $1.00 \cdot 10^{-6} \frac{1 \mathrm{~m}^{2}}{\mathrm{~m}^{2}}$\\
b) Finner effekt augitt ved

$$
\begin{aligned}
P_{a v} & =4 \pi r_{1}^{2} I_{1} \\
& =4 \pi(10.0 \mathrm{~m})^{2} \cdot 1.00 \cdot 10^{-4} \frac{\mathrm{w}}{\mathrm{~m}^{2}} \\
P_{a v} & =126 \mathrm{mw}
\end{aligned}
$$

effelcten levert au kilden er 126 mW\\

MARKPROB

15.60)

$$
\begin{aligned}
& \beta_{1}=90 \mathrm{~dB} \\
& \beta_{2}=70 \mathrm{~dB}
\end{aligned}
$$

Lydintensitetsnivà er gitt ved

$$
\beta=(10 d B) \log \left(\frac{I}{I_{0}}\right)
$$

Ser pä differanser i lydintensitetsnivá for a finne andelen ou effekten $W^{\prime}$ ' en lyd som má reduseres

$$
\begin{aligned}
& \Delta \beta=\beta_{1}-\beta_{2} \\
&=20 \mathrm{~dB} \\
& \Delta \beta=(10 \mathrm{~dB}) \cdot \log \left(\frac{I_{90}}{I_{0}}\right)-\left(10 \mathrm{~dB} \cdot \log \left(\frac{I_{70}}{I_{0}}\right)\right. \\
& 20 \mathrm{~dB}=(10 \mathrm{~dB}) \cdot \log \left(\frac{I_{90}}{I_{70}}\right) \\
& \log \left(\frac{I_{70}}{I_{70}}\right)=2 \mathrm{~dB} \\
& I_{90}=10^{2} \\
& I_{70} \\
& I_{90}=100 \cdot I_{70}
\end{aligned}
$$

Ser pä endringsandelen for à finse redukgioner

$$
\frac{I_{90}-I_{70}}{I_{70}}=\frac{100 I_{70}-I_{70}}{100 I_{70}}=99 \%
$$

reduksjonen; lydintensitetsn: väet leder til en $99 \%$ reduksjon i levert effekt fra lydkilden\\

MARKPROB

15.79)

$$
\begin{aligned}
& f_{s}=625 \mathrm{MHz} \\
& d=50 \mathrm{~km} \\
& \Delta f=325 \mathrm{~Hz}
\end{aligned}
$$

\begin{itemize}
  \item Bruker likningen for besknivelse au dopploreffect a uttikke frekuenser mottat au regndrapene
\end{itemize}


\begin{align*}
& f_{r}=\frac{c \pm u_{r}}{c \pm u_{s}} \cdot f_{s} \\
& f_{r}=\frac{c+u_{r}}{c} \cdot f_{s} \tag{I}
\end{align*}


\begin{itemize}
  \item Belgene som reflekteres au drapene er som belger som sendes ut fra en kilde som beveger seg mot den originale kilden
\end{itemize}

$$
\begin{aligned}
& f_{r}^{\prime}=\frac{c \pm u_{r}}{c \pm u_{s}} \cdot f_{r} \\
& f_{r}^{\prime}=\frac{c}{c-u_{s}} \cdot f_{r} \text { (II) }
\end{aligned}
$$

\begin{itemize}
  \item Setter likning (I) inni likning (II) ogfäs
\end{itemize}

$$
\begin{aligned}
f_{r}^{\prime} & =\left(\frac{c}{c-u_{s}}\right)\left(\frac{c+u_{r}}{c}\right) f_{s} \\
& =\frac{c+u_{r}}{c-u_{s}} \cdot f_{s} \\
f_{r}^{\prime} & =\left(1+\frac{u_{r}}{c}\right)\left(1-\frac{u_{s}}{c}\right)^{-1} f_{s}
\end{aligned}
$$

siden $u_{r} \ll c$ og $u_{s} \ll c \Rightarrow$

$$
\begin{aligned}
& 1-\frac{u_{s}}{c} \approx 1-\frac{u_{r}}{c} \quad \text { og } \quad\left(1-\frac{u_{r}}{c}\right)^{-1} \approx 1+\frac{u_{r}}{c} \Rightarrow \\
& \cdot f_{r}^{\prime}=\left(1+\frac{u_{r}}{c}\right)^{2} \cdot f_{s} \text { der } u_{r} \ll c \Rightarrow\left(1+\frac{u_{r}}{c}\right)^{2} \approx 1+\frac{2 u_{r}}{c} \\
& f_{r}^{\prime}=\left(1+\frac{2 u_{r}}{c}\right)^{\prime} f_{s} \\
& \Delta f=f_{r}^{\prime}-f_{s} \\
& \Delta f=\left(1+\frac{\left.2 \frac{u_{r}}{c}\right) f_{s}-f_{s}}{}\right. \\
& u_{r}=\frac{c}{2 f_{s}} \Delta f=\frac{3.00 \cdot 10^{8} \frac{\mathrm{~m}}{\mathrm{~s}} \cdot 325 \mathrm{~Hz}}{2 \cdot 625 \cdot 10^{6} \mathrm{~Hz}} \\
& u_{r}=78.0 \frac{m}{s} \quad \text { forten til vinder er } 78.0 \frac{\mathrm{~m}}{\mathrm{~s}}
\end{aligned}
$$


\end{document}