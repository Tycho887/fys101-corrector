\documentclass[10pt]{article}
\usepackage[norsk]{babel}
\usepackage[utf8]{inputenc}
\usepackage[T1]{fontenc}
\usepackage{amsmath}
\usepackage{amsfonts}
\usepackage{amssymb}
\usepackage[version=4]{mhchem}
\usepackage{stmaryrd}
\usepackage{graphicx}
\usepackage[export]{adjustbox}
\graphicspath{ {./images/} }

\begin{document}

MARKPROB

Losningsforslag oblig 3\\
Del A\\
Terminalfart - den storste farten et legeme oppniar ved fall gjennom uft fra store hayder.

Sentripetalkraft-netto kraft sam pavirker et legeme til a folge sirkulcer bane.\\
Sentiffugalkraft-fiktiv kraft sam brukes til a forklare bevegelse $i$ et akselevert referansesystem

Massesenter\\
-tyngdepunktet (i et hamogent tyngdefelt)

\begin{itemize}
  \item det punktet i (eller utenfer) et legeme som beveger seg san an legemets masse var samblet; punktet
  \item det punktet der et legeme kan balanseres
\end{itemize}

MARKPROB

Del $B$\\
Eks-bok i ro pá bordet\\
\includegraphics[max width=\textwidth, center]{2025_01_13_b5bd1e6815146d849970g-02}

Tyngdekraften pà boka (Fg) or en fjemkraft, dors virker mellom legemer sam ikke er i fysisk kontalt. Den virher mellom boka og jorda.\\
Normalkraften pà boka (Fn) er en kontakticraft, durs virker mellom legemer sam er i kantakt med hwerandre, i dette filfellet meliom boka og bordet.\\
Etersam boka er i ro har is $N T: \overline{Z t}=0$.\\
For $y$-retning (der is har definert nedaver sam positiv) har ui

$$
\begin{aligned}
& \sum F_{y}=0 \\
& F_{g}-F_{n}=0 \\
& F_{g}=F_{n}
\end{aligned}
$$

De er altsio like store og motzatt rettet fordi boka er i ro, men de er

Ulike type krefter og har sine respektive motkrefter:\\
\includegraphics[max width=\textwidth, center]{2025_01_13_b5bd1e6815146d849970g-03}\\
$F_{n}$-normalkraft\\
$F_{n}^{\prime}$-motkraft iil normalkraft

Fg-tyrgdekraft\\
$F_{g}^{\prime}$ - motkraft fil tyngdekraft

MARKPROB

Del C\\
5.43\\
\includegraphics[max width=\textwidth, center]{2025_01_13_b5bd1e6815146d849970g-04}

$$
\begin{aligned}
& m_{1}=250 \mathrm{~g}=0,250 \mathrm{~kg} \\
& m_{2}=200 \mathrm{~g}=0,100 \mathrm{~kg} \\
& \theta=30^{\circ} \\
& \mu_{k}=0,100 \\
& \Delta y=30,0 \mathrm{~cm}=0,300 \mathrm{~m}
\end{aligned}
$$

$V i$ deler opp systemet $i$ to delsystemer;\\
1 -kloss pà surâtt plan-begger koordinatsystem parallelt med bevegelsesretning\\
$2-$ Kloss ; fall-definerer positiv retring\\
nedaver\\
Stram, masselos snar gir\\
$T_{1}=T_{2}=T \quad$ obs-forutsetter\\
$a_{1}=a_{2}=a \quad$ retninger is har definert sam positivt

Vi analyserer huert delsystem for seg for $\dot{a}$ finne aksebrasjonen. $V i$ antar at denne er konstant slik at il kan brike bevegersesligningene til à finne kloss 2 sin fart.\\
system 1 -vi ma dekomponere Fg for à analysere huer rething for seg:\\
\includegraphics[max width=\textwidth, center]{2025_01_13_b5bd1e6815146d849970g-05}

$$
\begin{aligned}
& F_{g x}=F_{g} \sin \theta=m_{1} g \sin \theta \\
& F_{g y}=F_{g} \cos \theta=m_{g} g \cos \theta \\
& f=\mu_{k} F_{n}
\end{aligned}
$$

Bruker sà N2 lor, $\bar{\Sigma} \vec{F}=m \vec{a}$ $y$-retning:\\
$\overline{F_{y}}=m_{a y}=0$ (ingen bevegelse i $y$-retn.)\\
$F_{n}+F_{g y}=0$ (ikke tatt hensyn fil refning, kun summert app krefter)\\
$F_{n}-m, g \cos \theta=O$ (her har is tatt hensyn Hil retring)

$$
F_{n}=m, g \cos \theta
$$

$x$-retning:

$$
\begin{aligned}
& \sum F_{x}=m a_{x}, a_{x}=a \\
& T_{1}+F_{g x}+f=m_{1} a \\
& T_{1}-m_{1} g \sin \theta-\mu_{k} F_{n}=m_{1} a
\end{aligned}
$$

Setter inn fer Fn:


\begin{equation*}
T_{1}-m_{1} \cdot g \sin \theta-\mu_{k} m_{1} g \cos \theta=m_{1} a \tag{*}
\end{equation*}


System 2:\\
Ser kon pà $y$-retning, har ingen kreffer' $x$-retning

$$
\begin{aligned}
& \sum F_{y}=m_{2} a \\
& T_{2}+F_{g_{2}}=m_{2} a \\
& -T_{2}+m_{2} g=m_{2} a \\
& T_{2}=m_{2} g-m_{2} a
\end{aligned}
$$

(fortegn utifra positio retning nedaer)\\
setter $T_{1}=T_{2}=T$ og seter inn $i(*)$

$$
m_{2} g-m_{2} a-m_{1} g \sin \theta-\mu_{k} m_{1} g \cos \theta=m a
$$

Alle ledd med a på venstre side og ave ledd med $g$ pä nogre side

$$
\begin{aligned}
& m_{1} a+m_{2} a=m_{2} g-m_{1} g \sin \theta-\mu_{k} m_{1} g \cos \theta \\
& a\left(m_{1}+m_{2}\right)=g\left(m_{2}-m_{1} \sin \theta-\mu_{k} g \cos \theta\right)
\end{aligned}
$$

Loser mhp $a$ :

$$
a=\frac{g\left(m_{2}-m_{1} \sin \theta-\mu_{k} g \cos \theta\right)}{m_{1}+m_{2}}
$$

setter til shutt inn verdier

$$
\begin{aligned}
& a=\frac{a, 81 \mathrm{~m} / \mathrm{s}^{2}\left(0,200 \mathrm{~kg}-0,250 \mathrm{~kg} \cdot \sin 30^{\circ}-0,100 \cdot 981 \mathrm{mg} \cdot \cos 30^{\circ}\right)}{0,250 \mathrm{~kg}+0,200 \mathrm{~kg}} \\
&=1,163 \mathrm{~m} / \mathrm{s}^{2} \quad \text { (to ellstra } 95 \cdot \operatorname{silfer}: \\
&\text { mellamregning })
\end{aligned}
$$

Bevegelsesligning:

$$
\begin{aligned}
v^{2} & =v_{0}^{2}+2 a \Delta y, v_{0}=0 \\
v & =\sqrt{2 a \Delta y} \\
& =\sqrt{2 \cdot 1,163 \mathrm{~m} / \mathrm{s}^{2} \cdot 0,300 \mathrm{~m}} \\
& =0,84 \mathrm{~m} / \mathrm{s}
\end{aligned}
$$

(To gieldende siffer isvaret pga $30^{\circ}$ Kan voere alt fra $29,6^{\circ}$ til $30,4^{\circ}$ )\\

MARKPROB

5.93\\
\includegraphics[max width=\textwidth, center]{2025_01_13_b5bd1e6815146d849970g-08(1)}

$$
\begin{aligned}
m & =750 \mathrm{~kg} \\
r & =160 \mathrm{~m} \\
v & =90 \mathrm{~km} / \mathrm{t} \\
& =25 \mathrm{~m} / \mathrm{s}
\end{aligned}
$$

\begin{center}
\includegraphics[max width=\textwidth]{2025_01_13_b5bd1e6815146d849970g-08}
\end{center}

Dersom kraften fra underlaget skal stà normalt pai underlaget mai friksjanskraftern poi bilen vare sull. vi dekomponerer kreftene $i x$-og $y$-retning (se figur-disse er ikke parallect og normact pä skräplaret slik ui pleïer)

$$
N_{2}: \vec{\varepsilon} \vec{F}=m \vec{a}
$$

$x$-retning:

$$
\begin{aligned}
& \sum F_{x}=m a_{x}, a_{x}=\frac{v^{2}}{r} \text { (sentripeti(alseleasian) } \\
& F_{n x}=\frac{v^{2}}{r} \\
& F_{n} \sin \theta=\frac{v^{2}}{r} \quad(*)
\end{aligned}
$$


\end{document}