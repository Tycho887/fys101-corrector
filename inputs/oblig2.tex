\documentclass[10pt]{article}
\usepackage[norsk]{babel}
\usepackage[utf8]{inputenc}
\usepackage[T1]{fontenc}
\usepackage{amsmath}
\usepackage{amsfonts}
\usepackage{amssymb}
\usepackage[version=4]{mhchem}
\usepackage{stmaryrd}
\usepackage{graphicx}
\usepackage[export]{adjustbox}
\graphicspath{ {./images/} }

\begin{document}

MARKPROB

Oblig 2\\
(A)

Referansesystem - et system vi beskriver bevegeke i forhold til

Treghetssystem - et referansesystem som itcke er under aleselerasjon

Uachengighetsprinsippet - det at man kan dekomponere vektorer og se pà huer retving for seg, og hoer retring er uachengig av hoevandre

Friksjonsteraft - leraft mellom to objekter som er i kontalat med hererandre som motvirker glidning mellom objektene\\

MARKPROB

(B)

Friksjonstereftene virker parallelt med kontaktoverflatene og har vetring som motarbeider glidning mellom obljektene\\
(c)

MARKPROB

Hookes lov\\
Nar en fjor blir strakket eller presset sammen fra sin Likevektsposisjon er leraften den utgjor gitt ved

$$
F_{x}=-k x
$$

der ker fiarkonstanten, et mal po foevens stichet. En negativ verdi ao $x$ betyr at fjaren har blitt presset sammen en austand $|x|$ fra likeveletsposisjonen. Det negative fortegnet i likeningen\\
negative fortegnet i likeningen betyr at noir fjoren blir strukket eller presset sammen i en retuing virker leraften i motsatt retning.\\

MARKPROB

(D)\\
\includegraphics[max width=\textwidth, center]{2025_01_16_0a7b6f4f4d45045ef380g-3}

$$
\begin{aligned}
& m=50 \mathrm{~kg} \\
& \theta=60^{\circ} \\
& g=9,81 \mathrm{~m} / \mathrm{s}^{2} \\
& v=0
\end{aligned}
$$

a) Fiun snordraget $\vec{T}$ og normalkeraften $\vec{F}_{n}$

Necutons 2. Lov i $x$-retring:

$$
\begin{aligned}
& \sum F_{x}=\max \quad a_{x}=0 \quad(v=0) \\
& \sum F_{x}=0 \\
& T-F_{g x}=0 \\
& T=F_{g x} \\
& T=F_{a} \cdot \sin \theta
\end{aligned}
$$

$$
\begin{aligned}
& T=F g \cdot \sin \theta \\
& T=m \cdot g \cdot \sin \theta \\
& T=50 \mathrm{~kg} \cdot 9,81 \frac{\mathrm{~m}}{\mathrm{~s}^{2}} \cdot \sin 60^{\circ}=0,42 \mathrm{kN}
\end{aligned}
$$

Necetons 2. loo i $y$-vetricing

$$
\begin{aligned}
& \sum F_{y}=m a_{y} \quad a y=0 \quad(v=0) \\
& \sum F_{y}=0 \\
& F_{n}-F_{g y}=0 \\
& F_{n}=F_{g y} \\
& F_{n}=F_{g} \cdot \cos \theta \\
& F_{n}=m \cdot g \cdot \cos \theta \\
& F_{n}=50 \mathrm{lg} \cdot 9,81 \frac{\mathrm{~m}}{\mathrm{~s}^{2}} \cdot \cos 60^{\circ}=0,25 \mathrm{kN}
\end{aligned}
$$

Snordraget er 0,42kN og normalkraften er 0,25kN\\
b) Finn snordraget som en funlesion der $\theta$ og $m$

Braker vesonnementet fra deloppgave a):

$$
\begin{aligned}
& T=m \cdot g \cdot \sin \theta \\
& \text { Nar } \theta=0^{\circ} \Rightarrow \sin 0^{\circ}=0 \\
& T=0
\end{aligned}
$$

$$
T=0
$$

Nar $\theta=90^{\circ} \Rightarrow \sin 90^{\circ}=1$

$$
T=m \cdot g=F g
$$

Disse resultatene er begge rimelige.\\

MARKPROB

\includegraphics[max width=\textwidth, center]{2025_01_16_0a7b6f4f4d45045ef380g-5(1)}

Kraftmalingen pà velaten er lik leraften veliten\\
\includegraphics[max width=\textwidth, center]{2025_01_16_0a7b6f4f4d45045ef380g-5}

Newtons 2. Lov i $y$-retriens:

$$
\begin{aligned}
& \sum F_{y}=m a_{y} \quad d y=0 \\
& \sum F_{y}=0 \\
& F_{n}-F_{g y}=0 \\
& F_{n}=F_{g y} \\
& F_{1 n}=F_{n} \cdot \ln A
\end{aligned}
$$

$$
\begin{aligned}
& 1 n-1 g y \\
& F_{n}=F_{g} \cdot \cos \theta \\
& F_{n}=m \cdot g \cdot \cos \theta \\
& F_{n}=65 \mathrm{~kg} \cdot 9,81 \frac{\mathrm{~m}}{\mathrm{~s}^{2}} \cdot \cos 30^{\circ}=0,55 \mathrm{kN}
\end{aligned}
$$

Kraftualingen pai velcten viser $0,55 \mathrm{kN}$\\

MARKPROB

\includegraphics[max width=\textwidth, center]{2025_01_16_0a7b6f4f4d45045ef380g-6}

$$
\begin{aligned}
& T=T^{\prime} \\
& a_{1 y}=0 \\
& a_{1 x}=a_{2 x}=a x \\
& m_{1}=m_{2}=5,0 \mathrm{~kg} \\
& \theta=30^{\circ}
\end{aligned}
$$

a) $V_{i}$ skal finne $a_{x}$ of $T$ uttrglet ved $\theta, m_{1}, \circ g m_{2}$

Newtons 2. loo for $m_{1}$ i $x$-retning:

$$
\begin{aligned}
& \sum F_{1 x}=m_{1} a_{1 x} \\
& T-F_{g x 1}=m_{1} a_{x} \\
& T=m_{1} d_{x}+F_{g x 1} \\
& T=m_{1} a_{x}+m_{1} \cdot g \cdot \sin \theta
\end{aligned}
$$

Newtons 2. lov for $m_{2}$ i $x$-retring:

$$
\begin{array}{ll}
\sum F_{2 x}=m_{2} a_{2 x} \\
F_{\text {an }}-T=\ln -1
\end{array} \quad d_{2 x}=d x
$$

$$
\begin{array}{ll}
L 2 x & a_{2 x}=a_{x} \\
F g 2-T=m_{2 x} & \\
T=F_{g 2}-m_{2} d_{x} & \\
T=m_{2} \cdot g-m_{2} a_{x} &
\end{array}
$$

Vi hor ni to uttralke for $T$. Vi setter de lik huerandre org loser for $d x$ :

$$
\begin{aligned}
& m_{1} d x+m_{1} g \sin \theta=m_{2} g-m_{2} a_{x} \\
& m_{1} d_{x}+m_{2} d_{x}=m_{2} g-m_{1} g \sin \theta \\
& d_{x}\left(m_{1}+m_{2}\right)=g\left(m_{2}-m_{1} \sin \theta\right) \\
& d_{x}=\frac{g\left(m_{2}-m_{1} \sin \theta\right)}{m_{1}+m_{2}}
\end{aligned}
$$

Vi kan nim sette ax inn i en our de to uttrghkene for $T$ org lose for $T$. Vi setter inn i den andre?

$$
\begin{aligned}
& T=m_{2} g-m_{2} d x \\
& T=m_{2} g-m_{2} \cdot \frac{\left(m_{2}-m_{1} \sin \theta\right)}{m_{1}+m_{2}} g \\
& T=\frac{m_{2}\left(m_{1}+m_{2}\right)-m_{2}\left(m_{2}-m_{1} \sin \theta\right)}{m_{1}+m_{2}} g \\
& T=\frac{m_{2}\left(m_{1}+m_{2}-m_{2}+m_{1} \sin \theta\right)}{m_{1}+m_{2}} g
\end{aligned}
$$

$$
T=\frac{m_{1} m_{2}(1+\sin \theta)}{m_{1}+m_{2}} g
$$

b) $m_{1}=m_{2}=m=5,0 \mathrm{~kg}, \theta=30^{\circ}$

Finn $\alpha_{x}$ :

$$
\begin{aligned}
& d_{x}=\frac{g\left(m_{2}-m_{1} \sin \theta\right)}{m_{1}+m_{2}} \quad m_{1}=m_{2}= \\
& d_{x}=\frac{g(m-m \sin \theta)}{m+m} \\
& d_{x}=\frac{\operatorname{mg}(1-\sin \theta)}{2 m} \\
& d_{x}=\frac{1}{2} g(1-\sin \theta) \\
& d_{x}=\frac{1}{2} \cdot 9,81 \frac{m}{s^{2}}\left(1-\sin 30^{\circ}\right)=2,5 \mathrm{~m} / \mathrm{s}^{2}
\end{aligned}
$$

Finn T:

$$
\begin{aligned}
& T=\frac{m_{1} m_{2}(1+\sin \theta)}{m_{1}+m_{2}} g \quad m_{1}=m_{2}=m \\
& T=\frac{m \cdot m(1+\sin \theta)}{m+m} g
\end{aligned}
$$

$$
\begin{aligned}
& T=\frac{1}{2} \operatorname{mg}(1+\sin \theta) \\
& T=\frac{1}{2} \cdot 5,0 \mathrm{~kg} \cdot 9,81 \frac{\mathrm{~m}}{\mathrm{~s}^{2}} \cdot\left(1+\sin 30^{\circ}\right)=37 \mathrm{~N}
\end{aligned}
$$

Aleselerajponen er $2,5 \mathrm{~m} / \mathrm{s}$ og snordraget er 37 N


\end{document}