\documentclass[10pt]{article}
\usepackage[norsk]{babel}
\usepackage[utf8]{inputenc}
\usepackage[T1]{fontenc}
\usepackage{graphicx}
\usepackage[export]{adjustbox}
\graphicspath{ {./images/} }
\usepackage{amsmath}
\usepackage{amsfonts}
\usepackage{amssymb}
\usepackage[version=4]{mhchem}
\usepackage{stmaryrd}

\begin{document}

MARKPROB

FYS101\\
Losningsferstag oblig 4\\
Del A\\
Arbeid - generelt:\\
\includegraphics[max width=\textwidth, center]{2025_01_16_a7a10def23a6031962cag-1}

MARKPROB

Del B\\
Potensiell energi - generelt

$$
\Delta U=U_{2}-U_{1}=-\int_{1}^{2} \vec{F} \cdot d \vec{U}
$$

(samme figur som for orbeid)\\

MARKPROB

7.39)\\
\includegraphics[max width=\textwidth, center]{2025_01_16_a7a10def23a6031962cag-2}

$$
\begin{aligned}
& m=3.00 \mathrm{~kg} \\
& k=400 \mathrm{~m} \\
& y_{0}=5.00 \mathrm{~m} \\
& y_{1}=0 \\
& x_{0}=0
\end{aligned}
$$

a) Har bevaring av energi ettersom at det ikee virker noer friksjonskraft pá klossen.

\begin{itemize}
  \item Braker arbeid-eregi-loven:
\end{itemize}

$$
\begin{aligned}
& E_{1}=E_{0} \\
& K_{1}+U_{g_{1}}+U_{f_{1}}=K_{0}+U_{g_{0}}+U_{f_{0}} \\
& \frac{1}{2} m y_{1}^{20}+m g y_{1}^{0}+\frac{1}{2} K_{1}^{2}=\frac{1}{2} m{f_{0}^{2}}_{20}+m g y_{0}+\frac{1}{2} k y_{0}^{20}
\end{aligned}
$$

\begin{itemize}
  \item Kcossen starter i ro og slutter i ro, som gir $v_{0}=v_{1}=0$, og sitter igjen med
\end{itemize}

$$
\frac{1}{2} k x_{1}^{2}=m g y_{0}
$$

og laser mhp $x_{1}$

$$
\begin{aligned}
x_{1} & =\sqrt{\frac{2 m g y_{0}}{k}} \\
& =\sqrt{\frac{2 \cdot 3.00 \mathrm{~kg} \cdot 9.81 \mathrm{~N} \cdot 5.00 \mathrm{~m}}{400 \mathrm{~m}}} \\
x_{1} & =0.858 \mathrm{~m}
\end{aligned}
$$

Fjeren ble komprimert omtrent 85.8 cm .\\
b) Blokken vil bli tilfort arbeid fra fiaren sem vil skyve der i negativ $x$-retning. Siden erergien er bevart, vil klosser skyves tilbake til oppronelig hoyde\\

MARKPROB

7.65)\\
\includegraphics[max width=\textwidth, center]{2025_01_16_a7a10def23a6031962cag-3}

$$
\begin{aligned}
& m=2.0 \mathrm{~kg} \\
& y_{0}=3.0 \mathrm{~m} \\
& y_{1}=y_{2}=0 \\
& v_{0}=v_{2}=0 \\
& s=9.0 \mathrm{~m}
\end{aligned}
$$

\includegraphics[max width=\textwidth, center]{2025_01_16_a7a10def23a6031962cag-3(1)}\\
a) Ser farst pai prosessen : figur (1) og 16 sermhp $v_{1}$

\begin{itemize}
  \item Bruker arbeid-energi-loven:
\end{itemize}

$$
\begin{aligned}
& E_{1}=E_{0} \\
& K_{1}+U_{1}=K_{0}+U_{2}^{1} \\
& \frac{1}{2} m v_{1}^{2}+m g y_{1}^{\prime}=\frac{1}{2} g v_{0}^{2}+m g y_{0} \\
& \frac{1}{2} x v_{1}^{2}=x g y_{0} \\
& v_{1}=\sqrt{2 g y_{0}} \\
& =\sqrt{2.9 .81 \frac{\mathrm{~N}}{\mathrm{~kg}} \cdot 3.0 \mathrm{~m}} \\
& \underline{U}_{1}=7.7 \frac{\mathrm{~m}}{\mathrm{~s}}
\end{aligned}
$$

Klosser has er fort pai $7.7 \frac{m}{3}$; bunner our rampen\\
b) Har et friksonsarseid som virker pai klosser i prosess (2). Skal finse disne

$$
\begin{aligned}
& E_{2}=E_{1}+W_{R} \\
& K_{2}+U_{2}=K_{1}+U_{1}+W_{R} \\
& \frac{1}{2} m{y_{2}^{20}}_{20}^{2 g} y_{2}^{0}=\frac{1}{2} m v_{1}^{2}+m g g_{1}^{0}+W_{R}
\end{aligned}
$$

\begin{itemize}
  \item Loser mhp friksjonsarbeidet $W_{R}$ :
\end{itemize}

$$
\begin{aligned}
W_{R} & =-\frac{1}{2} m v_{1}^{2} \\
& =-\frac{1}{2} \cdot 2.0 \mathrm{~kg} \cdot\left(7.672 \frac{\mathrm{~m}}{3}\right)^{2} \\
W_{R} & =-59 \mathrm{~J}
\end{aligned}
$$

Et arbeid pä 59 J forsviner som resultat av friksjon.\\
C)\\
\includegraphics[max width=\textwidth, center]{2025_01_16_a7a10def23a6031962cag-5}

\begin{itemize}
  \item Finner friksjonskoeffisienten ved a beyyte
\end{itemize}

Neurtons 2. Wor i $y$ - retning:

$$
\begin{aligned}
& \sum F_{y}=m a_{y}\left(a_{y}=0, \text { all bevegelse skjer i } x \text {-retring }\right) \\
& F_{n}-F_{g}=0 \\
& F_{n}=m g \text { (I) } \\
& f=\mu F_{n} \text { (II) }
\end{aligned}
$$

\begin{itemize}
  \item Setter (I) inn i (II):
\end{itemize}

$$
f=\mu m g \text { (III) }
$$

\begin{itemize}
  \item Friksjonsarbeidet er gitt ved
\end{itemize}

$$
W_{R}=f S \cdot \cos 180^{\circ}(\mathrm{IV})
$$

\begin{itemize}
  \item Setter (III) im (IV) og loser mhp M:
\end{itemize}

$$
\begin{aligned}
W_{R} & =M M^{M g} \\
M & =-\frac{W_{R}}{m g S} \\
& =-\frac{158.86 \mathrm{~J}}{2.0 \mathrm{~kg} \cdot 9.81 \mathrm{~kg} \cdot 9.0 \mathrm{~m}}
\end{aligned}
$$

$$
p=0.33
$$

Friksjonskoeffisienter mellom blokker og det horisontale underlaget er 0.33

MARKPROB

Del D\\
(.py-versjon av denne legges på Canvas - merk at dette er kun et forslag)

\begin{verbatim}
import numpy as np
import matplotlib.pyplot as plt
# the gravitational constant is
g = 9.81
mass = 20.0 #kg (mass of a curling stone)
def vertical_normal_force(mass):
    # mu is the friction coefficient
    N = mass * g
    return N
normal_force_magnitude = vertical_normal_force(mass)
# %% Exercise 1) define a function with two parameters that return the force
# done by friction with a coefficent mu ( }\mu\mathrm{ ) on a horizontal plane, based on the
# normal force [Eq. 5-4 of Tipler and Mosca] (substitute 'missing_parameter'
# and '???' below
def force_by_friction(normal_force_magnitude, mu):
    return normal_force_magnitude * mu
# %% Exercise 2) create a function called 'kinetic_energy' which return the
# kinetic energy. The function should have two parameters only:
def kinetic_energy(mass, speed):
    K = mass
    return K
# %% Exercise 3) We know that a curling stone that has a speed of 1.0 m/s travels
# exactly }6.4\textrm{m}\mathrm{ and stops right at the centre of the curling target. We
# also know that the only force acting on the curling stone as it travels is the
# friction force. Find the friction coefficient mu ( }\mu\mathrm{ )?
# (You should make use of \DeltaE_mech = - W_nonconservative and E_mech = K+U)
# E_mech = K (because U does not change)
K = kinetic_energy(mass, speed=1)
# (we know that the curling stone stops after 6.4 m)
# substitute 'W_nonconservative' below and find mu ( }\mu\mathrm{ )
mu = K/(mass * g * 6.4)
\end{verbatim}

In [3]: mu\\[0pt]
Out[3]: 0.007963812436289498\\
$\mu=0,0080$\\
(to gjeldende siffer pga oppgitt fart 1,0 m/s og lengde 6,4 m)

\begin{verbatim}
4 6
    # %% Exercise 4) knowing the friction coefficient mu ( }\mu\mathrm{ ), plot the
    # speed of the curling stone as it moves from the starting line (hog line)
    # starting with 1 m/s, all the way to the centre of the target (at 6.4 metres).
    # Define a function for the speed of the curling stone. It should be a function
    # of four parameters: the starting kinetic energy ' }K\mathrm{ ', the coefficient 'mu',
    # the mass 'mass', and the distance travelled 'L' (to use a square-root you can
    # use np.sqrt())
    def speed_curling_stone(K, mass, mu, L):
        v = np.sqrt(2 * K / mass - 2 * mu * g * L)
        return v
    # We need to define a distance array from 0 to 6.4 metres. Let's do it using
    # np.linspace with 100 'datapoints'
    distance = np.linspace(0.0, 6.4, 100)
    plt.plot(distance, speed_curling_stone(K, mass, mu, distance), 'o-')
    plt.ylabel('speed of the curling stone')
    plt.xlabel('distance from the starting line')
\end{verbatim}

\begin{center}
\includegraphics[max width=\textwidth]{2025_01_16_a7a10def23a6031962cag-7}
\end{center}


\end{document}