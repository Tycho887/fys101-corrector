\documentclass[10pt]{article}
\usepackage[italian]{babel}
\usepackage[utf8]{inputenc}
\usepackage[T1]{fontenc}
\usepackage{graphicx}
\usepackage[export]{adjustbox}
\graphicspath{ {./images/} }
\usepackage{amsmath}
\usepackage{amsfonts}
\usepackage{amssymb}
\usepackage[version=4]{mhchem}
\usepackage{stmaryrd}

\begin{document}

MARKPROB

Lesnings forslag oblig 9\\
11.32)\\
\includegraphics[max width=\textwidth, center]{2025_01_16_ac697858c9e67eb974c3g-1}

$$
\begin{aligned}
& R_{E}=6370 \mathrm{~km} \\
& h=400 \mathrm{~km} \\
& M_{E}=5.97 \cdot 10^{21} \mathrm{~kg}
\end{aligned}
$$

a) Finner tyngdefeltet ved

$$
g=\frac{5_{g}}{m}
$$

der $m$ er massen fil romstasjonen, og har

$$
F_{g}=\frac{\operatorname{Gm} M_{E}}{\left(R_{E}+h\right)^{2}} \text { (Newtons gravitasionslov) }
$$

Setter inn; utrykket for $g$ og finner

$$
\begin{aligned}
g & =\frac{G M_{E}}{\left(R_{E}+h\right)^{2}} \\
& =\frac{6.673 \cdot 10^{-11} \mathrm{~N} \frac{\mathrm{mg}^{2}}{\mathrm{~kg}^{2}} \cdot 5.97 \cdot 10^{24} \mathrm{~kg}}{\left(6370 \cdot 10^{3} \mathrm{~m}+400 \cdot 10^{3} \mathrm{~m}\right)^{2}} \\
g & =8.69 \mathrm{~N}
\end{aligned}
$$

Tyngdefeltet har omtrent samme verdi pa romstasionen som ved jordoverflaten\\
b) Det som forer til muskelsuakhet er ikke mangel pà tyngdefratten, men at astronautere iklee opplever nover normalkraft.\\
Mangeler pà sammentrybling forer til at musclere og beina blic suackere

MARKPROB

$$
\begin{aligned}
& T=27.3 d=2358720 \mathrm{~s} \\
& r_{m}=3.84110^{8} \mathrm{~m} \\
& \frac{1}{9} G^{M} \quad \begin{array}{l}
\mathrm{g}=\mathrm{MmG}^{2} \\
\mathrm{~g}=\frac{G M E}{R^{2}}
\end{array}
\end{aligned}
$$

skal finne jordas masse\\
Begynner med á brike N2L pa mánen

$$
\begin{aligned}
& \sum F=M_{M} a \quad\left(a=\frac{v^{2}}{r}\right) \\
& F_{g}=M_{M} \frac{v^{2}}{r_{m}} \\
& \frac{G M_{M} M_{E}}{r_{m}^{2}}=M_{M} \frac{v_{1}^{2}}{r_{m}} \quad \frac{r_{m}}{M_{m}} \\
& G M_{E}=v^{2} \quad\left(v=\frac{2 \pi r_{m}}{T}\right) \\
& r_{m} \\
& \frac{G M_{E}}{r_{m}}=\left(\frac{2 \pi r_{m}}{T}\right)^{2}
\end{aligned}
$$

og loser mhp $M_{E}$

$$
\begin{aligned}
& M_{E}=\frac{4 \pi^{2}}{6} \frac{r_{m}^{3}}{T^{2}} \\
& =\frac{4 \pi^{2}\left(13.84 \cdot 10^{8} m\right)^{3}}{6.673 \cdot 10^{-11} N \frac{x^{2}}{x^{2}}{ }^{2}(2358720)^{2}} \\
& M_{E}=16.02 \cdot 10^{24} \mathrm{~kg}
\end{aligned}
$$

Jordas masse ble kalkulert til $6.02 \cdot 10^{24} \mathrm{~kg}$, noe som stemmer greit overens med tabelluerdien $5.97 .10^{24} \mathrm{~kg}$\\

MARKPROB

\includegraphics[max width=\textwidth, center]{2025_01_16_ac697858c9e67eb974c3g-3}

\begin{itemize}
  \item Energien til en sattelit med masse m som gar bane rundt jorda es gitt ved
\end{itemize}

$$
\begin{aligned}
& E=K+U \\
& E=\frac{1}{2} m v^{2}-\frac{G M E m,(I)}{R}
\end{aligned}
$$

der $R$ er austainden mullom sattelitten og jordas sentrum

\begin{itemize}
  \item Bruker N2L for à fine farten v sattelitten beveger seg ; bane med
\end{itemize}

$$
\begin{aligned}
& \sum F=m a\left(a=\frac{v^{2}}{R}\right) \\
& F_{g}=\frac{m v^{2}}{R} \\
& \frac{G M E m}{R^{2}}=\frac{m v^{2}}{R} \left\lvert\, \frac{R}{m}+C(m) \quad g=\frac{m M}{R^{2}}\right. \\
& v^{2}=\frac{G M E}{R}
\end{aligned}
$$

\begin{itemize}
  \item Setter likning (II) inn i living (I) Cor a knne et utrykk for totcalenergien til sattel; tten
\end{itemize}

$$
\begin{aligned}
& E=\frac{1}{2} \frac{G M_{E m}}{R}-\frac{G M_{E} m}{R} \\
& E=-\frac{G M E m}{2 R}
\end{aligned}
$$

\begin{itemize}
  \item Finner energiforskjellen til sayelitten for de to ulike banene
\end{itemize}

$$
\begin{aligned}
\Delta E & =E_{2}-E_{1} \\
& =-\frac{G M_{E} m}{2\left(R E+h_{2}\right)}+\frac{G M E m}{2\left(R_{E}+h_{1}\right)} \\
& =\frac{1}{2} G M_{E M}\left(\frac{1}{R_{E}+h_{1}}-\frac{1}{R_{\varepsilon}+h_{2}}\right) \\
& =6.67 \cdot 10^{-11} \mathrm{Nm}^{\frac{\mathrm{m}^{2}}{2}} 5.97 \cdot 10^{24} \mathrm{~kg} \cdot 500 \mathrm{~kg}\left(\frac{1}{6370 \cdot 10^{3} \mathrm{~m}+1000 \cdot 10^{2} \mathrm{~m}}-\frac{1}{6370 \cdot 10^{3}+35790 \cdot 10^{3} \mathrm{~m}}\right) \\
\Delta E & =1.11 \cdot 10^{10} \mathrm{~J}
\end{aligned}
$$

Det krever 11.1 GJ mer energ; for à fà sattelitten i den geosynkrone sattelittbaner enn: der "vanlige" Satielitibasen


\end{document}