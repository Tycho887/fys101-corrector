\documentclass[10pt]{article}
\usepackage[norsk]{babel}
\usepackage[utf8]{inputenc}
\usepackage[T1]{fontenc}
\usepackage{amsmath}
\usepackage{amsfonts}
\usepackage{amssymb}
\usepackage[version=4]{mhchem}
\usepackage{stmaryrd}
\usepackage{graphicx}
\usepackage[export]{adjustbox}
\graphicspath{ {./images/} }

\begin{document}
FYS101-Lesiningsfórslag oblig. 6\\

MARKPROB

Dei A\\
Vinhelfart (u) ag barefast (v.t):

$$
v_{t}=r u t
$$

Vinkelaleselerasjon ( $\alpha$ ) og tangintiell banealeselerasjon (at)

$$
a_{t}=r \alpha
$$

MARKPROB

Del B\\
Neustens 2. lar for rotasjon

$$
\sum_{i} \tau_{i}, \text { eust }=I \alpha
$$

Summen air elesterne kraftmanent er Cll treghetsmament vinhelaleseleasjon\\

MARKPROB

9.28)\\
\includegraphics[max width=\textwidth, center]{2025_01_16_b355359b6e574693917ag-2}\\
a) Finner vinkelfarten ved:

$$
\begin{aligned}
w & =\frac{v}{r} \\
& =\frac{25 \mathrm{~m}}{90 \mathrm{~m}} \\
w & =0.28 \frac{\mathrm{rad}}{\mathrm{~s}}
\end{aligned}
$$

Vinkelfarten er pà $0.28 \frac{\mathrm{rad}}{\mathrm{s}}$\\
b) Finner antall onganger i lopet our $t=30 \mathrm{~s}$ $\theta=w \cdot t \quad$ (for konstart vinkel fart)

$$
=0.2778 \frac{\mathrm{rad}}{\mathrm{~s}} \cdot 30 \mathrm{~s} \cdot \frac{1 \mathrm{rev}}{2 \pi \mathrm{rad}}
$$

$$
\theta=1.3 \mathrm{rev}
$$

MARKPROB

\begin{center}
\includegraphics[max width=\textwidth]{2025_01_16_b355359b6e574693917ag-3}
\end{center}

$$
\begin{aligned}
& \omega_{1}=0 \\
& \theta_{1}=5.0 \mathrm{rad} \\
& \theta_{0}=0 \\
& t=2.8 \mathrm{~s}
\end{aligned}
$$

Konstant vinkelakselerasjon

$$
\text { - } \begin{aligned}
w_{1} & =w_{0}+\alpha \cdot t \\
\alpha & =\frac{w_{1}^{0}-w_{0}}{t} \\
\alpha & =\frac{-w_{0}}{t}(I) \\
\cdot & \theta_{1}=\theta_{0}+w_{0} t+\frac{1}{2} \alpha t^{2}
\end{aligned}
$$

og setter in likning (I) $\Rightarrow$

$$
\begin{aligned}
\theta_{1} & =\not \theta_{0}^{0}+w_{0} t-\frac{1}{2} \frac{w_{0}}{t} t^{2} \\
& =w_{0} t-\frac{1}{2} w_{0} t \\
\theta_{1} & =\frac{1}{2} w_{0} t
\end{aligned}
$$

og loser med hensyn pä $w_{0}$

$$
\begin{aligned}
w_{0} & =\frac{2 \theta_{1}}{t} \\
& =\frac{2.5 .0 \mathrm{rad}}{2.8 \mathrm{~s}} \\
w_{0} & =3.6 \mathrm{rgd}
\end{aligned}
$$

Vinkelfarten var pà $3.6 \frac{\mathrm{rad}}{\mathrm{s}}$ for hiulet begynte a bremse\\

MARKPROB

9.44)\\
\includegraphics[max width=\textwidth, center]{2025_01_16_b355359b6e574693917ag-4}

Treghetsmomentet for en solid sfere om sitt eget massesenter er gitt ved

$$
I_{\mathrm{cm}}=\frac{2}{5} M R^{2} \quad \text { (tabell 9.1) }
$$

Finner treghetsmonestet nair der solide sfaren roterer om en akse som ligger tangent pa overflaten ved a berytc parallellakse-teorenet

$$
I=I_{c m}+M h^{2}, \text { der } h=R
$$

Setter inn litninges for $I_{\mathrm{cm}}$ og Rimer

$$
\begin{aligned}
& I=\frac{2}{5} M R^{2}+M R^{2} \\
& I=\frac{7}{5} M R^{2}
\end{aligned}
$$

MARKPROB

\begin{center}
\includegraphics[max width=\textwidth]{2025_01_16_b355359b6e574693917ag-5}
\end{center}

$$
\begin{aligned}
& M=8.00 \mathrm{~kg} \\
& m=1.20 \mathrm{~kg} \\
& d=1.00 \mathrm{~m}
\end{aligned}
$$

Masser M sitt treghetsmoment:

$$
I_{\text {njul }}=M R^{2}
$$

Massen m sit treghetsmoment:

$$
I_{\text {eike }}=\frac{1}{3} \mathrm{~mL}^{2}
$$

Har $R=L=\frac{D}{2}$\\
Hele hiulets treghetsmoment er gitt vedsumner air alle delkomponentere sine treghetsmoment

$$
\begin{aligned}
I & =I_{\text {njul }}+6 \cdot I_{\text {eike }} \\
& =M R^{2}+6 \cdot \frac{1}{3} m L^{2} \\
& =\frac{1}{4} M D^{2}+\frac{1}{2} m D^{2} \\
& =\frac{1}{4}(M+2 \mathrm{~m}) D^{2} \\
& =\frac{1}{4}(8.00 \mathrm{~kg}+2 \cdot 1.20 \mathrm{~kg}) \cdot(1.00 \mathrm{~m})^{2} \\
I & =2.60 \mathrm{kgm}^{2}
\end{aligned}
$$


\end{document}