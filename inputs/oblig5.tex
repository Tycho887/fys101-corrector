\documentclass[10pt]{article}
\usepackage[norsk]{babel}
\usepackage[utf8]{inputenc}
\usepackage[T1]{fontenc}
\usepackage{amsmath}
\usepackage{amsfonts}
\usepackage{amssymb}
\usepackage[version=4]{mhchem}
\usepackage{stmaryrd}
\usepackage{graphicx}
\usepackage[export]{adjustbox}
\graphicspath{ {./images/} }

\begin{document}

MARKPROB

FY5101-Losningsfarslag oblig 5\\
$\operatorname{Dela}$\\
Kraftstot - mäl pai baide styrke og vainghet our en leraft

$$
\vec{I}=\int_{t_{i}}^{t_{f}} \vec{F} d t
$$

MARKPROB

Del $B$\\
Elastisk stoot - stot der den kinetishe energien er bevart\\
Velastist strot-stot der den kinetiske energlen ithe er bevart. I begge tilfeller er bevegelsesmengden bevort.\\

MARKPROB

a) 8.53

For (i)\\
\includegraphics[max width=\textwidth, center]{2025_01_16_5f84a3aaf9b94e598751g-2}

$$
\begin{aligned}
& m_{1}=m_{2}=2000 \mathrm{~kg} \\
& v_{i i}=30 \mathrm{~m} / \mathrm{s} \\
& v_{2 i}=10 \mathrm{~m} / \mathrm{s}
\end{aligned}
$$

EHer (f)

$$
m m \rightarrow v_{f}
$$

$$
v_{f}=?
$$

Fullstendig velastisk kollistan $\rightarrow$ bevegelsesmengoten er bevart

$$
\begin{aligned}
& P_{1 i}+P_{2 i}=P_{f} \\
& m v_{1 i}+M v_{2 i}=2 m v_{f}
\end{aligned}
$$

Loser mhp $v f$ :

$$
\begin{aligned}
v_{f} & =\frac{v_{i i}+v_{2 i}}{2} \\
& =\frac{30 \mathrm{~m} / \mathrm{s}+10 \mathrm{~m} / \mathrm{s}}{2} \\
v_{f} & =20 \mathrm{~m} / \mathrm{s}
\end{aligned}
$$

Felles fart er $20 \mathrm{~m} / \mathrm{s}$ etter kollisjonen\\
b) Finner uttrykk for den kinetiske enegien til systemet for og etter kollisjonen:

$$
\begin{aligned}
K_{\text {sys } i} & =K_{1 i}+K_{2 i} \\
& =\frac{1}{2} m v_{1 i}^{2}+\frac{1}{2} m v_{2 i}^{2} \\
& =\frac{1}{2} m\left(v_{1 i}^{2}+v_{2 i}^{2}\right) \\
K_{\text {sys } f} & =\frac{1}{2}(2 m) v_{f}^{2} \\
& =m v_{f}^{2}
\end{aligned}
$$

Andelen som tapes

$$
\begin{aligned}
& =\frac{\text { endring } i \text { kinetisle enorgi }}{\text { opprinnelig kinetish energi }} \\
& =\frac{\Delta K}{K_{\text {sys }} \text { : }} \\
& =\frac{K_{\text {oysi }}-K_{\text {yst }}}{K \text { Kysi }} \\
& =\frac{\frac{1}{2} m\left(v_{1 i}^{2}+v_{2 i}^{2}\right)-m v_{f}^{2}}{\frac{1}{2} m \alpha\left(v_{1 i}^{2}+v_{2 i}^{2}\right)}
\end{aligned}
$$

$$
\begin{aligned}
& =\frac{\frac{1}{2}\left(v_{1 i}^{2}+v_{2 i}^{2}\right)-v_{f}^{2}}{\frac{1}{2}\left(v_{1 i}^{2}+v_{2 i}^{2}\right)} \\
& =1-\frac{2 v_{f}^{2}}{v_{1 i}^{2}+v_{2 i}^{2}} \\
& =1-\frac{2 \cdot(20 \mathrm{~m} / \mathrm{s})^{2}}{(30 \mathrm{~m} / \mathrm{s})^{2}+(10 \mathrm{~m} / \mathrm{s})^{2}} \\
& =0,20
\end{aligned}
$$

$20 \%$ our den kiretiske energien tapes. Den andarres til andre energiformer san varme\\

MARKPROB

8.87) (1)\\
\includegraphics[max width=\textwidth, center]{2025_01_16_5f84a3aaf9b94e598751g-5}

$$
\begin{aligned}
& m=5.0 \mathrm{~kg} \\
& U_{i 1}=2.0 \mathrm{~m} \\
& \mathcal{U}_{12}=0 \\
& \theta_{1}=30^{\circ} \\
& \theta_{2}=60^{\circ}
\end{aligned}
$$

\includegraphics[max width=\textwidth, center]{2025_01_16_5f84a3aaf9b94e598751g-5(1)}\\
a) Legemet beveges seg med konstant fart fram til dur kolliderer med et legene: ro. Dette paivirker legenet slik at der beveger seg med virkeler $\theta_{1}$ pá horisontales

$$
\begin{aligned}
& \cos \theta_{1}=\frac{v_{1}}{v_{i 1}} \\
& \begin{aligned}
u_{1} & =v_{i 1} \cos \theta_{1} \\
& =2.0 \mathrm{~m} \cdot \cos 30^{\circ} \\
v_{1} & =1.7 \frac{\mathrm{~m}}{3}
\end{aligned}
\end{aligned}
$$

\begin{itemize}
  \item Brever vi videre:
\end{itemize}

$$
v_{1}=1.732 \mathrm{~m}
$$

\begin{itemize}
  \item Finner $v_{2}$ ved arbeid-erergi-setninger
\end{itemize}

$$
\begin{aligned}
& K_{i 1}+K_{i 2}=K_{1}+K_{2} \\
& \frac{1}{2}\left(x v_{i 1}^{2}+\frac{1}{2} m v_{i 2}^{20}=\frac{y}{2} x v_{1}^{2}+\frac{1}{2} x v_{2}^{2}\right. \\
& v_{2}=\sqrt{v_{i 1}^{2}-v_{1}^{2}} \\
& =\sqrt{\left(2 \frac{m}{5}\right)^{2}-(1.732 m)^{2}} \\
& v_{2}=1.0 \frac{m}{5}
\end{aligned}
$$

Legemure beveger seg med hasholdsvis $1.7 \%$ og 1.0 m $30^{\circ} \operatorname{\circ g} 60^{\circ}$ pá honisontalen huer sin vei\\
b)\\
stotet es dermed elastisk\\

MARKPROB

8.108)\\
\includegraphics[max width=\textwidth, center]{2025_01_16_5f84a3aaf9b94e598751g-6}

$$
m+M V_{i}
$$

$$
\begin{aligned}
& m=0.400 \mathrm{~kg} \\
& M=13 \mathrm{~kg} \\
& x=0.15 \mathrm{~m} \\
& \mu=0.40 \\
& U_{1}=0 \\
& V_{1}=0
\end{aligned}
$$

\begin{itemize}
  \item Der kinetiske enegien fra klosser gár over hil friksjonsarbeid när closser bremser opp, der friksjonsarbeidet er gitt ved
\end{itemize}

$$
W_{f}=-f x \text {, der } f=\mu l F_{n}
$$

\begin{itemize}
  \item Brwker N2L i y-retning for a hime et utryle for $F$
\end{itemize}

$$
\begin{aligned}
& \sum F_{y}=0 \\
& F_{n}-F_{g}=0 \\
& F_{n}=(m+M) g
\end{aligned}
$$

$$
\begin{aligned}
& \frac{E_{0}}{E_{1}}=\frac{\frac{1}{2} \operatorname{sp} v_{11}^{2}+\frac{1}{2} m v_{22}^{270}}{\frac{1}{2} x v_{1}^{2}+\frac{1}{2} \operatorname{m} v_{2}^{2}} \\
& =\frac{v_{i 1}^{2}}{v_{1}^{2}+v_{2}^{2}} \\
& =\frac{\left(2.0 \frac{m}{3}\right)^{2}}{\left(1.732 \frac{m}{3}\right)^{2}+\left(1.0 \frac{m}{0}\right)^{2}} \\
& \underset{E_{0}}{E_{0}}=1.0
\end{aligned}
$$

\begin{itemize}
  \item Bruker arbeid-energiz likningen for a Rime et uttrykle for $v_{0}$ i prosesser: figur (2)
\end{itemize}

$$
\begin{aligned}
& E_{1}=E_{0}+W_{f} \\
& K_{1}=K_{0}+W_{f} \\
& \frac{1}{2}(m+N) V_{1}^{2}=\frac{1}{2}(m+m) V_{0}^{2}-\mu(\infty+m) g x \\
& \frac{1}{2} V_{0}^{2}=\mu g x \\
& V_{0}
\end{aligned}=\sqrt{2 \mu g x} .
$$

\begin{itemize}
  \item Har et velastisk stoot, der dutte gir $p_{1}=p_{0}$ for prosesser: figur (1)
\end{itemize}

$$
(m+M) V_{0}=m v_{0}+M F_{1}^{0}
$$

der dutte loses mhp $v_{0}$ :

$$
\begin{aligned}
v_{0} & =\frac{m+M}{m} V_{0} \\
& =\frac{0.400 \mathrm{~kg}+13 \mathrm{~kg}}{0.400 \mathrm{~kg}} \cdot 1.085 \frac{\mathrm{~m}}{\mathrm{~s}} \\
v_{0} & =36 \mathrm{~m}
\end{aligned}
$$


\end{document}