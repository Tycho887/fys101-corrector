\documentclass[10pt]{article}
\usepackage[italian]{babel}
\usepackage[utf8]{inputenc}
\usepackage[T1]{fontenc}
\usepackage{graphicx}
\usepackage[export]{adjustbox}
\graphicspath{ {./images/} }
\usepackage{amsmath}
\usepackage{amsfonts}
\usepackage{amssymb}
\usepackage[version=4]{mhchem}
\usepackage{stmaryrd}

\begin{document}
Losningsforslag oblig 10\\

MARKPROB

13.32)\\
\includegraphics[max width=\textwidth, center]{2025_01_16_c2d628273d1fc4cd4d56g-1}

$$
\begin{aligned}
& m=1500 \mathrm{~kg} \\
& P_{y}=200 \mathrm{kPa} \\
& A=?
\end{aligned}
$$

\begin{itemize}
  \item Skal finne det totale arealet huert hjul er i kontakt med bakken. Finner dutte ved à se pà kreftene hjulene opplever fra bilen og trykeffordelingen pà hjulene
\end{itemize}

MARKPROB

$$
F_{g b_{i} 1}=m g(I)
$$

\begin{itemize}
  \item Kreftene pa huert hjul er gitt ved
\end{itemize}


\begin{align*}
& F_{\text {ghul }}=\frac{F_{\text {gbil }}}{\text { antall hjul }}, \text { og setter ion likning (I) } \\
& F_{\text {ghjul }}=\frac{m \text { g. }}{4} \text { (II) } \tag{II}
\end{align*}


\begin{itemize}
  \item Bruker trykk er gitt ved krefter per areal og loser minp A
\end{itemize}

$$
\begin{aligned}
P & =\frac{(F \text { Fhiul }}{A}, \text { og setter inn likning } \\
P & =\frac{m g}{4 A} \\
A & =\frac{m g}{4 P} \\
& =\frac{1500 \mathrm{~kg} \cdot 9.8) \frac{\mathrm{N}}{}}{4 \cdot 200 \cdot 10^{3} \mathrm{~Pa}}=0.018393 \mathrm{~m}^{2} \\
A & =184 \mathrm{~cm}^{2}
\end{aligned}
$$

Det totalet arealet til den delen au hulene som er i kontakt med bakken er $184 \mathrm{~cm}^{2}$.

$$
\begin{aligned}
& F_{g}=5.00 \mathrm{~N} \\
& T=4.55 \mathrm{~N}
\end{aligned}
$$

$$
\rho_{v}=1000 \frac{\mathrm{k}}{\mathrm{~N} \mathrm{~m}_{0}}
$$

a) Skal finne blokkens tetthet, og begynner med a finne oppdriften. Blokken er iro, og benyter derfor Newtons 1. Lou

$$
\begin{aligned}
& \sum F=0 \\
& \pi+B-F_{g}=0 \\
& B=F_{g}-T \\
& =5.00 \mathrm{~N}-4.55 \mathrm{~N} \\
& B=0.45 \mathrm{~N}
\end{aligned}
$$

\begin{itemize}
  \item Oppdriften er det samme som tyngdes til den fortrengte veesken (Arkimedes prinsipp)
\end{itemize}


\begin{align*}
B & =\text { Fffortrengt } \\
& =m_{\text {fortrengt }} \cdot g \quad\left(m_{\text {forttroggt }}=m_{v}=\rho_{v} \cdot V\right) \\
B & =\rho_{v} \cdot v \cdot g \\
V & =\frac{B}{\rho_{v} \cdot g}(I) \tag{I}
\end{align*}


\begin{itemize}
  \item Tettheten til blokken er gitt ved $\rho=\frac{m}{v}$, og setter inn likning
\end{itemize}


\begin{align*}
& =\frac{\rho_{v} \cdot m}{\beta} g \quad\left(m=\frac{F g}{g}\right)  \tag{I}\\
& =\frac{\rho_{v} \cdot G}{\beta} \\
& =\frac{1000 \frac{\mathrm{~kg}}{\mathrm{mb}} \cdot 5.00 \mathrm{~N}}{0.45 \mathrm{~N}} \\
g & =11.1 \cdot 10^{3} \frac{\mathrm{~kg}}{\mathrm{~m}}
\end{align*}


b) Tettheten til materialet blokken bestair air er omtrent $11.100 \frac{69}{m^{3}}$, som er rundt samme tetthet som bly.\\

MARKPROB

13.56)\\

\includegraphics[max width=\textwidth, center]{2025_01_16_c2d628273d1fc4cd4d56g-3}

$$
\begin{aligned}
& d=2.00 \mathrm{~cm} \\
& P_{1}^{\prime}=142 \mathrm{kPa} \\
& P_{2}^{\prime}=101 \mathrm{kPa} \\
& I_{U}=2.80 \frac{\mathrm{~L}}{5} \\
& d_{2}=?
\end{aligned}
$$

Skal finge diameteren som tillater trykkendringen. Begynne med a Rinne et utryke for faster $v_{i}$ ved likninges for volumstram


\begin{align*}
& I_{v}=A_{1} v_{1} \\
& v_{1}=\frac{4 I v}{\pi d_{1}^{2}}(I) \tag{I}
\end{align*}\left(A_{1}=\pi\left(\frac{d_{1}}{2}\right)^{2}\right)


\begin{itemize}
  \item Bruker bernoullis likning for a finne et uttryke for farten etter kompresionen, $v_{2}$
\end{itemize}

$$
\begin{aligned}
& \rho_{2}+\rho g h_{2}+\frac{1}{2} \rho v_{2}^{2}=p_{1}+\rho g h_{1}+\frac{1}{2} \rho v_{1}^{2} \quad\left(h_{2}=h_{1}\right) \\
& P_{2}+\frac{1}{2} \mathcal{J} v_{2}^{2}=P_{1}+\frac{1}{2} \rho v_{1}^{2} \\
& \rho v_{2}^{2}=2\left(p_{1}-p_{2}\right)+\rho v_{1}^{2} \\
& v_{2}=\sqrt{\frac{2\left(P_{1}-P_{2}\right)}{g}+v_{1}^{2}} \text {, og setter inn likning (I) } \\
& =\sqrt{2\left(\rho_{1}-P_{2}\right)+\left(\frac{4 I v}{\pi d_{1}^{2}}\right)^{2}}
\end{aligned}
$$

$$
\begin{aligned}
& v_{2}=12.706 \frac{\mathrm{~m}}{\mathrm{~s}}
\end{aligned}
$$

\begin{itemize}
  \item Anser vasker som inkompressibel, og berytter dermed kontinuitetslikningen for a fine et utrick for $d_{2}$
\end{itemize}

$$
\begin{aligned}
& A_{2} v_{2}=A_{1} v_{1} \quad\left(A_{2}=\pi\left(\frac{d_{2}}{2}\right)^{2}\right) \\
& \frac{\pi d_{2}}{4} d_{2}^{2} v_{2}=I v \\
& d_{2}=\sqrt{\frac{4 I v}{\pi v_{2}}} \\
& =\sqrt{\frac{4 \cdot 2.80 \frac{1}{5} \cdot \frac{1030}{100}}{\pi \cdot 12.706 \frac{m}{3}}} \\
& d_{2}=1.68 \mathrm{~cm}
\end{aligned}
$$ for at trykendringer skal fime sted.

MARKPROB

13.75)\\
\includegraphics[max width=\textwidth, center]{2025_01_16_c2d628273d1fc4cd4d56g-4}

$$
\begin{aligned}
& h=250 \mathrm{~cm} \\
& l=5.00 \mathrm{~cm} \\
& r=0.75 \mathrm{~cm} \\
& \rho_{\text {oje }}=860 \frac{\mathrm{~kg}}{n^{3}} \\
& y_{\text {olje }}=180 \mathrm{mPa} . \mathrm{s}
\end{aligned}
$$

Har at krafta fra tryklforskjellen gir ass et utrybk for $P$ gitt ved

$$
P=P_{0}+\rho g h
$$

\begin{itemize}
  \item Bruker poiseuilles lov for à finne volumstrommen
\end{itemize}

$$
\begin{aligned}
& \Delta P=\frac{8 y \text { dje } e}{\pi r^{4}} I v \quad\left(\Delta P=P-P_{0}\right) \\
& \text { Iv }=\frac{\pi\left(P_{-} P_{0}\right) r^{4}}{8 y_{\text {dje }}} \text {, og setter im libring (I) } \\
& =\frac{\pi\left(\rho_{0}+\rho_{0} j e g h-\rho_{0}\right) r^{4}}{8 y_{o j l e}} \\
& =\frac{\pi J_{0 l j e} g h r^{4}}{8 J_{\text {Jje }} l} \\
& =\frac{\pi \cdot 860 \frac{\mathrm{~kg}}{\mathrm{~kg}} \cdot 9.81 \mathrm{~kg} \cdot 250 \cdot 10^{-2} \mathrm{~m} \cdot\left(0.75 \cdot 10^{-2} \mathrm{~m}\right)^{4}}{8 \cdot 180 \cdot 10^{-3} \mathrm{~Pa} \cdot \frac{1000 \mathrm{~L}}{1 \mathrm{~m}^{3}} \cdot 5.00 \cdot 10^{-2} \mathrm{~m}} \\
& I v=2.91 \frac{\mathrm{t}}{\mathrm{~s}}
\end{aligned}
$$

Volumstrommen när oljen forst forlater tanken er $2.91 \frac{1}{5}$.


\end{document}