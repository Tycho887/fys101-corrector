\documentclass[10pt]{article}
\usepackage[swedish]{babel}
\usepackage[utf8]{inputenc}
\usepackage[T1]{fontenc}
\usepackage{amsmath}
\usepackage{amsfonts}
\usepackage{amssymb}
\usepackage[version=4]{mhchem}
\usepackage{stmaryrd}
\usepackage{graphicx}
\usepackage[export]{adjustbox}
\graphicspath{ {./images/} }

\begin{document}
FYS101-oblig 11\\

MARKPROB

En shinghing kan gererelt beskines ved

$$
x=A \cos (\omega t+\delta)
$$

Argumentet $(\omega t+\delta)$ er bevegelsens fase, durs hur ii er suinghingen. konstanten or kalles fasekonstant, dette tilsuarer fasen ved tiden $t=0$.\\
Resonanser et fenomen smapstiós rar fremvensen $犬 1$ en ytre, periodisk kraft er lik egenfrelvensen thl systemet. Da cil man fä en strorre amplitude pai sovirghingere.

Q-verdi beshrier dempede svingninger og sier nce an" "lualifeten" $\mathrm{f}_{1}$ I swinghingene. Den er relatert " til hor mye energl sam tapes $i$ en suingesyucus:

$$
Q=\frac{2 \pi}{(\Delta E l / E) \text { syncos }}
$$

Tidskonstant $\tau$ er tiden det tar feer energien til slirghingen er redusert med $e^{-1}(\sim 37 \%)$

MARKPROB

Lasmingsforslag obligil\\
14.29) Posisjonen til en partikel er gitt ved

$$
x=(7.0 \mathrm{~cm}) \cos (6 \pi t)
$$

a) Finner frekuensen Led

$$
\begin{aligned}
& u=2 \pi f, \text { og loses mhp f } \\
& f=\frac{w}{}=\frac{1}{2 \pi} \text { der unkelfrekvenser til partiklelen } \\
& =\frac{6 \pi \frac{1}{5}}{2 \pi} \\
& f=30 \mathrm{~Hz}
\end{aligned}
$$

frelevensen til partilkulen er 3.0 Hz\\
b) Peñoder Hil partikkeler er gittved

$$
\begin{aligned}
& T=\frac{1}{f} \\
& =\frac{1}{3.0 \mathrm{~Hz}} \\
& T=0.33 \mathrm{~s}
\end{aligned}
$$

Parikkeler bruker 0,335 pa én full suingong\\
c) Finner amplituden fira funksjonen som besknves posisjonen al en partikket

$$
\begin{aligned}
& x=A \cos (\omega t+\sigma), \operatorname{og} \operatorname{ser} a t \\
& A=7.0 \mathrm{~cm}
\end{aligned}
$$

amplituder fil svingningere er 7.0 cm fra likuektsposisjonen (seat' $\mathrm{cos} \theta \mid \leq 111$ )

$$
\theta=w^{\prime} t+v
$$

d) Partikleelen er ved likevekt for forste gang ettes $t=0$ ved $\cos (6 \pi t)=0$

$$
\begin{aligned}
& \cos \left(n k \pi+\frac{\pi}{2}\right)=0, \quad \operatorname{der} k=6 \text { og } n=0, \pm 1, \pm 2 \ldots \text {. } \\
& 6 \pi t=\frac{\pi}{12}
\end{aligned}
$$

BRUNNEN:IT $-t=0.0835$\\
partiduelen er i likurect etter 0.083 s etter $t=0$, og beveger seg in negativ $x$-retning\\

MARKPROB

\includegraphics[max width=\textwidth, center]{2025_01_16_a1e33228680468e1848fg-3}\\
a) Finner ïnkelfrelurensen til bevegelsen när klossen slippes

$$
w=\sqrt{\frac{k}{m}}
$$

$$
=\sqrt{\frac{700 \mathrm{~kg}}{5.00 \mathrm{~kg}}}=
$$

$\omega_{2}=11.832 \frac{1}{5}$\\
Buker relasioner mellom frekuens og\\
unnku frekuens for a fine frekuensen

$$
\begin{aligned}
& f=\frac{w}{2 \pi} \\
&=\frac{11.832 \frac{1}{2}}{2 \pi}=1.8831 \mathrm{~Hz} \\
& f=1.881 \mathrm{~Hz}
\end{aligned}
$$

frekuenser ná klossen slippes er 1.88 Hz , og sider underlacet es fiksionslost, vil Arecu-ensen bevares, altsa vare konstant\\
b) Finner perioden\\
$T=\frac{1}{f}$\\
$=\frac{11}{1.8831 \mathrm{~Hz}}$\\
$T=0.531 \mathrm{~S}$\\
Klosser baues 0.531 s pá én full suingnixa\\
c) Finner amplituder úed à denvere likningen for posisjonen (des u' Io endrer klossen rening)

$$
\begin{aligned}
& v=\frac{d x}{d t}=\frac{d}{d t}(\cos (\omega r t+\delta) \\
& v=-\omega A: \sin (\omega t+\delta 1)
\end{aligned}
$$

Mä frist finne $\delta$ og b́new at

$$
\begin{aligned}
& v(0)=v_{0}=0=7 \\
& \sin (\delta)=0 \\
& 0=0
\end{aligned}
$$

slik at

$$
\begin{aligned}
& x_{0}=A \cos (\omega \cdot 0+0) \\
& A=x_{0} \\
& A=8.00 \mathrm{~cm}
\end{aligned}
$$

amplituiden blir bestemt au huor langt blossen strekes siden us dute tilfellet har ef Priksonslost underlag, og amplituden blir 8.00 cm\\
d) Finner maksimalhastigheten ved

$$
\begin{aligned}
& v=-\omega A \sin (\omega t+\delta)] \text { der } \quad \frac{\sin \theta \mid \leq \prod 1}{\theta=\omega t+\delta} \Rightarrow \\
& v=\omega A \\
& =11.832 \frac{1}{5} \cdot 8.00 \cdot 10^{-2} \mathrm{~m} \\
& v=0.947 \mathrm{~m}
\end{aligned}
$$

maksimal farta til klossen er. 0.947 .\\
e) Finner máksmalakselesasonen ued

$$
\begin{aligned}
& a=\frac{d v}{d t}=\frac{d}{d t}(-w A \sin (w s t+\delta)) \\
& a=-\omega^{2} A \cos \left(\omega t+\sigma^{\circ}\right), d \omega|\cos \theta| \leq \| \mid=3 \\
& a=\sin ^{2} A \\
& =\left(11.832 \frac{1}{5}\right)^{2} \cdot 8.00 \cdot 10^{-2} \mathrm{~m} \\
& a=11.2 \frac{\mathrm{~m}^{2}}{2}
\end{aligned}
$$

f) Finner et urruke for tid ved $x=0$\\
$x=8.00 \mathrm{~cm} \cdot \cos \left(11.832 \frac{1}{5 t}\right)$\\
$\cos \left(11.832 \frac{1}{3} t\right)=0$\\
$11.832 \frac{1}{5} t=\cos ^{-1}(0)$\\
$11.832 \frac{1}{3} t=\frac{\pi}{2}$\\
$t=\frac{\pi}{2 \cdot 11.832 t}=0.13276 \mathrm{~s}$\\
$t=0.133 \mathrm{~s}$\\
\includegraphics[max width=\textwidth]{2025_01_16_a1e33228680468e1848fg-5} Far $x=0$ har ui $\sum F_{x}=0$ dus at is iht Nedutens 2J lar limbe hat inven alasebras,\\
$a=0$\\

MARKPROB

14.85) Finner resonans Rekuens for elastisk pendel.\\
a) def $k=400 \frac{m}{m}$ og m $=10 \mathrm{~kg}$, $f=\frac{v}{2 \pi}$\\
$=\frac{1}{2 \pi} \sqrt{\frac{k}{r}}$\\
$=\frac{1}{2 \pi} \cdot \sqrt{\frac{400 \mathrm{~m}}{10 \mathrm{~kg}}}$\\
$\frac{f=1.0 \mathrm{~Hz}}{}$\\
b) Elastisk pendel der $k=800 \mathrm{M} \circ \mathrm{m} . m=5.0 \mathrm{~kg}$ $f=\frac{u}{2 \pi}$\\
$=\frac{1}{2 \pi} \sqrt{\frac{k}{m}}$\\
$=\frac{1}{2 \pi} \cdot \sqrt{\frac{800 \mathrm{~N}}{5.0 \mathrm{~kg}}}$\\
$f=2.0 \mathrm{~Hz}$

$$
\text { c) Matenatisk pendel des } L=2,0 m, m=40 \mathrm{~kg} \text { og }
$$

MARKPROB

14:87)\\
a) Finner unkelfreküensen til blolkens suingning

$$
\begin{aligned}
w_{0} & =\sqrt{\frac{k}{n}} \\
& =\sqrt{\frac{400 \frac{N}{m}}{200 \mathrm{~kg}}}
\end{aligned}
$$

$$
\omega_{6}=14.14 \frac{1}{5},
$$

og bruker denne til á finne a mplituder\\
ti) sungningere

$$
A=4,98 \mathrm{~m}
$$

b) Resonans oppstär ved den naturlige frekuensen $\omega_{0}=14,1 \frac{\mathrm{ad}}{5}$\\
ved en vinkel felvers pa 14, rad vil suingningene\\
oppád en resonans'freuens

$$
\begin{aligned}
& A=\frac{F_{a}}{\sqrt{m^{2}\left(\omega_{0}^{2}-w_{d}^{2}\right)^{2}+(b \omega \sqrt{a})^{2}}}
\end{aligned}
$$

\begin{center}
\includegraphics[max width=\textwidth]{2025_01_16_a1e33228680468e1848fg-7}
\end{center}


\end{document}